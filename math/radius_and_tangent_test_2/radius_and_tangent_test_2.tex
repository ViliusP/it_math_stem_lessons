\mathchardef\period=\mathcode`.
\documentclass[a4paper]{article}
\usepackage[top=1.45cm, bottom=1cm, left=3cm, right=3cm]{geometry}

\usepackage{graphicx}
\usepackage[export]{adjustbox}
\usepackage{caption}

% Lithuanian language
\usepackage[L7x,T1]{fontenc}
\usepackage[utf8]{inputenc}
% ----------------------------------

% Tikz 
\usepackage{tikz} % Package for drawing
\usepackage{tkz-euclide}

\usetikzlibrary{fit,positioning}
\usetikzlibrary{arrows.meta}
\usetikzlibrary{patterns,patterns.meta}
\usetikzlibrary{intersections, through}
\usetikzlibrary{calc}
% -----------------------------------

% Layout
\usepackage{fancyhdr}
\usepackage{adjustbox}
\usepackage{graphicx}
\usepackage[inline]{enumitem}
% German these -> ""
\newcommand{\germanqq}[1]{{\selectlanguage{german}\glqq#1\grqq\selectlanguage{english}}}
% ----------------------------------------------------

% Lithuanian math things 
\usepackage{amsmath,amssymb}
\usepackage{icomma}
\usepackage{siunitx}

\DeclareMathOperator{\tg}{tg}
\newcommand{\tgx}{\tg x}

\DeclareMathOperator{\arctg}{arctg}
\newcommand{\arctgx}{\arctg x}
% -----------------------------------------------------------

% Utilities
\usepackage{import}
% ----------------------------

\usepackage{kantlipsum}   % Dummy text

% Meta data
\title{Kontrolinis darbas - spindulys ir liestinė}
\author{Vilius Paliokas}
\date{2024/05/08}
% -----------------------------------------------

% Fancy header
% Header | Footer 
\fancyhf{} % clear all header and footer fields
\fancyhead[R]{Spindulio ir liestinės savybės | kontrolinis darbas}
% L for Left, you can also use R for Right or C for Center
\fancyfoot[R]{Spindulio ir liestinės savybės | kontrolinis darbas}

% L for Left, you can also use R for Right or C for Center
\setlength{\headheight}{0.5pt} % Adjust the head height
\renewcommand{\headrulewidth}{0.4pt} % Line under the header
\renewcommand{\footrulewidth}{0.4pt} % Line above the footer
% Header | Footer 

% -----------------------------------------------

\begin{document}
% First page layout
\newcommand\blankpage{%
      \null
      \thispagestyle{empty}%
      \addtocounter{page}{-1}%
      \newpage
}

\thispagestyle{fancy}
% -------------------------

% tikz style
\tikzstyle{every node}=[font=\LARGE]
% ------------------------

\subsection*{1 variantas}
\vspace*{5mm}

\subsubsection*{1 užduotis}
\begin{minipage}[t]{0.22\textwidth}
      \resizebox{\linewidth}{!}{%
            \subimport{plots/}{plot2.tex}
      }
      \centering
\end{minipage}\hfill
\begin{minipage}{0.69\textwidth}

      (\textit{1 taškas}) Taškas $O$ yra apskritimo centas, $AB$ - skersmuo,
      $OB = BC$. Kam lygu $\frac{x}{6}$?
\end{minipage}

\subsubsection*{2 užduotis}

\begin{minipage}[t]{0.22\textwidth}
      \resizebox{\linewidth}{!}{%
            \subimport{plots/}{plot4.tex}
      }
      \centering
\end{minipage}\hfill
\begin{minipage}{0.69\textwidth}
      (\textit{1 taškas}) Apskritimo skersmenys $AB$ ir $CD$ kertasi taške $O$.
      Apskritimo spindulio ilgis lygus $6\frac{2}{3}$. Apskaičiuokite
      paryškintų
      lankų ilgių sumą.
\end{minipage}

\subsubsection*{3 užduotis}

\begin{minipage}[t]{0.28\textwidth}
      \resizebox{\linewidth}{!}{%
            \subimport{plots/}{plot8.tex}
      }
      \centering
\end{minipage}\hfill
\begin{minipage}{0.69\textwidth}
      (\textit{1 taškas}) Per tašką $N$ nubrėžtos dvi apskritimo liestinės.
      Paryškinto lanko dydis lygus $260^\circ$. Apskaičiuokite kampo $x$ dydį.
\end{minipage}

\subsubsection*{4 užduotis}

\begin{minipage}[t]{0.27\textwidth}
      \resizebox{\linewidth}{!}{%
            \subimport{plots/}{plot9.tex}
      }
      \centering
\end{minipage}\hfill
\begin{minipage}{0.69\textwidth}
      (\textit{1 taškas}) $AC$ ir $AD$ yra apskritimo liestinės, $\angle ACB =
            38^\circ$. Apskaičiuokite kampo $AOB$ dydį.
\end{minipage}

\subsubsection*{5 užduotis}

\begin{minipage}[t]{0.25\textwidth}
      \resizebox{\linewidth}{!}{%
            \subimport{plots/}{plot6.tex}
      }
      \centering
\end{minipage}\hfill
\begin{minipage}{0.69\textwidth}
      (\textit{1 taškas}) Keturkampis 
\end{minipage}

\subsubsection*{6 užduotis}

\begin{minipage}[t]{0.22\textwidth}
      \resizebox{\linewidth}{!}{%
            \subimport{plots/}{plot10.tex}
      }
      \centering
\end{minipage}\hfill
\begin{minipage}{0.69\textwidth}
      (\textit{1 taškas}) Keturkampis 
\end{minipage}

\vfill
\begin{small}
      \begin{enumerate*}[label={(\arabic*)}]
            \item \textbf{Visur} \textbf{nurodykite atsakymus} ($Ats\ldots$);
            \item Jokio sukčiavimo. Negalima naudotis užrašais, vadovėliais,
            elektroniniais prietaisais;
            \item Jokio kalbėjimo;
            \item Rašyti aiškiai, nedviprasmiškai;
            \item Galima naudotis tik savo skaičiuotuvu ir formulių lapu;
      \end{enumerate*}
\end{small}

\end{document}
