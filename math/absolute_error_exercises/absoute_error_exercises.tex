\documentclass[32pt,a4paper]{article}

\usepackage{fullpage} % Package to use full page
\usepackage{parskip} % Package to tweak paragraph skipping
\usepackage{tikz} % Package for drawing
\usepackage{tkz-euclide}

\usepackage[margin=.25cm]{geometry}

\usepackage{amsmath,amssymb}

\usepackage{amsmath}
\usepackage{hyperref}
\usepackage[main=lithuanian, german, shorthands=off]{babel}
\usepackage{tgpagella}
\usepackage[L7x,T1]{fontenc}
\usepackage[utf8]{inputenc}
\usepackage{enumitem}
\usepackage{booktabs} % For better looking tables
\usepackage{venndiagram}
\usepackage{subfig}
\newcommand{\germanqq}[1]{{\selectlanguage{german}\glqq#1\grqq\selectlanguage{english}}}

\title{Absoliučioji paklaida. Uždaviniai.}
\author{Vilius Paliokas}
\date{2023/10/12}

\begin{document}
\maketitle

\section*{Teorija}

Absoliučiosios paklaidos formulė:

$\Delta x=x-a$;

$\Delta x$ - absoliučioji paklaida;

$x$ - tikslus skaičius;

$a$ - artinys (apytikslė reikšmė);

\section*{Uždaviniai}

\begin{enumerate}
    \item Deguonies tankis yra $1,43 kg/m^{3}$, o vandenilio tankis yra $0,09
              kg/m^3$. Suapvalinkite šiuos skaičius iki dešintųjų ir raskite
          kiekvienos
          gautos apytikslės reikšmės absoliučiąją paklaidą.
    \item Benzino tankis yra $0,71 g/cm^{3}$, o gyvsidabrio tankis yra $13,55
              g/cm^{3}$. Suapvalinkite šiuos skaičius iki dešimtųjų ir raskite kiekvienos
          gautos apytikslės reikšmės santykinę paklaidą.
    \item Mokinys išmatavo 15 cm ilgio pieštuką, tačiau tikrasis jo ilgis buvo
          14,8 cm. Raskite absoliučiąją paklaidą.
    \item Eksperimento metu išmatuotas šviesos greitis buvo $3.00 \cdot 10^{8}$
          m/s, o tikrasis šviesos greitis yra $2.998 \cdot 10^{8}$ m/s.
          Apskaičiuokite absoliučiąją paklaidą.
    \item Du mokslininkai išmatavo gravitacinį pagreitį. Mokslininkas A
          išmatavo $9.80 m/s^{2}$, mokslininkas B išmatavo $9.78 m/s^2$.
          Tikrasis
          gravitacinis pagreitis yra $9.81 m/s^2$. Kas buvo tikslesnis ir kokia
          absoliuti paklaida abiem atvejais?
    \item Buvo apskaičiuota, kad erdvėlaivis nusileis už 1200 km. nuo
          starto vietos, tačiau iš tikrųjų jis nusileido už 1185 km. Raskite
          absoliučiąją paklaidą ir išreikškite ją procentais nuo tikrojo
          nusileidimo
          atstumo.
\end{enumerate}

\end{document}