\documentclass[a4paper]{article}
\usepackage[top=1.5cm, bottom=1.5cm, left=1cm, right=1cm]{geometry}

\usepackage{parskip} % Package to tweak paragraph skipping
\usepackage{tikz} % Package for drawing
\usepackage{tkz-euclide}
\usetikzlibrary{fit,positioning}
\usetikzlibrary{arrows.meta}
\usetikzlibrary{patterns,patterns.meta}
\usepackage[inline]{enumitem}
\usepackage{amsmath,amssymb}
\usepackage{tasks}
\usepackage{amsmath}
\usepackage{hyperref}
\usepackage[main=lithuanian, german, shorthands=off]{babel}
\usepackage{tgpagella}
\usepackage[L7x,T1]{fontenc}
\usepackage[utf8]{inputenc}
\usepackage{enumitem}
\usepackage{booktabs} % For better looking tables
\usepackage{venndiagram}
\usepackage{subfig}
\usepackage{multirow}
\newcommand{\germanqq}[1]{{\selectlanguage{german}\glqq#1\grqq\selectlanguage{english}}}

\title{Kontrolinis darbas nr. 1}
\author{Vilius Paliokas}
\date{2023/10/17}

\setlist{after=\vspace{\baselineskip}}

\begin{document}
\thispagestyle{empty}
\section*{1 variantas}

\begin{enumerate}
      \item Raskite $x$ reikšmę su kuria lygybė yra teisinga:

            \begin{tasks}[item-format={\normalfont}, after-item-skip=4mm](5)
                  \task $4^{x}=16$;
                  \task $2^{x}=\frac{1}{8}$;
                  \task $2^{x}=7$;
                  \task $2^{x}=\sqrt[5]{2}$;
                  \task $92^{x}=1$;
            \end{tasks}

      \item Raskite $x$  reikšmę su kuria lygybė yra teisinga:
            \begin{tasks}[item-format={\normalfont}, after-item-skip=4mm](5)
                  \task $\log_{3} x=4$;
                  \task $\lg x=\frac{1}{11}$;
                  \task $\log_{8} x=-\frac{1}{2}$;
            \end{tasks}
      \item Raskite $x$  reikšmę su kuria reiškinys turi reikšmę:
            \begin{tasks}[item-format={\normalfont}, after-item-skip=4mm](4)
                  \task $\log_{13} (x+2)$;
                  \task $\sqrt{x+1}$;
                  \task $\sqrt[5]{x^{2}+4}$;
                  \task $\log_{x}88$;
            \end{tasks}
      \item Apskaičiuokite reiškinių reikšmes:
            \begin{tasks}[item-format={\normalfont}, after-item-skip=4mm](2)
                  \task $-\left|\log_{15}1+1-\sqrt{3}\right|+2\sqrt{3}$;
                  \task $5^{-2} - ((3 \cdot \log_{8}\frac{1}{8}-\sqrt{0,01}):(-2))$;
            \end{tasks}

      \item Pašalinkite iracionalumą šaknyje: $\frac{6\sqrt{3}+3}{\sqrt{3}}$.

      \item Turime aibę $A$, kuri lygi nelygybės $-2x-5\geq5$ sprendinių aibei
            ir aibę $B=(-\infty;12]$. Raskite:
            \begin{tasks}[item-format={\normalfont}, after-item-skip=2mm](1)
                  \task $B \setminus A$;
                  \task Aibę $C$, kurią sudaro visi aibės $A$ natūralieji skaičiai;
                  \task Tris aibės $C$ poaibius;
            \end{tasks}
\end{enumerate}

\begin{table}[!htpb]
      \centering
      \begin{tabular}{|cccccccccccccccccc|}
            \hline
            \multicolumn{18}{|c|}{Užduočių vertės}

            \\ \hline
            \multicolumn{5}{|c|}{1.}
                                                    &
            \multicolumn{3}{c|}{2.}                 &
            \multicolumn{4}{c|}{3.}
                                                    & \multicolumn{2}{c|}{4.} &
            \multicolumn{1}{c|}{5.}                 & \multicolumn{3}{c|}{6.}
            \\ \hline
            \multicolumn{1}{|c|}{a)}                & \multicolumn{1}{c|}{b)} &
            \multicolumn{1}{c|}{c)}                 & \multicolumn{1}{c|}{d)} & \multicolumn{1}{c|}{e)} &
            \multicolumn{1}{c|}{a)}                 & \multicolumn{1}{c|}{b)} & \multicolumn{1}{c|}{c)} &
            \multicolumn{1}{c|}{a)}                 & \multicolumn{1}{c|}{b)} & \multicolumn{1}{c|}{c)} &
            \multicolumn{1}{c|}{d)}                 & \multicolumn{1}{c|}{a)} & \multicolumn{1}{c|}{b)} &
            \multicolumn{1}{c|}{\multirow{2}{*}{1}} & \multicolumn{1}{c|}{a)} &
            \multicolumn{1}{c|}{b)}                 & c)                                                  \\
            \multicolumn{1}{|c|}{1}                 & \multicolumn{1}{c|}{1}  &
            \multicolumn{1}{c|}{1}                  & \multicolumn{1}{c|}{1}  & \multicolumn{1}{c|}{1}  &
            \multicolumn{1}{c|}{1}                  & \multicolumn{1}{c|}{1}  & \multicolumn{1}{c|}{1}  &
            \multicolumn{1}{c|}{1}                  & \multicolumn{1}{c|}{1}  & \multicolumn{1}{c|}{1}  &
            \multicolumn{1}{c|}{1}                  & \multicolumn{1}{c|}{1}  & \multicolumn{1}{c|}{1}  &
            \multicolumn{1}{c|}{}                   & \multicolumn{1}{c|}{1}  &
            \multicolumn{1}{c|}{1}                  & 1                                                   \\ \hline
      \end{tabular}
\end{table}
\bigskip

\section*{2 variantas}

\begin{enumerate}
      \item Raskite $x$ reikšmę su kuria lygybė yra teisinga:

            \begin{tasks}[item-format={\normalfont}, after-item-skip=4mm](5)
                  \task $3^{x}=81$;
                  \task $2^{x}=\frac{1}{32}$;
                  \task $2^{x}=9$;
                  \task $2^{x}=\sqrt[3]{2}$;
                  \task $133^{x}=1$;
            \end{tasks}

      \item Raskite $x$  reikšmę su kuria lygybė yra teisinga:
            \begin{tasks}[item-format={\normalfont}, after-item-skip=4mm](3)
                  \task $\log_{5} x=3$;
                  \task $\lg x=\frac{1}{10}$;
                  \task $\log_{16} x=-\frac{1}{2}$;
            \end{tasks}

      \item Raskite $x$  reikšmę su kuria reiškinys turi reikšmę:
            \begin{tasks}[item-format={\normalfont}, after-item-skip=4mm](4)
                  \task $\lg (x+1)$;
                  \task $\sqrt{x+2}$;
                  \task $\sqrt[5]{x^{2}+16}$;
                  \task $\log_{x}99$;
            \end{tasks}

      \item Apskaičiuokite reiškinių reikšmes:
            \begin{tasks}[item-format={\normalfont}, after-item-skip=10mm](2)
                  \task $4\sqrt{2}-\left|\log_{2}32-8\sqrt{2}\right|$;
                  \task $0,2 \cdot
                        ((\log_{8}1+\sqrt[5]{\frac{1}{32}}):2^{-5}+0,4^2)$.
            \end{tasks}
      \item Pašalinkite iracionalumą šaknyje: $\frac{6\sqrt{3}}{\sqrt{3}+3}$.
      \item Turime aibę $A$, kuri lygi nelygybės $-2x-5>5$ sprendinių aibei ir
            aibę $B=(-\infty;4]$. Raskite:
            \begin{tasks}[item-format={\normalfont}, after-item-skip=2mm](1)
                  \task $A \setminus B$;
                  \task Aibę $C$, kurią sudaro visi aibės B natūralieji skaičiai;
                  \task Tris aibės $C$ poaibius;
            \end{tasks}

\end{enumerate}

\begin{table}[!htpb]
      \centering
      \begin{tabular}{|cccccccccccccccccc|}
            \hline
            \multicolumn{18}{|c|}{Užduočių vertės}

            \\ \hline
            \multicolumn{5}{|c|}{1.}
                                                    &
            \multicolumn{3}{c|}{2.}                 &
            \multicolumn{4}{c|}{3.}
                                                    & \multicolumn{2}{c|}{4.} &
            \multicolumn{1}{c|}{5.}                 & \multicolumn{3}{c|}{6.}
            \\ \hline
            \multicolumn{1}{|c|}{a)}                & \multicolumn{1}{c|}{b)} &
            \multicolumn{1}{c|}{c)}                 & \multicolumn{1}{c|}{d)} & \multicolumn{1}{c|}{e)} &
            \multicolumn{1}{c|}{a)}                 & \multicolumn{1}{c|}{b)} & \multicolumn{1}{c|}{c)} &
            \multicolumn{1}{c|}{a)}                 & \multicolumn{1}{c|}{b)} & \multicolumn{1}{c|}{c)} &
            \multicolumn{1}{c|}{d)}                 & \multicolumn{1}{c|}{a)} & \multicolumn{1}{c|}{b)} &
            \multicolumn{1}{c|}{\multirow{2}{*}{1}} & \multicolumn{1}{c|}{a)} &
            \multicolumn{1}{c|}{b)}                 & c)                                                  \\
            \multicolumn{1}{|c|}{1}                 & \multicolumn{1}{c|}{1}  &
            \multicolumn{1}{c|}{1}                  & \multicolumn{1}{c|}{1}  & \multicolumn{1}{c|}{1}  &
            \multicolumn{1}{c|}{1}                  & \multicolumn{1}{c|}{1}  & \multicolumn{1}{c|}{1}  &
            \multicolumn{1}{c|}{1}                  & \multicolumn{1}{c|}{1}  & \multicolumn{1}{c|}{1}  &
            \multicolumn{1}{c|}{1}                  & \multicolumn{1}{c|}{1}  & \multicolumn{1}{c|}{1}  &
            \multicolumn{1}{c|}{}                   & \multicolumn{1}{c|}{1}  &
            \multicolumn{1}{c|}{1}                  & 1                                                   \\ \hline
      \end{tabular}
\end{table}

\end{document}