\documentclass[a4paper]{article}

\usepackage{fullpage} % Package to use full page
\usepackage{parskip} % Package to tweak paragraph skipping
\usepackage{tikz} % Package for drawing
\usepackage{tkz-euclide}
\usetikzlibrary{fit,positioning}
\usetikzlibrary{arrows.meta}
\usetikzlibrary{patterns,patterns.meta}
\usepackage[inline]{enumitem}
\usepackage{amsmath,amssymb}
\usepackage{tasks}
\usepackage{amsmath}
\usepackage{hyperref}
\usepackage[main=lithuanian, german, shorthands=off]{babel}
\usepackage{tgpagella}
\usepackage[L7x,T1]{fontenc}
\usepackage[utf8]{inputenc}
\usepackage{enumitem}
\usepackage{booktabs} % For better looking tables
\usepackage{venndiagram}
\usepackage{subfig}
\newcommand{\germanqq}[1]{{\selectlanguage{german}\glqq#1\grqq\selectlanguage{english}}}

% Definition of circles
\def\firstcircle{(0,0) circle (1.5cm)}
\def\secondcircle{(0:2cm) circle (1.5cm)}

\colorlet{circle edge}{blue!50}
\colorlet{circle area}{blue!20}

\tikzset{filled/.style={fill=circle area, draw=circle edge, thick},
      outline/.style={draw=circle edge, thick}}

\setlength{\parskip}{5mm}

\tikzset{
      venn box/.style={
                  draw=black, very thick,
                  rounded corners=10,
                  inner xsep=10pt, inner ysep=15pt, outer ysep=5pt
            },
      venn numbers/.style={
                  %    draw,
                  inner ysep=0pt,
                  align=center
            },
      venn title/.style={
                  fill=black, text=white
            }
}

\title{Kontrolinis darbas nr. 1}
\author{Vilius Paliokas}
\date{2023/10/17}

\begin{document}
\thispagestyle{empty} 
\begin{enumerate}
      \item Raskite $x$ reikšmę su kuria lygybė yra teisinga:
      
      \begin{tasks}[item-format={\normalfont}, after-item-skip=4mm](5)
            \task $4^{x}=16$;
            \task $2^{x}=\frac{1}{8}$;
            \task $2^{x}=7$;
            \task $2^{x}=\sqrt[5]{2}$;
            \task $92^{x}=1$;
      \end{tasks}

      \item Raskite $x$  reikšmę su kuria lygybė yra teisinga:
      \begin{tasks}[item-format={\normalfont}, after-item-skip=4mm](5)
            \task $\log_{3} x=4$;
            \task $\lg x=\frac{1}{11}$;
            \task $\log_{8} x=-\frac{1}{2}$;
      \end{tasks}
      \item Raskite $x$  reikšmę su kuria reiškinys turi reikšmę:
      \begin{tasks}[item-format={\normalfont}, after-item-skip=4mm](3)
            \task $\log_{13} (x+2)$;
            \task $\sqrt{x+1}$;
            \task $\sqrt[5]{x^{2}+4}$;
      \end{tasks}
      \item Apskaičiuokite reiškinių reikšmes:
      \begin{tasks}[item-format={\normalfont}, after-item-skip=10mm](2)
            \task $-\left|\log_{15}1+1-\sqrt{3}\right|+2\sqrt{3}$;
            \task $5^{-2} - ((3 \cdot \log_{8}\frac{1}{8}-\sqrt{0,01}):(-2))$;
      \end{tasks}
\end{enumerate} 

\clearpage
\thispagestyle{empty} 
\begin{enumerate}
      \item Raskite $x$ reikšmę su kuria lygybė yra teisinga:
    
      \begin{tasks}[item-format={\normalfont}, after-item-skip=4mm](5)
            \task $3^{x}=81$;
            \task $2^{x}=\frac{1}{32}$;
            \task $2^{x}=9$;
            \task $2^{x}=\sqrt[3]{2}$;
            \task $133^{x}=1$;
      \end{tasks}

      \item Raskite $x$  reikšmę su kuria lygybė yra teisinga:
      \begin{tasks}[item-format={\normalfont}, after-item-skip=4mm](3)
            \task $\log_{5} x=3$;
            \task $\lg x=\frac{1}{10}$;
            \task $\log_{16} x=-\frac{1}{2}$;
      \end{tasks}
      
      \item Raskite $x$  reikšmę su kuria reiškinys turi reikšmę:
      \begin{tasks}[item-format={\normalfont}, after-item-skip=4mm](3)
            \task $\lg (x+1)$;
            \task $\sqrt{x+2}$;
            \task $\sqrt[5]{x^{2}+16}$;
      \end{tasks}

      \item Apskaičiuokite reiškinių reikšmes:
      \begin{tasks}[item-format={\normalfont}, after-item-skip=10mm](2)
            \task $4\sqrt{2}-\left|\log_{2}32-8\sqrt{2}\right|$;
            \task $0,2 \cdot ((\log_{8}1+\sqrt[5]{\frac{1}{32}}):2^{-5}+0,4^2)$;
      \end{tasks}
\end{enumerate} 

\end{document}