\documentclass[a4paper]{article}
\usepackage[top=1.45cm, bottom=1.45cm, left=1cm, right=1cm]{geometry}

\usepackage{parskip} % Package to tweak paragraph skipping
\usepackage{tikz} % Package for drawing
\usepackage{tkz-euclide}
\usepackage{siunitx}
\usepackage{wrapfig}
\usepackage{graphicx}

\usetikzlibrary{fit,positioning}
\usetikzlibrary{arrows.meta}
\usetikzlibrary{patterns,patterns.meta}
\usepackage[inline]{enumitem}
\usepackage{amsmath,amssymb}
\usepackage{tasks}
\usepackage{amsmath}
\usepackage{hyperref}
\usepackage[main=lithuanian, german, shorthands=off]{babel}
\usepackage{tgpagella}
\usepackage[L7x,T1]{fontenc}
\usepackage[utf8]{inputenc}
\usepackage{enumitem}
\usepackage{booktabs} % For better looking tables
\usepackage{venndiagram}
\usepackage{subfig}
\usepackage{multirow}
\usepackage{tabularray}
\usepackage{lipsum}
\usepackage{fancyhdr}

\usepackage{blindtext}
\usepackage{adjustbox}
\AfterEndEnvironment{wrapfigure}{\setlength{\intextsep}{0mm}}

% Header | Footer 
\fancyhf{} % clear all header and footer fields
\fancyhead[R]{pagrindinė trigonometrinė tapatybė | sinusų teorema | kosinusių
      teorema | trikampio plotas | planimetrija}
% L for Left, you can also use R for Right or C for Center
\fancyfoot[R]{pagrindinė trigonometrinė tapatybė | sinusų teorema | kosinusių
      teorema | trikampio plotas | planimetrija}
% L for Left, you can also use R for Right or C for Center
\setlength{\headheight}{0.5pt} % Adjust the head height
\renewcommand{\headrulewidth}{0.4pt} % Line under the header
\renewcommand{\footrulewidth}{0.4pt} % Line above the footer
% Header | Footer 

\newcommand{\germanqq}[1]{{\selectlanguage{german}\glqq#1\grqq\selectlanguage{english}}}

\DeclareMathOperator{\tg}{tg}
\newcommand{\tgx}{\tg x}

\DeclareMathOperator{\arctg}{arctg}
\newcommand{\arctgx}{\arctg x}

\title{Kontrolinis darbas nr. 2}
\author{Vilius Paliokas}
\date{2023/10/17}

\setlist{after=\vspace{\baselineskip}}

% Title spacing
\usepackage{titlesec}
\titlespacing*{\subsection}{0pt}{\baselineskip}{0.5\baselineskip}
% ------------------------ 

\begin{document}
\thispagestyle{fancy}
\subsection*{1 variantas}

\begin{enumerate}
      \item Užduočių sprendimo pavyzdys. Stačiojo trikampio dviejų statinių
            ilgiai yra 3 ir 4. Apskaičiuokite įžambinės ilgį.

            \begin{enumerate}

                  \item

                        \begin{adjustbox}{minipage={\linewidth}, valign=t}

                              \begin{wrapfigure}{t}{0.17\linewidth}

                                    \begin{resizebox}{0.17\textwidth}{!}{
                                                \begin{tikzpicture}
    \coordinate (a) at (0,0);
    \coordinate (b) at (4,0);
    \coordinate (c) at (4,5);

    \draw (a) -- (b) node[midway, below]{\large 3} -- (c) node[midway, right]{\large 4} -- (a); % Larger labels

    \draw (a) node[anchor=east, align=center] {\Large A};
    \draw (b) node[anchor=west, align=center] {\Large B};
    \draw (c) node[anchor=south] {\Large C};
\end{tikzpicture}

  }
                                    \end{resizebox}
                                    \vspace{-2\baselineskip}

                              \end{wrapfigure}

                              \vspace*{0.15em}

                              Pirmiausia nusibrėžiamas trikampis ir suteikiamos
                              raidės viršunėms arba kraštinėmis naudojantis
                              jūsų vardo ar pavardės raidėmis.
                        \end{adjustbox}

                  \item \parbox{0.8\textwidth}{
                              Tada užrašoma naudojama teorema (jeigu naudojama
                              teorema, turi būti parašyta: \germanqq{pagal
                                    ...}) ar formulė:
                        }

                        \begin{minipage}{0.5\textwidth}
                              Pagal sinusų teorema: $VL^{2}=LI^{2}+LI^{2}$.

                        \end{minipage}

                  \item \parbox{0.65\textwidth}{
                              Toliau galima išsireikšti ieškomą kraštinę iš
                              raidinio reiškinio arba sustatyti turimas
                              reikšmes:
                        }

                              $$VL^{2}=3^{2}+4^{2};$$
                              $$VL^{2}=9+16;$$
                              $$VL^{2}=25;$$
                              $$VL=\sqrt{25}=5;$$
                              $$\text{Ats.:} \; 25$$.
                  \item Už teisingą teoremos pritaikymą užduočiai skiriami 
                        
            \end{enumerate}

            % \begin{enumerate}
            %       \item Pirmiausiai nusibraižome trikampį ir priskiriame raides
            %             kraštinėms ar viršunėms
            %             \textbf{naudojant betkurias savo vardo ir pavardės
            %                   raides}.

            %       \item 	\begin{minipage}[t]{\linewidth}
            %                   \lipsum[1]
            %                   \begin{wrapfigure}{l}{0.3\textwidth}
            %                         % 'r' for right side, '5cm' for the width of the wrap area
            %                         \begin{tikzpicture}[
            %                                     my angle/.style = {draw,
            %                                                 fill=teal!30,
            %                                                 angle radius=7mm,
            %                                                 angle
            %                                                 eccentricity=1.1,
            %                                                 right, inner
            %                                                 sep=1pt,
            %                                                 font=\footnotesize}
            %                               ]
            %                               \draw (0,0)
            %                               coordinate[label=below:$A$] (a) --
            %                               (4,0) coordinate[label=below:$C$] (c)
            %                               --
            %                               (4,4) coordinate[label=above:$B$] (b)
            %                               --
            %                               cycle;
            %                               % \pic[my angle, "$\alpha=\SI{45}{\degree}$"] {angle =
            %                               %       c--a--b};
            %                         \end{tikzpicture}
            %                   \end{wrapfigure}
            %             \end{minipage}

            %       \item bla baselineskip
            %       \item bla bla
            %       \item bla
            % \end{enumerate}

      \item Apskaičiuokite trikampio kraštinės ilgį vieno metro tikslumu, kai
            viena trikampio kraštinė lygi 120 metrų, kita 200 metrų, o kampas
            tarp jų lygus
            $128^\circ$.

      \item Bla bla bla:

      \item Kuriame ketvirtje yra posūkio kampas $\alpha$, jeigu:

      \item Apskaičiuokite $\sin \alpha$, $\cos \alpha$, $\tg \alpha$, kai

\end{enumerate}

\begin{table}[!htpb]
      \centering
      \begin{tblr}{
                  cell{1}{1} = {c=18}{},
                  cell{2}{1} = {c=3}{},
                  cell{2}{4} = {c=2}{},
                  cell{2}{6} = {c=4}{},
                  cell{2}{10} = {c=2}{},
                  cell{2}{12} = {c=4}{},
                  cell{2}{16} = {c=3}{},
                  hlines,
                  vlines,
            }
            Užduočių vertė &	   &	     &	       &	 &	   &
            &	      & 	&	 &	  &	   &
            &	     &	      &        &	&	 \\
            1		 &	   &	     & 2       &	 & 3	   &
            &	      & 	& 4	 &	  & 5	   &
            &	     &	      & 6      &	&	 \\
            {a\\ 1}	 & {b\\ 1} & {c\\ 2} & {a\\ 2} & {b\\ 2} & {a\\ 3} &
            {b\\			    4} & {c\\ 4} & {d\\ 4} & {a\\3} &
            {b\\3} & {a\\2} &
            {b\\2} & {c\\2} & {d\\2} & {a\\1} & {b\\1} & {c\\1}
      \end{tblr}
\end{table}

\begin{small}
      \begin{enumerate*}[label={(\arabic*)}]
            \item \textbf{Visur užrašykite atsakymus} ($Ats\ldots$);
            \item Jokio sukčiavimo. Negalima naudotis užrašais, vadovėliais,
            elektroniniais prietaisais;
            \item Jokio kalbėjimo;
            \item Rašyti aiškiai, nedviprasmiškai;
            \item Galima naudotis tik savo skaičiuotuvu ir formulių lapu;
      \end{enumerate*}
\end{small}

\end{document}