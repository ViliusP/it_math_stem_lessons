\mathchardef\period=\mathcode`.
\documentclass[a4paper]{article}
\usepackage[top=1.45cm, bottom=1cm, left=1cm, right=1cm]{geometry}

\usepackage{parskip} % Package to tweak paragraph skipping
\usepackage{tikz} % Package for drawing
\usepackage{tkz-euclide}
\usepackage{siunitx}
\usepackage{wrapfig}
\usepackage{graphicx}

\usetikzlibrary{fit,positioning}
\usetikzlibrary{arrows.meta}
\usetikzlibrary{patterns,patterns.meta}
\usepackage[inline]{enumitem}
\usepackage{amsmath,amssymb}
\usepackage{tasks}
\usepackage{amsmath}
\usepackage{hyperref}
\usepackage[main=lithuanian, german, shorthands=off]{babel}
\usepackage{tgpagella}
\usepackage[L7x,T1]{fontenc}
\usepackage[utf8]{inputenc}
\usepackage{enumitem}
\usepackage{booktabs} % For better looking tables
\usepackage{venndiagram}
\usepackage{subfig}
\usepackage{multirow}
\usepackage{tabularray}
\usepackage{lipsum}
\usepackage{fancyhdr}

\usepackage{blindtext}
\usepackage{adjustbox}
\AfterEndEnvironment{wrapfigure}{\setlength{\intextsep}{0mm}}

\usepackage{icomma}

% Header | Footer 
\fancyhf{} % clear all header and footer fields
\fancyhead[R]{skaičių seka | aritmetinė progresija | aritmetinės progresijos
      \textit{n} narių suma | savarankiškas darbas}
% L for Left, you can also use R for Right or C for Center
\fancyfoot[R]{skaičių seka | aritmetinė progresija | aritmetinės progresijos
      \textit{n} narių suma | savarankiškas darbas}

% L for Left, you can also use R for Right or C for Center
\setlength{\headheight}{0.5pt} % Adjust the head height
\renewcommand{\headrulewidth}{0.4pt} % Line under the header
\renewcommand{\footrulewidth}{0.4pt} % Line above the footer
% Header | Footer 

\newcommand{\germanqq}[1]{{\selectlanguage{german}\glqq#1\grqq\selectlanguage{english}}}

\DeclareMathOperator{\tg}{tg}
\newcommand{\tgx}{\tg x}

\DeclareMathOperator{\arctg}{arctg}
\newcommand{\arctgx}{\arctg x}

\makeatletter
\newcommand*{\rom}[1]{\expandafter\@slowromancap\romannumeral #1@}
\makeatother

\title{Savarankiškas darbas nr. 2}
\author{Vilius Paliokas}
\date{2023/10/17}

\setlist{after=\vspace{\baselineskip}}

% Title spacing
\usepackage{titlesec}
\titlespacing*{\subsection}{0pt}{\baselineskip}{0.5\baselineskip}
% ------------------------ 

\begin{document}
\thispagestyle{fancy}

\subsection*{1 variantas}

\begin{enumerate}
      \item Kokia yra pagrindinė aritmetinės progresijos savybė, kuri skiria ją
            nuo įprastos skaičių sekos? Užrašykite dvi skaičių sekas: vieną
            įprastą ir kitą aritmetinę progresiją. Pagrįskite kodėl pirmoji
            seka nėra
            aritmetinė progresija, o antroji yra.

      \item Parašykite sekos ($a_{n}$) pirmuosius penkis
            narius, kai jos \textit{n}-ojo nario formulė yra:

            $a_{n}=\frac{2^{n} + 1}{10 - n^2}$;

      \item Parašykite sekos ($b_{n}$) pirmuosius penkis
            narius, kai jos seka užrašyta rekurentiškai:

            $b_1 = -1, b_{2} = 1, b_{n+2}=3b_{n+1}-2b_{n}$;

      \item Parašykite aritmetinės progresijos ($c_{n}$) pirmuosius penkis
            narius, kai pirmi jos nariai yra: $10$, $13$.

      \item Raskite aritmetinės progressijos ($g_{n}$) \textit{n}-tojo nario
            formulę, antrąjį narį ir
            skirtumą, 17 pirmųjų narių sumą, kai:
            $g_{11} = -1$, $g_{17}=2$;
      \item $ 3x -4 $, $x + 2$, $2$ skaičių seka yra aritmetinės progresijos
            pirmieji trys nariai. Užrašykite šios aritmetinės progresijos
            \textit{n}-tojo
            nario formulę ir apskaičiuokite $S_{10}$.

      \item \textit{(papildomas)} Sekos \textit{n}-tojo nario formulė yra $h_{n} = 3n-4$. 
      \begin{enumerate}[label= (\alph*)]
            \item Įrodykite, kad ši seka yra aritmetinė progresija.
            \item Apskaičiuokite šios progresijos pirmųjų dviejų šimtų narių sumą.
      \end{enumerate} 
\end{enumerate}

\begin{small}
      \begin{enumerate*}[label={(\arabic*)}]
            \item \textbf{Visur užrašykite atsakymus} ($Ats\ldots$);
            \item Jokio sukčiavimo. Negalima naudotis užrašais, vadovėliais,
            elektroniniais prietaisais;
            \item Jokio kalbėjimo;
            \item Rašyti aiškiai, nedviprasmiškai;
            \item Galima naudotis tik savo skaičiuotuvu ir formulių lapu;
      \end{enumerate*}
\end{small}

\vspace{10 em}

\subsection*{1 variantas}

\begin{enumerate}
      \item Kokia yra pagrindinė aritmetinės progresijos savybė, kuri skiria ją
            nuo įprastos skaičių sekos? Užrašykite dvi skaičių sekas: vieną
            įprastą ir kitą aritmetinę progresiją. Pagrįskite kodėl pirmoji
            seka nėra
            aritmetinė progresija, o antroji yra.

      \item Parašykite sekos ($a_{n}$) pirmuosius penkis
            narius, kai jos \textit{n}-ojo nario formulė yra:

            $a_{n}=\frac{2^{n} + 1}{10 - n^2}$;

      \item Parašykite sekos ($b_{n}$) pirmuosius penkis
            narius, kai jos seka užrašyta rekurentiškai:

            $b_1 = -1, b_{2} = 1, b_{n+2}=3b_{n+1}-2b_{n}$;

      \item Parašykite aritmetinės progresijos ($c_{n}$) pirmuosius penkis
            narius, kai pirmi jos nariai yra: $10$, $13$.

      \item Raskite aritmetinės progressijos ($g_{n}$) \textit{n}-tojo nario
            formulę, antrąjį narį ir
            skirtumą, 17 pirmųjų narių sumą, kai:
            $g_{11} = -1$, $g_{17}=2$;
      \item $ 3x -4 $, $x + 2$, $2$ skaičių seka yra aritmetinės progresijos
            pirmieji trys nariai. Užrašykite šios aritmetinės progresijos
            \textit{n}-tojo
            nario formulę ir apskaičiuokite $S_{10}$.

      \item \textit{(papildomas)} Sekos \textit{n}-tojo nario formulė yra $h_{n} = 3n-4$. 
      \begin{enumerate}[label= (\alph*)]
            \item Įrodykite, kad ši seka yra aritmetinė progresija.
            \item Apskaičiuokite šios progresijos pirmųjų dviejų šimtų narių sumą.
      \end{enumerate} 
\end{enumerate}

\begin{small}
      \begin{enumerate*}[label={(\arabic*)}]
            \item \textbf{Visur užrašykite atsakymus} ($Ats\ldots$);
            \item Jokio sukčiavimo. Negalima naudotis užrašais, vadovėliais,
            elektroniniais prietaisais;
            \item Jokio kalbėjimo;
            \item Rašyti aiškiai, nedviprasmiškai;
            \item Galima naudotis tik savo skaičiuotuvu ir formulių lapu;
      \end{enumerate*}
\end{small}

\end{document}