\documentclass[a4paper]{article}
\usepackage[top=1cm, bottom=1cm, left=1cm, right=1cm]{geometry}

\usepackage{parskip} % Package to tweak paragraph skipping
\usepackage{tikz} % Package for drawing
\usepackage{tkz-euclide}
\usetikzlibrary{fit,positioning}
\usetikzlibrary{arrows.meta}
\usetikzlibrary{patterns,patterns.meta}
\usepackage[inline]{enumitem}
\usepackage{amsmath,amssymb}
\usepackage{tasks}
\usepackage{amsmath}
\usepackage{hyperref}
\usepackage[main=lithuanian, german, shorthands=off]{babel}
\usepackage{tgpagella}
\usepackage[L7x,T1]{fontenc}
\usepackage[utf8]{inputenc}
\usepackage{enumitem}
\usepackage{booktabs} % For better looking tables
\usepackage{venndiagram}
\usepackage{subfig}
\usepackage{multirow}
\newcommand{\germanqq}[1]{{\selectlanguage{german}\glqq#1\grqq\selectlanguage{english}}}

\DeclareMathOperator{\tg}{tg}
\newcommand{\tgx}{\tg x}

\DeclareMathOperator{\arctg}{arctg}
\newcommand{\arctgx}{\arctg x}

\title{Kontrolinis darbas nr. 2}
\author{Vilius Paliokas}
\date{2023/10/17}

\setlist{after=\vspace{\baselineskip}}

\begin{document}
\thispagestyle{empty}
\subsection*{1 variantas}

\begin{enumerate}
      \item Apskaičiuokite:

            \begin{tasks}[item-format={\normalfont}, after-item-skip=4mm](3)
                  \task $\arcsin{\frac{1}{2}} + \arcsin{\frac{\sqrt{2}}{2}} $;
                  \task $\arccos{-\frac{\sqrt{3}}{2}} + 2\arcsin{1} $;
                  \task $\cos({2\arctg{\sqrt{3}} +
                              \arctg{\frac{3}{3\sqrt{3}}}})  $;

            \end{tasks}

      \item Kuriame ketvirtje yra posūkio kampas $\alpha$, jeigu:
            \begin{tasks}[item-format={\normalfont}, after-item-skip=4mm](2)
                  \task $\sin \alpha = -0.8$, o $\cos \alpha > 0$;
                  \task $\tg \alpha = \frac{2}{3}$, o $\sin \alpha < 0$;
            \end{tasks}
      \item Apskaičiuokite $\sin \alpha$, $\cos \alpha$, $\tg \alpha$, kai
            stačiakampėje koordinačių plokštumoje pasukus spindulį $OX$ kampu
            $\alpha$ spindulio taškas $A(1; 0)$ perėjo į tašką $A_{1}$, kurio
            koordinatės
            yra:
            \begin{tasks}[item-format={\normalfont}, after-item-skip=4mm](4)
                  \task $(0; 1)$;
                  \task $(1; 0)$;
                  \task $(\frac{1}{2}; -\frac{\sqrt{3}}{2})$;
                  \task $(\frac{\sqrt{2}}{2}; \frac{\sqrt{2}}{2})$;
            \end{tasks}

      \item Supaprastinkite reiškinį, kad posūkio kampas būtų \textbf{nuo
                  $\boldsymbol{0^\circ}$
                  iki $\boldsymbol{90^\circ}$} ir tada apskaičiuokite jo
            reikšmę
            (\textbf{būtina parodyti veiksmų seką}):
            \begin{tasks}[item-format={\normalfont},
                        after-item-skip=4mm](3)
                  \task $\tg (-780^\circ)$;
                  \task $\cos 660^\circ$;
                  \task $\sin 315^\circ$;
            \end{tasks}

      \item Supaprastinkite reiškinį, kad posūkio kampas būtų nuo
            \textbf{$\boldsymbol{-90^\circ}$
                  iki  $\boldsymbol{0^\circ}$} ir tada apskaičiuokite jo
            reikšmę
            (\textbf{būtina parodyti veiksmų seką}):
            \begin{tasks}[item-format={\normalfont},
                        after-item-skip=4mm](3)
                  \task $\tg 330^\circ$;
                  \task $\cos (-780^\circ)$;
                  \task $\sin 660^\circ$;
            \end{tasks}

      \item Kuriame koordinačių plokštumos OXY ketvirtyje ar ašyje yra
            posukūkio kampas $\alpha$, jei:

            \begin{tasks}[item-format={\normalfont},
                        after-item-skip=4mm](3)
                  \task $\alpha = 116^\circ$
                  \task $\alpha = -1560^\circ$
                  \task $\alpha = 1845^\circ$
            \end{tasks}

\end{enumerate}

\begin{table}[!htpb]
      \centering
      \begin{tabular}{|lllllllllllllll|}
            \hline
            \multicolumn{15}{|l|}{Užduočių vertė}

            \\ \hline
            \multicolumn{3}{|l|}{1}

             & \multicolumn{2}{l|}{2}

             & \multicolumn{4}{l|}{3}

             &
            \multicolumn{3}{l|}{4}

             & \multicolumn{3}{l|}{5}

            \\
            \hline
            \multicolumn{1}{|l|}{\begin{tabular}[c]{@{}l@{}}a\\ 1\end{tabular}}
             &
            \multicolumn{1}{l|}{\begin{tabular}[c]{@{}l@{}}b\\ 1\end{tabular}}
             &
            \multicolumn{1}{l|}{\begin{tabular}[c]{@{}l@{}}c\\ 2\end{tabular}}
             &
            \multicolumn{1}{l|}{\begin{tabular}[c]{@{}l@{}}a\\ 2\end{tabular}}
             &
            \multicolumn{1}{l|}{\begin{tabular}[c]{@{}l@{}}b\\ 2\end{tabular}}
             &
            \multicolumn{1}{l|}{\begin{tabular}[c]{@{}l@{}}a\\ 3\end{tabular}}
             &
            \multicolumn{1}{l|}{\begin{tabular}[c]{@{}l@{}}b\\ 4\end{tabular}}
             &
            \multicolumn{1}{l|}{\begin{tabular}[c]{@{}l@{}}c\\ 4\end{tabular}}
             &
            \multicolumn{1}{l|}{\begin{tabular}[c]{@{}l@{}}d\\ 4\end{tabular}}
             &
            \multicolumn{1}{l|}{\begin{tabular}[c]{@{}l@{}}a\\ 1\end{tabular}}
             &
            \multicolumn{1}{l|}{\begin{tabular}[c]{@{}l@{}}b\\ 2\end{tabular}}
             &
            \multicolumn{1}{l|}{\begin{tabular}[c]{@{}l@{}}c\\ 2\end{tabular}}
             &
            \multicolumn{1}{l|}{\begin{tabular}[c]{@{}l@{}}a\\ 1\end{tabular}}
             &
            \multicolumn{1}{l|}{\begin{tabular}[c]{@{}l@{}}b\\ 1\end{tabular}}
             &
            \begin{tabular}[c]{@{}l@{}}c\\ 1\end{tabular}
            \\ \hline
      \end{tabular}
\end{table}

\begin{small}
      \begin{enumerate*}[label={(\arabic*)}]
            \item \textbf{Visur užrašykite atsakymus} ($Ats\ldots$);
            \item Jokio sukčiavimo. Negalima naudotis užrašais, vadovėliais,
            elektroniniais prietaisais;
            \item Jokio kalbėjimo;
            \item Rašyti aiškiai, nedviprasmiškai;
            \item Galima naudotis tik savo skaičiuotuvu ir formulių lapu;
      \end{enumerate*}
\end{small}

\subsection*{1 variantas}

\begin{enumerate}
      \item Apskaičiuokite:

            \begin{tasks}[item-format={\normalfont}, after-item-skip=4mm](3)
                  \task $\arcsin{\frac{1}{2}} + \arcsin{\frac{\sqrt{2}}{2}} $;
                  \task $\arccos{-\frac{\sqrt{3}}{2}} + 2\arcsin{1} $;
                  \task $\cos({2\arctg{\sqrt{3}} +
                              \arctg{\frac{3}{3\sqrt{3}}}})  $.

            \end{tasks}

      \item Kuriame ketvirtje yra posūkio kampas $\alpha$, jeigu:
            \begin{tasks}[item-format={\normalfont}, after-item-skip=4mm](2)
                  \task $\sin \alpha = -0.8$, o $\cos \alpha > 0$;
                  \task $\tg \alpha = \frac{2}{3}$, o $\sin \alpha < 0$.
            \end{tasks}
      \item Apskaičiuokite $\sin \alpha$, $\cos \alpha$, $\tg \alpha$, kai
            stačiakampėje koordinačių plokštumoje pasukus spindulį $OX$ kampu
            $\alpha$ spindulio taškas $A(1; 0)$ perėjo į tašką $A_{1}$, kurio
            koordinatės
            yra:
            \begin{tasks}[item-format={\normalfont}, after-item-skip=4mm](4)
                  \task $(0; 1)$;
                  \task $(-1; 0)$;
                  \task $(\frac{1}{2}; -\frac{\sqrt{3}}{2})$;
                  \task $(\frac{\sqrt{2}}{2}; \frac{\sqrt{2}}{2})$.
            \end{tasks}

      \item Supaprastinkite reiškinį, kad posūkio kampas būtų \textbf{nuo
                  $\boldsymbol{0^\circ}$
                  iki $\boldsymbol{90^\circ}$} ir tada apskaičiuokite jo
            reikšmę
            (\textbf{būtina parodyti veiksmų seką}):
            \begin{tasks}[item-format={\normalfont},
                        after-item-skip=4mm](3)
                  \task $\tg (-780^\circ)$;
                  \task $\cos 660^\circ$;
                  \task $\sin 315^\circ$.
            \end{tasks}

      \item Supaprastinkite reiškinį, kad posūkio kampas būtų nuo
            \textbf{$\boldsymbol{-90^\circ}$
                  iki  $\boldsymbol{0^\circ}$} ir tada apskaičiuokite jo
            reikšmę
            (\textbf{būtina parodyti veiksmų seką}):
            \begin{tasks}[item-format={\normalfont},
                        after-item-skip=4mm](3)
                  \task $\tg 330^\circ$;
                  \task $\cos (-780^\circ)$;
                  \task $\sin 660^\circ$.
            \end{tasks}

      \item Kuriame koordinačių plokštumos OXY ketvirtyje ar ašyje yra
            posukūkio kampas $\alpha$, jei:

            \begin{tasks}[item-format={\normalfont},
                        after-item-skip=4mm](3)
                  \task $\alpha = -180^\circ$;
                  \task $\alpha = -1560^\circ$;
                  \task $\alpha = 1845^\circ$.
            \end{tasks}

\end{enumerate}

\begin{table}[!htpb]
      \centering
      \begin{tabular}{|lllllllllllllll|}
            \hline
            \multicolumn{15}{|l|}{Užduočių vertė}

            \\ \hline
            \multicolumn{3}{|l|}{1}

             & \multicolumn{2}{l|}{2}

             & \multicolumn{4}{l|}{3}

             &
            \multicolumn{3}{l|}{4}

             & \multicolumn{3}{l|}{5}

            \\
            \hline
            \multicolumn{1}{|l|}{\begin{tabular}[c]{@{}l@{}}a\\ 1\end{tabular}}
             &
            \multicolumn{1}{l|}{\begin{tabular}[c]{@{}l@{}}b\\ 1\end{tabular}}
             &
            \multicolumn{1}{l|}{\begin{tabular}[c]{@{}l@{}}c\\ 2\end{tabular}}
             &
            \multicolumn{1}{l|}{\begin{tabular}[c]{@{}l@{}}a\\ 2\end{tabular}}
             &
            \multicolumn{1}{l|}{\begin{tabular}[c]{@{}l@{}}b\\ 2\end{tabular}}
             &
            \multicolumn{1}{l|}{\begin{tabular}[c]{@{}l@{}}a\\ 3\end{tabular}}
             &
            \multicolumn{1}{l|}{\begin{tabular}[c]{@{}l@{}}b\\ 4\end{tabular}}
             &
            \multicolumn{1}{l|}{\begin{tabular}[c]{@{}l@{}}c\\ 4\end{tabular}}
             &
            \multicolumn{1}{l|}{\begin{tabular}[c]{@{}l@{}}d\\ 4\end{tabular}}
             &
            \multicolumn{1}{l|}{\begin{tabular}[c]{@{}l@{}}a\\ 1\end{tabular}}
             &
            \multicolumn{1}{l|}{\begin{tabular}[c]{@{}l@{}}b\\ 2\end{tabular}}
             &
            \multicolumn{1}{l|}{\begin{tabular}[c]{@{}l@{}}c\\ 2\end{tabular}}
             &
            \multicolumn{1}{l|}{\begin{tabular}[c]{@{}l@{}}a\\ 1\end{tabular}}
             &
            \multicolumn{1}{l|}{\begin{tabular}[c]{@{}l@{}}b\\ 1\end{tabular}}
             &
            \begin{tabular}[c]{@{}l@{}}c\\ 1\end{tabular}
            \\ \hline
      \end{tabular}
\end{table}

\begin{small}
      \begin{enumerate*}[label={(\arabic*)}]
            \item \textbf{Visur užrašykite atsakymus} ($Ats\ldots$);
            \item Jokio sukčiavimo. Negalima naudotis užrašais, vadovėliais,
            elektroniniais prietaisais;
            \item Jokio kalbėjimo;
            \item Rašyti aiškiai, nedviprasmiškai;
            \item Galima naudotis tik savo skaičiuotuvu ir formulių lapu;
      \end{enumerate*}
\end{small}

\end{document}