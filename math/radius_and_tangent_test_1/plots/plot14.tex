\begin{tikzpicture}
    % Circle parameters
    \def\r{3} % radius of the circle
    \def\q{5.5} % distance from the center to the external point
    \def\angle{10} % Rotation angle for the setup

    % x and y coordinates for the tangent points, rotated
    \pgfmathsetmacro{\x}{\r^2/\q} % x coordinate of the point of tangency
    \pgfmathsetmacro{\y}{\r*sqrt(\q^2-\r^2)/\q} % y coordinate of the point of tangency

    % Define points
    \coordinate (O) at (0,0); % Center of the circle
    \coordinate (P) at ({\q*cos(\angle)},{\q*sin(\angle)}); % External point, rotated
    \coordinate (T1) at ({\x*cos(\angle) - \y*sin(\angle)}, {\x*sin(\angle) + \y*cos(\angle)}); % Tangent point 1, rotated
    \coordinate (T2) at ({\x*cos(\angle) + \y*sin(\angle)}, {\x*sin(\angle) - \y*cos(\angle)}); % Tangent point 2, rotated

    % Draw the circle with radius 3
    \draw[line width=.75pt] (0,0) circle (\r);

    % Draw radii
    \draw[line width=1.25pt] (O) -- (T1);
    \draw[line width=1.25pt] (O) -- (T2);

    % Extend the tangent lines for visual effect
    \draw[line width=1.25pt] (P) -- ($(P)!1.5!(T1)$); % Extended
    \draw[line width=1.25pt] (P) -- ($(P)!1.5!(T2)$); % Extended

    % Calculate angles for the arc
    \pgfmathsetmacro{\startAngle}{atan2(\y,\x) + \angle}
    \pgfmathsetmacro{\endAngle}{atan2(-\y,\x) + \angle}

    % Clip and fill the region outside the circle within the tangents
    \begin{scope}
        \clip (P) -- (T1) -- (T2) -- cycle;
        \fill[gray!50, opacity=0.5] (P) -- (T1) arc[start angle=\startAngle, end angle=\endAngle, radius=\r] -- (T2) -- cycle;    \end{scope}

    % Redraw the circle for clean edges
    \draw[line width=.75pt] (0,0) circle (\r);

    % Mark the center of the circle with a dot
    \fill (O) circle (2pt);
    \node at (O) [left] {$O$};

    % Label the external point and tangent points
    \node at (P) [right] {$B$};
    \node at (T1) [above] {$C$};
    \node at (T2) [below] {$A$};

    \pic [draw, "$120^\circ$", angle radius=0.35cm, angle eccentricity=2.25] {angle = T2--O--T1};


\end{tikzpicture}