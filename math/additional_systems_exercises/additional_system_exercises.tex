\mathchardef\period=\mathcode`.
\documentclass[a4paper]{article}
\usepackage[top=1.45cm, bottom=1cm, left=1cm, right=1cm]{geometry}

\usepackage{parskip} % Package to tweak paragraph skipping
\usepackage{tikz} % Package for drawing
\usepackage{tkz-euclide}
\usepackage{siunitx}
\usepackage{wrapfig}
\usepackage{graphicx}

\usetikzlibrary{fit,positioning}
\usetikzlibrary{arrows.meta}
\usetikzlibrary{patterns,patterns.meta}
\usepackage[inline]{enumitem}
\usepackage{amsmath,amssymb}
\usepackage{tasks}
\usepackage{amsmath}
\usepackage{hyperref}
\usepackage[main=lithuanian, german, shorthands=off]{babel}
\usepackage{tgpagella}
\usepackage[L7x,T1]{fontenc}
\usepackage[utf8]{inputenc}
\usepackage{enumitem}
\usepackage{booktabs} % For better looking tables
\usepackage{venndiagram}
\usepackage{subfig}
\usepackage{multirow}
\usepackage{tabularray}
\usepackage{lipsum}
\usepackage{fancyhdr}

\usepackage{blindtext}
\usepackage{adjustbox}
\AfterEndEnvironment{wrapfigure}{\setlength{\intextsep}{0mm}}
\usepackage{afterpage}

\usepackage{icomma}

% Header | Footer 
\fancyhf{} % clear all header and footer fields
\fancyhead[R]{+uždaviniai}
% L for Left, you can also use R for Right or C for Center

% L for Left, you can also use R for Right or C for Center
\setlength{\headheight}{0.5pt} % Adjust the head height
\renewcommand{\headrulewidth}{0.4pt} % Line under the header
\renewcommand{\footrulewidth}{0.4pt} % Line above the footer
% Header | Footer 

\newcommand{\germanqq}[1]{{\selectlanguage{german}\glqq#1\grqq\selectlanguage{english}}}

\DeclareMathOperator{\tg}{tg}
\newcommand{\tgx}{\tg x}

\DeclareMathOperator{\arctg}{arctg}
\newcommand{\arctgx}{\arctg x}

\makeatletter
\newcommand*{\rom}[1]{\expandafter\@slowromancap\romannumeral #1@}
\makeatother

\title{Tekstiniai egzaminų uždaviniai}
\author{Vilius Paliokas}
\date{2024/05/12}

\setlist{after=\vspace{\baselineskip}}

% Title spacing
\usepackage{titlesec}
\titlespacing*{\subsection}{0pt}{\baselineskip}{0.5\baselineskip}
% ------------------------ 

\newcommand\blankpage{%      \null
      \thispagestyle{empty}%
      \addtocounter{page}{-1}%
      \newpage}

\begin{document}
\thispagestyle{fancy}

\titlespacing*{\subsection}{0pt}{.75ex}{0.75ex}

\begin{enumerate}
      \item \textit{(ROL20191112)} Raskite visus lygčių sistemos
            realiuosius sprendinius $(x,y)$.

            $\left\{\begin{aligned}
                        2x+3\{y\} & = -4,5 \\
                        6\{x\}+7y & = 8,9
                  \end{aligned}\right.$

            Pastaba. Kiekvienam realiajam skaičiui $a$ didžiausias sveikasis skaičius,
            neviršijantis $a$, vadinamas skaičiaus $a$ sveikąja dalimi ir žymimas $[a]$, o
            skaičius a - $[a]$ vadinamas skaičiaus $a$ trupmenine dalimi ir žymimas
            $\{a\}$.

      \item \textit{(ROL20220910)} Raskite lygčių sistemos

            $\left\{\begin{aligned}
                        ab+c+d & = 0 \\
                        bd+d+a & = 8 \\
                        cd+a+b & =1  \\
                        da+b+c & =7
                  \end{aligned}\right.$

            visus realiuosius sprendinius $(a,b,c,d)$.

      \item \textit{(ROL20200910)} Raskite lygčių sistemos

            $\left\{\begin{aligned}
                         & x^2+3xy-3x = -27 \\
                         & y^2-xy-3y = 25
                  \end{aligned}\right.$

            visus realiuosius sprendinius $(x, y)$.

      \item \textit{(ROL20211112)} Raskite lygčių sistemos

            $\left\{\begin{aligned}
                         & x^2+y^2 = x+3y                 \\
                         & x^4 + y^4 = \frac{(x+3y)^2}{2}
                  \end{aligned}\right.$

            visus realiuosius sprendinius $(x, y)$.

      \item \textit{(UOL200811)} Metai, įėję į istoriją kaip spaudos išradimo metai, užrašomi keturiais
            skaitmenimis. Šimtų skaitmuo yra lygus dešimčių ir tūkstančių skaitmenų sumai;
            vienetų skaitmuo yra vienetu mažesnis už dešimčių ir šimtų skaitmenų sumą.
            Jeigu ieškomąjį keturženklį skaičių padalintume iš dvigubos jo skaitmenų sumos,
            tai gautume dalmenį 51 ir liekaną 8, o jeigu prie ieškomojo skaičiaus pridėtume
            4905, tai gautoji suma būtų skaičius, kuriame ieškomojo skaičiaus skaitmenys
            surašyti atvirkštine tvarka. Kokiais metais išrasta spauda?

      \item \textit{(G1)} Išspręskite tiesinių lygčių sistema naudojantis elementariaisiais pertvarkiais:

            $\left\{\begin{aligned}
                        2x_1 & -x_2 & +x_3  & = 2 \\
                        x_1  & -x_2 & +2x_3 & = 1 \\
                        3x_1 &      & -4x_3 & =2  \\
                        2x_1 & +x_2 & -3x_3 & =4
                  \end{aligned}\right.$

            Galimi tokie elementarūs petvarkymai:

            \begin{enumerate}
                  \item bet kurios eilutės elementus galima padauginti arba padalinti iš skaičiaus,
                        nelygaus nuliui;
                  \item bet kurią eilutę galima pakeisti, pridėjus prie jos kitą eilutę, padaugintą iš
                        skaičiaus, nelygaus nuliui;
                  \item nulinę eilutę, jei visi jos nariai lygūs nuliui ir laisvasis narys už brūkšnio
                        taip pat lygus nuliui, galime atmesti;
                  \item eilutes galima sukeisti vietomis.
            \end{enumerate}

      \item \textit{(G2)} Išspręskite tiesinių lygčių sistema naudojantis elementariaisiais pertvarkiais:

            $\left\{\begin{aligned}
                        2x_1 & +x_2-x_3 +4x_4    = 3 \\
                        x_1  & -x_2+2x_3 - 2x_4 = 1  \\
                        3x_1 & +x_3  +2x_4  =5
                  \end{aligned}\right.$

      \item \textit{(G3)} Išspręskite tiesinių lygčių sistema naudojantis elementariaisiais pertvarkiais:

            $\left\{\begin{aligned}
                        2x_1 & +x_2-x_3 +4x_4    = 5 \\
                        2x_2 & + x_3 + x_4 = 2       \\
                        x_1  & + x_2 +x_3  +x_4  =3  \\
                        3x_1 & -x_3  +4x_4  =6
                  \end{aligned}\right.$

            \pagebreak

      \item \textit{(IBLH)} Suraskite $a$ ir $b$ reikšmes, su kuriomis lygčių sistema

            $\left\{\begin{aligned}
                        x-3z         & = -2 \\
                        -3x + y + 6z & = 3  \\
                        2x-2y+(a-4)z & =b-3
                  \end{aligned}\right.$

            \begin{tasks}[item-format={\normalfont}, after-item-skip=4mm](3)
                  \task neturi sprendinių;
                  \task turi vieną sprendinį;
                  \task turi begalę sprendinių;
            \end{tasks}

      \item \textit{(GPTO1)} A farmer has 100 acres of land on which to plant two crops: wheat and corn.
            Each acre of wheat requires 3 hours of labor and 2 units of fertilizer, while
            each acre of corn requires 2 hours of labor and 4 units of fertilizer. The
            farmer has a total of 240 hours of labor and 320 units of fertilizer available.
            If the profit per acre of wheat is \$40 and the profit per acre of corn is \$30,
            how should the farmer allocate the land to maximize profit?

            Solve the problem analytically and present the solution graphically. Provide a
            detailed argument for the best solution.

      \item \textit{(O2)} The M \& M book publishers can produce at most 2000 books in one day. The company produces
            Mathematics and Marketing texts. Production costs are \$10 for a mathematics text and \$30 for a
            marketing text. The daily operating budget is \$30,000. How many of each text should be produced if
                  the profit is \$1.50 for each mathematics text and \$1.60 for each marketing text?
            Does the policy change if they must produce at least 600 mathematics books and 200 marketing
            books in one day? Does the policy change (from the original set-up) if they must produce at least 600
            marketing books in one day?

            Solve the problem analytically and present the solution graphically. Provide a
            detailed argument for the best solution.
\end{enumerate}

\pagenumbering{gobble}
\end{document}