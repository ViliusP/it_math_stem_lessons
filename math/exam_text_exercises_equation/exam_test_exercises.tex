\mathchardef\period=\mathcode`.
\documentclass[a4paper]{article}
\usepackage[top=1.45cm, bottom=1cm, left=1cm, right=1cm]{geometry}

\usepackage{parskip} % Package to tweak paragraph skipping
\usepackage{tikz} % Package for drawing
\usepackage{tkz-euclide}
\usepackage{siunitx}
\usepackage{wrapfig}
\usepackage{graphicx}

\usetikzlibrary{fit,positioning}
\usetikzlibrary{arrows.meta}
\usetikzlibrary{patterns,patterns.meta}
\usepackage[inline]{enumitem}
\usepackage{amsmath,amssymb}
\usepackage{tasks}
\usepackage{amsmath}
\usepackage{hyperref}
\usepackage[main=lithuanian, german, shorthands=off]{babel}
\usepackage{tgpagella}
\usepackage[L7x,T1]{fontenc}
\usepackage[utf8]{inputenc}
\usepackage{enumitem}
\usepackage{booktabs} % For better looking tables
\usepackage{venndiagram}
\usepackage{subfig}
\usepackage{multirow}
\usepackage{tabularray}
\usepackage{lipsum}
\usepackage{fancyhdr}

\usepackage{blindtext}
\usepackage{adjustbox}
\AfterEndEnvironment{wrapfigure}{\setlength{\intextsep}{0mm}}
\usepackage{afterpage}

\usepackage{icomma}

% Header | Footer 
\fancyhf{} % clear all header and footer fields
\fancyhead[R]{Tekstiniai uždaviniai iš VBE}
% L for Left, you can also use R for Right or C for Center

% L for Left, you can also use R for Right or C for Center
\setlength{\headheight}{0.5pt} % Adjust the head height
\renewcommand{\headrulewidth}{0.4pt} % Line under the header
\renewcommand{\footrulewidth}{0.4pt} % Line above the footer
% Header | Footer 

\newcommand{\germanqq}[1]{{\selectlanguage{german}\glqq#1\grqq\selectlanguage{english}}}

\DeclareMathOperator{\tg}{tg}
\newcommand{\tgx}{\tg x}

\DeclareMathOperator{\arctg}{arctg}
\newcommand{\arctgx}{\arctg x}

\makeatletter
\newcommand*{\rom}[1]{\expandafter\@slowromancap\romannumeral #1@}
\makeatother

\title{Tekstiniai egzaminų uždaviniai}
\author{Vilius Paliokas}
\date{2024/05/12}

\setlist{after=\vspace{\baselineskip}}

% Title spacing
\usepackage{titlesec}
\titlespacing*{\subsection}{0pt}{\baselineskip}{0.5\baselineskip}
% ------------------------ 

\newcommand\blankpage{%      \null
      \thispagestyle{empty}%
      \addtocounter{page}{-1}%
      \newpage}

\begin{document}
\thispagestyle{fancy}

\titlespacing*{\subsection}{0pt}{.75ex}{0.75ex}

\begin{enumerate}
      \item \textit{(2022 m. 27 užd.)} Įmonė turi naujos ir senos kartos lazerių. Naujos kartos lazerio per vieną valandą išpjautų detalių
            skaičius yra sveikasis ir didesnis už $8$. Senos kartos lazeris per vieną valandą išpjauna keturiomis
            detalėmis mažiau negu naujos kartos lazeris. Tam tikrą gautą užsakymą įmonė vienu naujos kartos
            lazeriu gali atlikti per sveikąjį valandų skaičių, o trimis senos kartos lazeriais – dviem valandomis
            greičiau. Kiek daugiausia detalių gali sudaryti šį užsakymą.

      \item \textit{(2022 m. pakartotinė sesija, 28 užd.)} Iš vietovių A ir B tuo pačiu metu vienas priešais kitą išvažiavo du dviratininkai ir susitiko po
            t minučių. Laikykime, kad visą kelią dviratininkai važiavo pastoviais greičiais (km/min). Jeigu
            pirmojo dviratininko greitis būtų buvęs dvigubai didesnis, o antrojo liktų toks pat, tai dviratininkai
            susitiktų $5$ minutėmis anksčiau. Jeigu antrojo dviratininko greitis būtų buvęs dvigubai didesnis, o
            pirmojo liktų toks pat, tai dviratininkai susitiktų $8$ minutėmis anksčiau. Apskaičiuokite t skaitinę
            reikšmę.

      \item \textit{(2021 m. 26 užd.)} Skaičiai $a$, $b$ ir $c$ yra trys iš eilės einantys aritmetinės progresijos nariai (čia $a \neq b$), o
            skaičiai $b$, $c$ ir $a$ yra trys iš eilės einantys geometrinės progresijos nariai. Apskaičiuokite
            geometrinės progresijos vardiklį.

      \item \textit{(2019 m. 25 užd.)} Skaičiai $a$, $b$, $10-a$ yra trys iš eilės einantys \textbf{didėjančiosios} aritmetinės progresijos nariai.
            Skaičiai $a+1, b+4, 29-a$ yra trys iš eilės einantys geometrinės progresijos nariai. Raskite
            skaičius $a$ ir $b$

      \item \textit{(2018 m. 18 užd.)} Dvi sesutės – Irutė ir Birutė – kurį laiką gaudė pokemonus. Irutė kasdien sugaudavo po
            $x$ pokemonų, o Birutė – trimis pokemonais daugiau. Irutė pokemonus gaudė viena diena ilgiau
            negu Birutė. Birutė iš viso sugavo $484$ pokemonus, o Irutė iš viso sugavo $437$ pokemonus.
            Apskaičiuokite $x$ reikšmę.

      \item \textit{(2017 m. 25 užd.)} Per sausio ir kovo mėnesius kartu paėmus buvo pagaminta dvigubai daugiau produkcijos negu
            per vasario mėnesį. Per vasario ir kovo mėnesius kartu paėmus buvo pagaminta trigubai daugiau
            produkcijos negu per sausio mėnesį. Kurį iš šių mėnesių buvo pagaminta daugiausia produkcijos,
            o kurį – mažiausia? Atsakymą argumentuokite.

      \item \textit{(2016 m. 23 užd.)} 100 metrų plaukimo varžybose dalyvavo Rūta, Julija ir Džesika. \textbf{Rūta savo finišo momentu}
            lenkė Juliją 2 metrais, o Julija \textbf{savo finišo momentu} lenkė Džesiką metru. Tarkime, kad jos
            distanciją plaukė pastoviais greičiais. Keliais metrais Rūta savo finišo momentu lenkė Džesiką?
            Skaičiuodami laikykite, kad plaukikės yra materialūs taškai, t. y. plaukikių matmenų
            nepaisykite.

      \item \textit{(2016 m. pakartotinė sesija, 24 užd.)} Du vandens siurbliai pripildo baseiną per $5$ valandas. Siurbliai, dirbdami po vieną, pripildo
            baseiną per skirtingą valandų skaičių. Laikydami, kad šie skaičiai yra sveikieji1, raskite, per kiek
            valandų baseiną pripildo kiekvienas siurblys, dirbdamas atskirai.

      \item \textit{(2015 m. 25 užd.)} Tuo pačiu metu iš miestelių $A$ ir $B$ pastoviais greičiais vienas priešais kitą išvažiavo du
            dviratininkai. Pirmasis važiavo iš miestelio $A$ į miestelį, $B$ o antrasis – iš miestelio $B$ į
            miestelį $A$. Pakeliui jie susitiko. Po susitikimo pirmasis dviratininkas į miestelįB atvyko po $36$
            minučių, o antrasis į miestelį $A$ atvyko po $25$ minučių. Kiek minučių pirmasis dviratininkas
            važiavo iš miestelio $A$ iki susitikimo su antruoju dviratininku?

      \item \textit{(2015 m. pakartotinė sesija 25 užd.)} Lentoje buvo užrašyti skirtingi natūralieji skaičiai ir apskaičiuotas jų sumos ir sandaugos
            santykis. Nutrynus mažiausią lentoje užrašytą skaičių, vėl apskaičiuotas likusių skaičių sumos ir
            sandaugos santykis. Jis buvo tris kartus didesnis už pirmąjį santykį. Raskite skaičių, kuris buvo
            nutrintas.

            Patarimas: spręsdami pažymėkite likusių skaičių sumą $S$, likusių skaičių
            sandaugą $P$ ir nutrintąjį skaičių $x$.

      \item \textit{(2014 m. 27 užd.)} Duoti keturi teigiami skaičiai. Pirmas, antras ir trečias skaičiai sudaro aritmetinę progresiją, o
            šių skaičių suma lygi $12$. Antras, trečias ir ketvirtas skaičiai sudaro geometrinę progresiją, jų suma lygi $19$. Raskite šiuos keturis skaičius.

      \item \textit{(2013 m. 31 užd.)} Trys dviratininkai kas valandą išvažiuoja iš tos pačios vietos ir važiuoja viena kryptimi.
      Pirmojo dviratininko greitis $12\, km/h$, antrojo – $10\, km/h$. Trečiasis dviratininkas, važiuodamas greičiau nei pirmasis, pirmiausia pavijo antrąjį, o praėjus dar $2$ valandoms – pirmąjį
      dviratininką. Koks trečiojo dviratininko greitis?

      \item \textit{(2010 m. 22 užd.)} Trys plaukikai turi nuplaukti m ilgio baseino takeliu iki galo, iškart apsisukti ir grįžti atgal į starto vietą.
      pradžių startuoja pirmasis plaukikas, po sekundžių – antrasis, dar po $5\,sek.$ – trečiasis. Vienu momentu, dar nepasiekę takelio galo, visi plaukikai buvo nuplaukę vienodą atstumą. Trečiasis plaukikas, nuplaukęs iki takelio
      galo ir apsisukęs, sutiko antrąjį plaukiką iki takelio galo buvo likę plaukti $4\,m$, po to sutiko pirmąjį plaukiką, kuriam iki takelio galo buvo likę
      plaukti $7m$. Raskite trečiojo plaukiko greitį.

      \item \textit{(2009 m. 18 užd.)} Dviejų irkluotojų greičiai stovinčiame vandenyje yra lygūs. Jie treniruojasi taip: Jonas iš bazės nuplaukia $5\,km$ upe prieš srovę ir grįžta
      atgal į ją, o Domas iš kitos bazės nuplaukia $5\,km$ ežeru (stovinčiame vandenyje) ir grįžta atgal į ją.

      Kuris irkluotojas sugaišta mažiau laiko treniruotėje? (Nekreipkite dėmesio į
      laiką sugaištą apsigręžiant.)




\end{enumerate}

\pagenumbering{gobble}
\end{document}

