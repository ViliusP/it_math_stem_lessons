% !TEX root = ./equation_test_2.tex
\documentclass[a4paper]{article}
\usepackage[top=1.45cm, bottom=1cm, left=1cm, right=1cm]{geometry}
\mathchardef\period=\mathcode`.

\usepackage{parskip} % Package to tweak paragraph skipping
\usepackage{siunitx}

\usepackage[inline]{enumitem}
\usepackage{amsmath,amssymb}
\usepackage{tasks}
\usepackage{amsmath}
\usepackage{hyperref}
\usepackage[main=lithuanian, german, shorthands=off]{babel}
\usepackage{tgpagella}
\usepackage[L7x,T1]{fontenc}
\usepackage[utf8]{inputenc}
\usepackage{enumitem}
\usepackage{lipsum}
\usepackage{fancyhdr}

\usepackage{blindtext}
\usepackage{adjustbox}
\AfterEndEnvironment{wrapfigure}{\setlength{\intextsep}{0mm}}

\usepackage{icomma}

% Header | Footer 
\fancyhf{} % clear all header and footer fields
\fancyhead[R]{ lygčių sistemos ir trigonometrinės lygtys | savarankiškas darbas }
% L for Left, you can also use R for Right or C for Center
\fancyfoot[R]{ lygčių sistemos ir trigonometrinės lygtys | savarankiškas darbas }

% L for Left, you can also use R for Right or C for Center
\setlength{\headheight}{0.5pt} % Adjust the head height
\renewcommand{\headrulewidth}{0.4pt} % Line under the header
\renewcommand{\footrulewidth}{0.4pt} % Line above the footer
% Header | Footer 

\newcommand{\germanqq}[1]{{\selectlanguage{german}\glqq#1\grqq\selectlanguage{english}}}

\DeclareMathOperator{\tg}{tg}
\newcommand{\tgx}{\tg x}

\DeclareMathOperator{\arctg}{arctg}
\newcommand{\arctgx}{\arctg x}

\makeatletter
\newcommand*{\rom}[1]{\expandafter\@slowromancap\romannumeral #1@}
\makeatother

\title{Kontrolinis darbas - lygtys}
\author{Vilius Paliokas}
\date{2024/05/010}

\setlist{after=\vspace{\baselineskip}}

% Title spacing
\usepackage{titlesec}
\titlespacing*{\subsection}{0pt}{\baselineskip}{0.5\baselineskip}
% ------------------------ 

\begin{document}
\thispagestyle{fancy}

\titlespacing*{\subsection}{0pt}{.75ex}{0.75ex}

\subsection*{2 variantas}

\begin{enumerate}
      \item Išspręskite trignometrinę lygtį:
            \begin{tasks}[item-format={\normalfont}, after-item-skip=4mm](2)
                  \task \textit{(1 taškas)} $-3\sin{x}=1,5$;
                  \task \textit{(1 taškas)} $-\sqrt{3}+\tg{2x}=0$;
            \end{tasks}

      \item Išspręskite lygčių sistemą \textit{(po 1 tašką)}:
            \begin{tasks}[item-format={\normalfont}, after-item-skip=4mm](3)
                  \task   \par\vspace{-1.3\baselineskip}% <--- this is special case due to use of braces
                  $\left\{\begin{aligned}
                              6x - 5y & = 1  \\
                              2x - 3y & = 33
                        \end{aligned}\right.$

                  \task   \par\vspace{-1.3\baselineskip}% <--- this is special case due to use of braces
                  $\left\{\begin{aligned}
                              x + y & = 8   \\
                              xy    & = -20
                        \end{aligned}\right.$
                  \task   \par\vspace{-1.3\baselineskip}% <--- this is special case due to use of braces
                  $\left\{\begin{aligned}
                              5(x + 2y) -3 & = x+2 \\
                              y+4(x-3y)    & = 50
                        \end{aligned}\right.$
            \end{tasks}
      \item \textit{(2 taškai)} Pirmame maiše miltų buvo 2 kartus daugiau negu antrame. Kai iš pirmo maišo nusėmė $30\; kg$ miltų, o į antrajį įpylė $5\; kg$, tai antrame maiše buvo 1,5 karto daugiau miltų pirmame. Kiek kilogramų miltų iš pradžių buvo abiejuose maišuose kartu?
      \item \textit{(2 taškai)} Turistas moterine valtimi nuplaukė $6 km$ prieš srove ir grįžo atgal. Kelionė truko $4,5$ valandas. Kokiu greičiu būtų plaukusi valtis stovinčiame vandenyje, jei upės tėkmės greitis lygus $1 \; km/h$.
      \item \textit{(2 taškai)} Stačiojo trikmapio plotas yra $216 cm^2$. Vienas trikampio statinis $6 cm$ trumpesnis už kitą. Raskite trikampio statinius.
\end{enumerate}

\begin{small}
      \begin{enumerate*}[label={(\arabic*)}]
            \item \textbf{Visur} \textbf{nurodykite atsakymus} ($Ats\ldots$);
            \item Jokio sukčiavimo. Negalima naudotis užrašais, vadovėliais, elektroniniais
                  prietaisais;
            \item Jokio kalbėjimo;
            \item Rašyti aiškiai, nedviprasmiškai;
            \item Galima naudotis tik savo skaičiuotuvu ir formulių lapu;
      \end{enumerate*}
\end{small}


\subsection*{2 variantas}

\begin{enumerate}
      \item Išspręskite trignometrinę lygtį:
            \begin{tasks}[item-format={\normalfont}, after-item-skip=4mm](2)
                  \task \textit{(1 taškas)} $-3\sin{x}=1,5$;
                  \task \textit{(1 taškas)} $-\sqrt{3}+\tg{2x}=0$;
            \end{tasks}

      \item Išspręskite lygčių sistemą \textit{(po 1 tašką)}:
            \begin{tasks}[item-format={\normalfont}, after-item-skip=4mm](3)
                  \task   \par\vspace{-1.3\baselineskip}% <--- this is special case due to use of braces
                  $\left\{\begin{aligned}
                              6x - 5y & = 1  \\
                              2x - 3y & = 33
                        \end{aligned}\right.$

                  \task   \par\vspace{-1.3\baselineskip}% <--- this is special case due to use of braces
                  $\left\{\begin{aligned}
                              x + y & = 8   \\
                              xy    & = -20
                        \end{aligned}\right.$
                  \task   \par\vspace{-1.3\baselineskip}% <--- this is special case due to use of braces
                  $\left\{\begin{aligned}
                              5(x + 2y) -3 & = x+2 \\
                              y+4(x-3y)    & = 50
                        \end{aligned}\right.$
            \end{tasks}
      \item \textit{(2 taškai)} Pirmame maiše miltų buvo 2 kartus daugiau negu antrame. Kai iš pirmo maišo nusėmė $30\; kg$ miltų, o į antrajį įpylė $5\; kg$, tai antrame maiše buvo 1,5 karto daugiau miltų pirmame. Kiek kilogramų miltų iš pradžių buvo abiejuose maišuose kartu?
      \item \textit{(2 taškai)} Turistas moterine valtimi nuplaukė $6 km$ prieš srove ir grįžo atgal. Kelionė truko $4,5$ valandas. Kokiu greičiu būtų plaukusi valtis stovinčiame vandenyje, jei upės tėkmės greitis lygus $1 \; km/h$.
      \item \textit{(2 taškai)} Stačiojo trikmapio plotas yra $216 cm^2$. Vienas trikampio statinis $6 cm$ trumpesnis už kitą. Raskite trikampio statinius.
\end{enumerate}

\begin{small}
      \begin{enumerate*}[label={(\arabic*)}]
            \item \textbf{Visur} \textbf{nurodykite atsakymus} ($Ats\ldots$);
            \item Jokio sukčiavimo. Negalima naudotis užrašais, vadovėliais, elektroniniais
                  prietaisais;
            \item Jokio kalbėjimo;
            \item Rašyti aiškiai, nedviprasmiškai;
            \item Galima naudotis tik savo skaičiuotuvu ir formulių lapu;
      \end{enumerate*}
\end{small}


\subsection*{2 variantas}

\begin{enumerate}
      \item Išspręskite trignometrinę lygtį:
            \begin{tasks}[item-format={\normalfont}, after-item-skip=4mm](2)
                  \task \textit{(1 taškas)} $-3\sin{x}=1,5$;
                  \task \textit{(1 taškas)} $-\sqrt{3}+\tg{2x}=0$;
            \end{tasks}

      \item Išspręskite lygčių sistemą \textit{(po 1 tašką)}:
            \begin{tasks}[item-format={\normalfont}, after-item-skip=4mm](3)
                  \task   \par\vspace{-1.3\baselineskip}% <--- this is special case due to use of braces
                  $\left\{\begin{aligned}
                              6x - 5y & = 1  \\
                              2x - 3y & = 33
                        \end{aligned}\right.$

                  \task   \par\vspace{-1.3\baselineskip}% <--- this is special case due to use of braces
                  $\left\{\begin{aligned}
                              x + y & = 8   \\
                              xy    & = -20
                        \end{aligned}\right.$
                  \task   \par\vspace{-1.3\baselineskip}% <--- this is special case due to use of braces
                  $\left\{\begin{aligned}
                              5(x + 2y) -3 & = x+2 \\
                              y+4(x-3y)    & = 50
                        \end{aligned}\right.$
            \end{tasks}
      \item \textit{(2 taškai)} Pirmame maiše miltų buvo 2 kartus daugiau negu antrame. Kai iš pirmo maišo nusėmė $30\; kg$ miltų, o į antrajį įpylė $5\; kg$, tai antrame maiše buvo 1,5 karto daugiau miltų pirmame. Kiek kilogramų miltų iš pradžių buvo abiejuose maišuose kartu?
      \item \textit{(2 taškai)} Turistas moterine valtimi nuplaukė $6 km$ prieš srove ir grįžo atgal. Kelionė truko $4,5$ valandas. Kokiu greičiu būtų plaukusi valtis stovinčiame vandenyje, jei upės tėkmės greitis lygus $1 \; km/h$.
      \item \textit{(2 taškai)} Stačiojo trikmapio plotas yra $216 cm^2$. Vienas trikampio statinis $6 cm$ trumpesnis už kitą. Raskite trikampio statinius.
\end{enumerate}

\begin{small}
      \begin{enumerate*}[label={(\arabic*)}]
            \item \textbf{Visur} \textbf{nurodykite atsakymus} ($Ats\ldots$);
            \item Jokio sukčiavimo. Negalima naudotis užrašais, vadovėliais, elektroniniais
                  prietaisais;
            \item Jokio kalbėjimo;
            \item Rašyti aiškiai, nedviprasmiškai;
            \item Galima naudotis tik savo skaičiuotuvu ir formulių lapu;
      \end{enumerate*}
\end{small}



\end{document}