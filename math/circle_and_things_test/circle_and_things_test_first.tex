\mathchardef\period=\mathcode`.
\documentclass[a4paper]{article}
\usepackage[top=1.45cm, bottom=1.45cm, left=1cm, right=1cm]{geometry}

\usepackage{parskip} % Package to tweak paragraph skipping
\usepackage{tikz} % Package for drawing
\usepackage{tkz-euclide}
\usepackage{siunitx}
\usepackage{wrapfig}
\usepackage{graphicx}
\usepackage{array}
\usepackage{changepage}

\usepackage{pgfplots}
\usetikzlibrary{fit,positioning}
\usetikzlibrary{arrows.meta}
\usetikzlibrary{patterns,patterns.meta}
\usetikzlibrary{calc,intersections}
\usetikzlibrary{angles, quotes}
\usetikzlibrary{shapes}
\usetikzlibrary{intersections,pgfplots.fillbetween}
\usepackage[inline]{enumitem}
\usepackage{amsmath,amssymb}
\usepackage{tasks}
\usepackage{amsmath}
\usepackage{hyperref}
\usepackage[main=lithuanian, german, shorthands=off]{babel}
\usepackage{tgpagella}
\usepackage[L7x,T1]{fontenc}
\usepackage[utf8]{inputenc}
\usepackage{enumitem}
\usepackage{booktabs} % For better looking tables
\usepackage{venndiagram}
\usepackage{subfig}
\usepackage{multirow}
\usepackage{tabularray}
\usepackage{lipsum}
\usepackage{fancyhdr}

\usepackage{blindtext}
\usepackage{adjustbox}
\AfterEndEnvironment{wrapfigure}{\setlength{\intextsep}{0mm}}

\usepackage{icomma}

% Header | Footer 
\fancyhf{} % clear all header and footer fields
\fancyhead[R]{skritulys | plotas | lanko ilgis
      | įbrėžtinis kampas | centrinis kampas | nuopjova | išpjova}
% L for Left, you can also use R for Right or C for Center
\fancyfoot[R]{skritulys | plotas | lanko ilgis
      | įbrėžtinis kampas | centrinis kampas | nuopjova | išpjova}
% L for Left, you can also use R for Right or C for Center
\setlength{\headheight}{0.5pt} % Adjust the head height
\renewcommand{\headrulewidth}{0.4pt} % Line under the header
\renewcommand{\footrulewidth}{0.4pt} % Line above the footer
% Header | Footer 

\newcommand{\germanqq}[1]{{\selectlanguage{german}\glqq#1\grqq\selectlanguage{english}}}

\DeclareMathOperator{\tg}{tg}
\newcommand{\tgx}{\tg x}

\DeclareMathOperator{\arctg}{arctg}
\newcommand{\arctgx}{\arctg x}

\makeatletter
\newcommand*{\rom}[1]{\expandafter\@slowromancap\romannumeral #1@}
\makeatother

\title{Kontrolinis darbas nr. 2}
\author{Vilius Paliokas}
\date{2023/10/17}

\setlist{after=\vspace{\baselineskip}}

% Title spacing
\usepackage{titlesec}
\titlespacing*{\subsection}{0pt}{\baselineskip}{0.5\baselineskip}
% ------------------------ 

\begin{document}
\thispagestyle{fancy}

\subsection*{1 variantas}

Užduočių sprendimo taisyklės:

\begin{enumerate}[label= (\alph*), after=\vspace{-\baselineskip}]

      \item Jeigu reikia, pirmiausia nubrėžiamas brėžinys ir suteikiamos
            raidės viršunėms arba kraštinėmis naudojantis.

      \item Užrašoma naudojama teorema, formulė ar taisyklė (jeigu
            naudojama teorema, turi būti parašyta: \germanqq{pagal
                  \ldots } arba \germanqq{kadangi \ldots } arba \germanqq{nes
                  \ldots }). Jeigu buvo
            nubrėžtas brėžinys, formulėje, teoremoje ar taisyklėje naudojamos
            brėžinio raides.
\end{enumerate}

\vspace{0.25cm}
\par\noindent\rule{\textwidth}{0.5pt}
\vspace{1mm}

\begin{minipage}{0.5\textwidth}
      \begin{enumerate}
            \item \textit{(2 taškai)} Apskaičiuokite kampus $BCA$ ir $BAC$, kai
                  kampo $ABC$ dydis yra $50^\circ$, BA - skersmuo.

                  \begin{tikzpicture}[scale=0.65]
                        % Draw the circle
                        \draw (0,0) circle (2cm);

                        % Define points A, B, and shifted C
                        \filldraw [black] (0,0) circle (2pt) node[anchor=north]
                              {O};
                        \filldraw [black] (2,0) circle (2pt) node[anchor=west]
                              {A};
                        \filldraw [black] (-2,0) circle (2pt) node[anchor=east]
                              {B};
                        \filldraw [black] (1,1.732) circle (2pt)
                        node[anchor=south
                                    west] {C};
                        % Shifted C to the right

                        % Draw arc for angle
                        \draw (2,0) arc (0:180:2cm);

                        % Draw the central angle AOB
                        \draw (0,0) -- (2,0);
                        \draw (0,0) -- (-2,0);

                        % Draw the inscribed angle ACB
                        \draw (1,1.732) -- (2,0);
                        \draw (1,1.732) -- (-2,0);

                        % Optional labels for the angles
                        % \draw (1,0.3) node {Central Angle};
                        % \draw (0.5,1) node {Inscribed Angle};
                  \end{tikzpicture}
      \end{enumerate}
\end{minipage}
\hfill % This command adds space between the minipages if needed
\begin{minipage}{0.5\textwidth}
      \begin{enumerate}
            \setcounter{enumi}{1} % This continues the numbering from the previous enumerate
            \item \textit{(1 taškas)} Apskaičiuokite kampą $BCA$, kai kampo
                  $BOA$ dydis yra $90^\circ$.

                  \begin{tikzpicture}[scale=0.65]
                        % Draw the circle
                        \draw (0,0) circle (2cm);

                        % Define points A, B, and C
                        \filldraw [black] (0,0) circle (2pt) node[anchor=north]
                              {O};
                        \filldraw [black] (2,0) circle (2pt) node[anchor=west]
                              {A};
                        \filldraw [black] (1,1.732) circle (2pt)
                        node[anchor=south]
                              {B};
                        \filldraw [black] (-1.732,1) circle (2pt)
                        node[anchor=east]
                              {C};

                        % Draw arc for angle
                        \draw (2,0) arc (0:120:2cm);

                        % Draw the central angle AOB
                        \draw (0,0) -- (2,0);
                        \draw (0,0) -- (1,1.732);

                        % Draw the inscribed angle ACB
                        \draw (-1.732,1) -- (2,0);
                        \draw (-1.732,1) -- (1,1.732);

                  \end{tikzpicture}
      \end{enumerate}
\end{minipage}

\begin{enumerate}
      \setcounter{enumi}{2} % This continues the numbering from the previous enumerate

      \item \textit{(3 taškai)} Centrinis kampas yra $28^\circ$ didesnis už
            įbrėžtinį kampą, besiremiantį į tą patį lanką. Apskaičiuokite
            abiejų kampų
            dydžius.
      \item \textit{(2 taškai)} Naudodamiesi brėžinio duomenis, nustatykite,
            kurie iš kampų, sudarytų iš pažymėtų atkarpų tarp $A$, $B$, $C$, $O$
            viršūnių, yra įbrėžtiniai, kurie centriniai. Jeigu tokių kampų
            nėra,
            parašykite \germanqq{nėra}. Taškas $O$ yra skritulio centras.

            \begin{table}[h]
                  \begin{adjustwidth}{-1cm}{-1cm}
                        % Adjust the values as needed
                        \centering

                        \begin{tabular}{m{3cm}|m{3cm}|m{3cm}|m{3cm}|m{3cm}|m{3cm}}
                              \hline
                              a)
                                                                          &
                              b)
                                                                          &
                              c)
                                                                          &
                              d)
                                                                          &
                              e)
                                                                          &
                              f)
                              \\
                              \hline
                              \centering
                              \begin{tikzpicture}[scale=0.55]
                                    \draw (0,0) circle (2cm);
                                    \draw (-2,0) -- (0,2) -- (2,0);
                                    % Correctly inscribed
                                    \filldraw [black] (0,0) circle (2pt)
                                    node[anchor=south] {O};
                                    % Center
                                    \filldraw [black] (-2,0) circle (2pt)
                                    node[anchor=east] {A};
                                    \filldraw [black] (0,2) circle (2pt)
                                    node[anchor=south] {B};
                                    \filldraw [black] (2,0) circle (2pt)
                                    node[anchor=west] {C};
                              \end{tikzpicture}       &
                              \centering
                              \begin{tikzpicture}[scale=0.55]
                                    \draw (0,0) circle (2cm);
                                    \draw (2,0) -- (0,-1) -- (-2,0);
                                    % Vertex inside the circle
                                    \filldraw [black] (0,0) circle (2pt)
                                    node[anchor=south] {O};
                                    % Center
                                    \filldraw [black] (-2,0) circle (2pt)
                                    node[anchor=east] {A};
                                    \filldraw [black] (0,-1) circle (2pt)
                                    node[anchor=north] {B};
                                    \filldraw [black] (2,0) circle (2pt)
                                    node[anchor=west] {C};
                              \end{tikzpicture}       &
                              \centering
                              \begin{tikzpicture}[scale=0.55]
                                    % Draw the circle
                                    \draw (0,0) circle (2cm);

                                    % Draw the radius and label it as "R"
                                    \draw (0,0) -- (2,0) node[midway,below] {};

                                    % Draw the tangent line and label it as "T"
                                    \draw (2,0) -- (0,-2) node[midway,right]{};

                                    % Draw the radius and label it as "R"
                                    \draw (-2,0) -- (0,0) node[midway,below]
                                    {};

                                    \draw (0,-2) -- (0,0) node[midway,below]
                                    {};

                                    % Vertex inside the circle
                                    \filldraw [black] (0,0) circle (2pt)
                                    node[anchor=north east] {O};

                                    % Center
                                    \filldraw [black] (-2,0) circle (2pt)
                                    node[anchor=east] {A};

                                    \filldraw [black] (0,-2) circle (2pt)
                                    node[anchor=south east] {B};

                                    \filldraw [black] (2,0) circle (2pt)
                                    node[anchor=west] {C};
                                    % Bigger Angle AOC (180 degrees)
                                    \draw (0.5,0) arc (0:180:0.5cm);
                                    \node at (0.2,1) {$180^\circ$};

                              \end{tikzpicture} &
                              \centering
                              \begin{tikzpicture}[scale=0.55]
                                    \draw (0,0) circle (2cm);
                                    \draw (1.41,1.41) -- (-1,-1) --
                                    (-1.41,1.41);
                                    % Diagonal vertex
                                    \filldraw [black] (0,0) circle (2pt)
                                    node[anchor=south] {O};
                                    % Center
                                    \filldraw [black] (-1.41,1.41) circle (2pt)
                                    node[anchor=south east]
                                          {A};
                                    \filldraw [black] (-1,-1) circle (2pt)
                                    node[anchor=west] {B};
                                    \filldraw [black] (1.41,1.41) circle (2pt)
                                    node[anchor=south west]
                                          {C};

                              \end{tikzpicture} &
                              \centering
                              \begin{tikzpicture}[scale=0.55]
                                    % Draw the circle
                                    \draw (0,0) circle (2cm);

                                    % Coordinates for the points where the right angle will touch the circle
                                    \coordinate (O) at (0,0);

                                    \coordinate (A) at (0:2cm);
                                    \coordinate (B) at ($(A)!2cm!90:(0,0)$);
                                    % 1cm away from A, at a 90 degree angle
                                    \coordinate (C) at ($(A)!2cm!-90:(0,0)$);
                                    % 1cm away from A, at a 0 degree angle

                                    % Vertex inside the circle
                                    \filldraw [black] (O) circle (2pt)
                                    node[anchor=north] {O};

                                    \filldraw [black] (A) circle (2pt)
                                    node[anchor=west] {A};

                                    \filldraw [black] (B) circle (2pt)
                                    node[anchor=west] {B};

                                    \filldraw [black] (C) circle (2pt)
                                    node[anchor=west] {C};

                                    % Draw the right angle
                                    \draw (B) -- (A) -- (O);
                                    \draw (A) -- (C);

                                    % Right angle symbols
                                    \pic [draw, angle radius=2.15mm] {right
                                          angle=B--A--O};
                                    \pic [draw, angle radius=1.9mm] {right
                                          angle=O--A--C};
                              \end{tikzpicture}
                                                                          &
                              \centering
                              \begin{tikzpicture}[scale=0.55]
                                    \draw (0,0) circle (2cm);
                                    \draw (0,0) -- (2,0) ;
                                    % Central angle
                                    \draw (0,0) -- (1.41,1.41) ;

                                    \draw (0,0) -- (-1.37,1.457) ;
                                    % Central angle
                                    \filldraw [black] (0,0) circle (2pt)
                                    node[anchor=north east] {O};
                                    % Center
                                    \filldraw [black] (2,0) circle (2pt)
                                    node[anchor=west] {A};
                                    \filldraw [black] (1.41,1.41) circle (2pt)
                                    node[anchor=south west]
                                          {B};
                                    \filldraw [black] (-1.37,1.457) circle
                                    (2pt)
                                    node[anchor=south east]
                                          {C};
                              \end{tikzpicture}
                        \end{tabular}
                  \end{adjustwidth}
            \end{table}
      \item \textit{(2 taškai)} Skritulys, kurio spindulys $14\;cm$,
            nubraižytas centrinis kampas lygus $200^\circ$. Apskaičiuokite šios
            išpjovos:

            \begin{tasks}[item-format={\normalfont}, after-item-skip=2mm](2)
                  \task lanko ilgį;
                  \task plotą;
            \end{tasks}

      \item \textit{(2 taškai)} Skritulio plotas $72\;cm^2$. Apskaičiuokite
            skritulį ribojančio apskritimo ilgį ($0,1\;cm$ tikslumu).

      \item \textit{(3 taškai)} Apskaičiuokite užbrūkšniuotos figūros plotą:

            \begin{tikzpicture}[scale=.75]
                  % Draw the circle
                  \draw (0,0) circle (2cm);

                  % Coordinates for the points where the right angle will touch the circle
                  \coordinate (O) at (0,0);
                  \coordinate (K) at (320:2cm);
                  \coordinate (P) at (220:2cm);

                  % Vertex inside the circle
                  \filldraw [fill=white] circle (2cm);

                  % Fill the minor segment with a pattern
                  \filldraw[pattern=crosshatch, pattern color=gray]
                  (K) -- (P) arc[start angle=220, end angle=320, radius=2cm] --
                  cycle;

                  % Marking the points
                  \filldraw [black] (O) circle (2pt) node[anchor=south] {O};
                  \filldraw [black] (K) circle (2pt) node[anchor=north,
                              yshift=-1mm]
                        {K};
                  \filldraw [black] (P) circle (2pt) node[anchor=north,
                              yshift=-1mm]
                        {P};

                  % Mark the angle
                  \pic [draw, angle radius=0.5cm, angle eccentricity=1.45,
                        "$110^\circ$"]
                  {angle=P--O--K};

                  % Draw lines
                  \draw (O) -- (P);
                  \draw (O) -- (K);

                  \draw (O) -- (K) node[midway, above, sloped] {$5\;cm$};
            \end{tikzpicture}

      \item \textit{(4 taškai)} Apskaičiuokite užbrūkšniuotos figūros plotą
            ($0,1\;cm^2$ tikslumu), kai į ketvirtadalį skritulio įbrėžtas kvadratas, kurio
            plotas yra $16\;cm^2$.

            \begin{tikzpicture}[scale=1.5]
                  % Define the radius
                  \def\radius{2}

                  % Calculate the side length of the square
                  \pgfmathsetmacro\squareside{\radius/sqrt(2)}

                  % Draw the quarter circle
                  \draw (0,0) -- (\radius,0) arc (0:90:\radius) -- cycle;

                  % Draw the square inside the quarter circle with increased border thickness
                  \draw[line width=0.75pt] (0,0) rectangle
                  (\squareside,\squareside);

                  % Fill the area of the quarter circle outside the square with a pattern
                  \begin{scope}
                        \fill[pattern=north east lines, pattern color=gray]
                        (0,0) --
                        (\radius,0) arc (0:90:\radius) -- cycle;
                        \clip (0,0) rectangle (\squareside,\squareside);
                        \fill[white] (0,0) -- (\radius,0) arc (0:90:\radius) --
                        cycle;
                  \end{scope}

                  % Label the vertices of the square with circles and text
                  \filldraw (0,0) circle (1pt) node[below left] {B};
                  \filldraw (\squareside,0) circle (1pt) node[below
                              right] {C};
                  \filldraw (\squareside,\squareside) circle (1pt)
                  node[above right]
                        {D};
                  \filldraw (0,\squareside) circle (1pt) node[above
                              left] {E};
            \end{tikzpicture}
\end{enumerate}

\begin{small}
      \begin{enumerate*}[label={(\arabic*)}]
            \item \textbf{Visur užrašykite atsakymus} ($Ats\ldots$);
            \item Jokio sukčiavimo. Negalima naudotis užrašais, vadovėliais,
            elektroniniais prietaisais;
            \item Jokio kalbėjimo;
            \item Rašyti aiškiai, nedviprasmiškai;
            \item Galima naudotis tik savo skaičiuotuvu ir formulių lapu;
      \end{enumerate*}
\end{small}

\end{document}