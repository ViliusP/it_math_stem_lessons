\mathchardef\period=\mathcode`.
\documentclass[a4paper]{article}
\usepackage[top=.75cm, bottom=.75cm, left=.1cm, right=.1cm]{geometry}

\usepackage{parskip} % Package to tweak paragraph skipping
\usepackage{tikz} % Package for drawing
\usepackage{tkz-euclide}
\usepackage{siunitx}
\usepackage{wrapfig}
\usepackage{graphicx}

\usetikzlibrary{fit,positioning}
\usetikzlibrary{arrows.meta}
\usetikzlibrary{patterns,patterns.meta}
\usepackage[inline]{enumitem}
\usepackage{amsmath,amssymb}
\usepackage{tasks}
\usepackage{amsmath}
\usepackage{hyperref}
\usepackage[main=lithuanian, german, shorthands=off]{babel}
\usepackage{tgpagella}
\usepackage[L7x,T1]{fontenc}
\usepackage[utf8]{inputenc}
\usepackage{enumitem}
\usepackage{booktabs} % For better looking tables
\usepackage{venndiagram}
\usepackage{subfig}
\usepackage{multirow}
\usepackage{tabularray}
\usepackage{lipsum}
\usepackage{fancyhdr}

\usepackage{array}   % For custom column definitions

% Define a new column type with padding
\newcolumntype{M}[1]{>{\centering\arraybackslash}m{#1}}

\usepackage{blindtext}
\usepackage{adjustbox}
\AfterEndEnvironment{wrapfigure}{\setlength{\intextsep}{0mm}}

\usepackage{icomma}

% L for Left, you can also use R for Right or C for Center
\setlength{\headheight}{0.5pt} % Adjust the head height
\renewcommand{\headrulewidth}{0.4pt} % Line under the header
\renewcommand{\footrulewidth}{0.4pt} % Line above the footer
% Header | Footer 

\newcommand{\germanqq}[1]{{\selectlanguage{german}\glqq#1\grqq\selectlanguage{english}}}

\DeclareMathOperator{\tg}{tg}
\newcommand{\tgx}{\tg x}

\DeclareMathOperator{\arctg}{arctg}
\newcommand{\arctgx}{\arctg x}

\makeatletter
\newcommand*{\rom}[1]{\expandafter\@slowromancap\romannumeral #1@}
\makeatother

\title{Kontrolinis darbas - progresijos}
\author{Vilius Paliokas}
\date{2023/10/17}

\setlist{after=\vspace{\baselineskip}}

% Title spacing
\usepackage{titlesec}
\titlespacing*{\subsection}{0pt}{\baselineskip}{0.5\baselineskip}
% ------------------------ 

\begin{document}

\noindent
\fbox{%
      \begin{minipage}{0.38\textwidth}
            \renewcommand{\arraystretch}{1.5} % Adjust the vertical spacing (1.5 is an example)
            \centering
            \begin{tabular}{|c|M{1.5cm}|M{1.5cm}|M{1.5cm}|}
                  \hline
                  Kampas               & $\sin \angle A$ &
                  $\cos \angle A$ & $\tg \angle A $
                  \\
                  \hline
                  $30^\circ$           & $\frac{1}{2}$        &
                  $\frac{\sqrt{3}}{2}$ & $\frac{\sqrt{3}}{3}$
                  \\
                  \hline
                  $45^\circ$           & $\frac{\sqrt{2}}{2}$ &
                  $\frac{\sqrt{2}}{2}$ & $1$
                  \\ \hline
                  $60^\circ$           & $\frac{\sqrt{3}}{2}$ &
                  $\frac{1}{2}$
                                       & $\sqrt{3}$
                  \\ \hline
            \end{tabular}
      \end{minipage}
      \begin{minipage}{0.25\textwidth}
            \begin{tikzpicture}[thick, scale=1]
                  % Define the coordinates of the triangle
                  \coordinate (B) at (0,0);
                  \coordinate (A) at (4,0);
                  \coordinate (C) at (0,3);

                  % Draw the sides of the triangle
                  \draw (A) -- (B) -- (C) -- cycle;

                  % Label the vertices
                  \node at (A) [below right] {$A$};
                  \node at (B) [below left] {$B$};
                  \node at (C) [above] {$C$};

                  % Draw the right angle symbol at A
                  \draw (B) rectangle +(0.3,0.3);

                  % Mark the angle at A
                  \pic[draw=black, angle eccentricity=1.5, angle radius=.75cm]
                  {angle=C--A--B};
            \end{tikzpicture}
      \end{minipage}%
      \begin{minipage}{0.35\textwidth}
            \Large
            $\sin \angle A = \frac{BC}{AC}
                  =\frac{\text{statinis priešais }	\angle
                        A}{\text{įžambinė}}$ \\
            $\cos \angle A = \frac{AB}{AC} =\frac{\text{statinis prie } \angle
                        A}{\text{įžambinė}}$	\\
            $\tg \angle A = \frac{BC}{AB} =\frac{\text{statinis priešais }	\angle
                        A}{\text{statinis prie } \angle A}$
      \end{minipage}%
}

\fbox{%
      \begin{minipage}{0.38\textwidth}
            \renewcommand{\arraystretch}{1.5} % Adjust the vertical spacing (1.5 is an example)
            \centering
            \begin{tabular}{|c|M{1.5cm}|M{1.5cm}|M{1.5cm}|}
                  \hline
                  Kampas               & $\sin \angle A$ &
                  $\cos \angle A$ & $\tg \angle A $
                  \\
                  \hline
                  $30^\circ$           & $\frac{1}{2}$        &
                  $\frac{\sqrt{3}}{2}$ & $\frac{\sqrt{3}}{3}$
                  \\
                  \hline
                  $45^\circ$           & $\frac{\sqrt{2}}{2}$ &
                  $\frac{\sqrt{2}}{2}$ & $1$
                  \\ \hline
                  $60^\circ$           & $\frac{\sqrt{3}}{2}$ &
                  $\frac{1}{2}$
                                       & $\sqrt{3}$
                  \\ \hline
            \end{tabular}
      \end{minipage}
      \begin{minipage}{0.25\textwidth}
            \begin{tikzpicture}[thick, scale=1]
                  % Define the coordinates of the triangle
                  \coordinate (B) at (0,0);
                  \coordinate (A) at (4,0);
                  \coordinate (C) at (0,3);

                  % Draw the sides of the triangle
                  \draw (A) -- (B) -- (C) -- cycle;

                  % Label the vertices
                  \node at (A) [below right] {$A$};
                  \node at (B) [below left] {$B$};
                  \node at (C) [above] {$C$};

                  % Draw the right angle symbol at A
                  \draw (B) rectangle +(0.3,0.3);

                  % Mark the angle at A
                  \pic[draw=black, angle eccentricity=1.5, angle radius=.75cm]
                  {angle=C--A--B};
            \end{tikzpicture}
      \end{minipage}%
      \begin{minipage}{0.35\textwidth}
            \Large
            $\sin \angle A = \frac{BC}{AC}
                  =\frac{\text{statinis priešais }	\angle
                        A}{\text{įžambinė}}$ \\
            $\cos \angle A = \frac{AB}{AC} =\frac{\text{statinis prie } \angle
                        A}{\text{įžambinė}}$	\\
            $\tg \angle A = \frac{BC}{AB} =\frac{\text{statinis priešais }	\angle
                        A}{\text{statinis prie } \angle A}$
      \end{minipage}%
}

\fbox{%
      \begin{minipage}{0.38\textwidth}
            \renewcommand{\arraystretch}{1.5} % Adjust the vertical spacing (1.5 is an example)
            \centering
            \begin{tabular}{|c|M{1.5cm}|M{1.5cm}|M{1.5cm}|}
                  \hline
                  Kampas               & $\sin \angle A$ &
                  $\cos \angle A$ & $\tg \angle A $
                  \\
                  \hline
                  $30^\circ$           & $\frac{1}{2}$        &
                  $\frac{\sqrt{3}}{2}$ & $\frac{\sqrt{3}}{3}$
                  \\
                  \hline
                  $45^\circ$           & $\frac{\sqrt{2}}{2}$ &
                  $\frac{\sqrt{2}}{2}$ & $1$
                  \\ \hline
                  $60^\circ$           & $\frac{\sqrt{3}}{2}$ &
                  $\frac{1}{2}$
                                       & $\sqrt{3}$
                  \\ \hline
            \end{tabular}
      \end{minipage}
      \begin{minipage}{0.25\textwidth}
            \begin{tikzpicture}[thick, scale=1]
                  % Define the coordinates of the triangle
                  \coordinate (B) at (0,0);
                  \coordinate (A) at (4,0);
                  \coordinate (C) at (0,3);

                  % Draw the sides of the triangle
                  \draw (A) -- (B) -- (C) -- cycle;

                  % Label the vertices
                  \node at (A) [below right] {$A$};
                  \node at (B) [below left] {$B$};
                  \node at (C) [above] {$C$};

                  % Draw the right angle symbol at A
                  \draw (B) rectangle +(0.3,0.3);

                  % Mark the angle at A
                  \pic[draw=black, angle eccentricity=1.5, angle radius=.75cm]
                  {angle=C--A--B};
            \end{tikzpicture}
      \end{minipage}%
      \begin{minipage}{0.35\textwidth}
            \Large
            $\sin \angle A = \frac{BC}{AC}
                  =\frac{\text{statinis priešais }	\angle
                        A}{\text{įžambinė}}$ \\
            $\cos \angle A = \frac{AB}{AC} =\frac{\text{statinis prie } \angle
                        A}{\text{įžambinė}}$	\\
            $\tg \angle A = \frac{BC}{AB} =\frac{\text{statinis priešais }	\angle
                        A}{\text{statinis prie } \angle A}$
      \end{minipage}%
}

\fbox{%
      \begin{minipage}{0.38\textwidth}
            \renewcommand{\arraystretch}{1.5} % Adjust the vertical spacing (1.5 is an example)
            \centering
            \begin{tabular}{|c|M{1.5cm}|M{1.5cm}|M{1.5cm}|}
                  \hline
                  Kampas               & $\sin \angle A$ &
                  $\cos \angle A$ & $\tg \angle A $
                  \\
                  \hline
                  $30^\circ$           & $\frac{1}{2}$        &
                  $\frac{\sqrt{3}}{2}$ & $\frac{\sqrt{3}}{3}$
                  \\
                  \hline
                  $45^\circ$           & $\frac{\sqrt{2}}{2}$ &
                  $\frac{\sqrt{2}}{2}$ & $1$
                  \\ \hline
                  $60^\circ$           & $\frac{\sqrt{3}}{2}$ &
                  $\frac{1}{2}$
                                       & $\sqrt{3}$
                  \\ \hline
            \end{tabular}
      \end{minipage}
      \begin{minipage}{0.25\textwidth}
            \begin{tikzpicture}[thick, scale=1]
                  % Define the coordinates of the triangle
                  \coordinate (B) at (0,0);
                  \coordinate (A) at (4,0);
                  \coordinate (C) at (0,3);

                  % Draw the sides of the triangle
                  \draw (A) -- (B) -- (C) -- cycle;

                  % Label the vertices
                  \node at (A) [below right] {$A$};
                  \node at (B) [below left] {$B$};
                  \node at (C) [above] {$C$};

                  % Draw the right angle symbol at A
                  \draw (B) rectangle +(0.3,0.3);

                  % Mark the angle at A
                  \pic[draw=black, angle eccentricity=1.5, angle radius=.75cm]
                  {angle=C--A--B};
            \end{tikzpicture}
      \end{minipage}%
      \begin{minipage}{0.35\textwidth}
            \Large
            $\sin \angle A = \frac{BC}{AC}
                  =\frac{\text{statinis priešais }	\angle
                        A}{\text{įžambinė}}$ \\
            $\cos \angle A = \frac{AB}{AC} =\frac{\text{statinis prie } \angle
                        A}{\text{įžambinė}}$	\\
            $\tg \angle A = \frac{BC}{AB} =\frac{\text{statinis priešais }	\angle
                        A}{\text{statinis prie } \angle A}$
      \end{minipage}%
}

\fbox{%
      \begin{minipage}{0.38\textwidth}
            \renewcommand{\arraystretch}{1.5} % Adjust the vertical spacing (1.5 is an example)
            \centering
            \begin{tabular}{|c|M{1.5cm}|M{1.5cm}|M{1.5cm}|}
                  \hline
                  Kampas               & $\sin \angle A$ &
                  $\cos \angle A$ & $\tg \angle A $
                  \\
                  \hline
                  $30^\circ$           & $\frac{1}{2}$        &
                  $\frac{\sqrt{3}}{2}$ & $\frac{\sqrt{3}}{3}$
                  \\
                  \hline
                  $45^\circ$           & $\frac{\sqrt{2}}{2}$ &
                  $\frac{\sqrt{2}}{2}$ & $1$
                  \\ \hline
                  $60^\circ$           & $\frac{\sqrt{3}}{2}$ &
                  $\frac{1}{2}$
                                       & $\sqrt{3}$
                  \\ \hline
            \end{tabular}
      \end{minipage}
      \begin{minipage}{0.25\textwidth}
            \begin{tikzpicture}[thick, scale=1]
                  % Define the coordinates of the triangle
                  \coordinate (B) at (0,0);
                  \coordinate (A) at (4,0);
                  \coordinate (C) at (0,3);

                  % Draw the sides of the triangle
                  \draw (A) -- (B) -- (C) -- cycle;

                  % Label the vertices
                  \node at (A) [below right] {$A$};
                  \node at (B) [below left] {$B$};
                  \node at (C) [above] {$C$};

                  % Draw the right angle symbol at A
                  \draw (B) rectangle +(0.3,0.3);

                  % Mark the angle at A
                  \pic[draw=black, angle eccentricity=1.5, angle radius=.75cm]
                  {angle=C--A--B};
            \end{tikzpicture}
      \end{minipage}%
      \begin{minipage}{0.35\textwidth}
            \Large
            $\sin \angle A = \frac{BC}{AC}
                  =\frac{\text{statinis priešais }	\angle
                        A}{\text{įžambinė}}$ \\
            $\cos \angle A = \frac{AB}{AC} =\frac{\text{statinis prie } \angle
                        A}{\text{įžambinė}}$	\\
            $\tg \angle A = \frac{BC}{AB} =\frac{\text{statinis priešais }	\angle
                        A}{\text{statinis prie } \angle A}$
      \end{minipage}%
}

\fbox{%
      \begin{minipage}{0.38\textwidth}
            \renewcommand{\arraystretch}{1.5} % Adjust the vertical spacing (1.5 is an example)
            \centering
            \begin{tabular}{|c|M{1.5cm}|M{1.5cm}|M{1.5cm}|}
                  \hline
                  Kampas               & $\sin \angle A$ &
                  $\cos \angle A$ & $\tg \angle A $
                  \\
                  \hline
                  $30^\circ$           & $\frac{1}{2}$        &
                  $\frac{\sqrt{3}}{2}$ & $\frac{\sqrt{3}}{3}$
                  \\
                  \hline
                  $45^\circ$           & $\frac{\sqrt{2}}{2}$ &
                  $\frac{\sqrt{2}}{2}$ & $1$
                  \\ \hline
                  $60^\circ$           & $\frac{\sqrt{3}}{2}$ &
                  $\frac{1}{2}$
                                       & $\sqrt{3}$
                  \\ \hline
            \end{tabular}
      \end{minipage}
      \begin{minipage}{0.25\textwidth}
            \begin{tikzpicture}[thick, scale=1]
                  % Define the coordinates of the triangle
                  \coordinate (B) at (0,0);
                  \coordinate (A) at (4,0);
                  \coordinate (C) at (0,3);

                  % Draw the sides of the triangle
                  \draw (A) -- (B) -- (C) -- cycle;

                  % Label the vertices
                  \node at (A) [below right] {$A$};
                  \node at (B) [below left] {$B$};
                  \node at (C) [above] {$C$};

                  % Draw the right angle symbol at A
                  \draw (B) rectangle +(0.3,0.3);

                  % Mark the angle at A
                  \pic[draw=black, angle eccentricity=1.5, angle radius=.75cm]
                  {angle=C--A--B};
            \end{tikzpicture}
      \end{minipage}%
      \begin{minipage}{0.35\textwidth}
            \Large
            $\sin \angle A = \frac{BC}{AC}
                  =\frac{\text{statinis priešais }	\angle
                        A}{\text{įžambinė}}$ \\
            $\cos \angle A = \frac{AB}{AC} =\frac{\text{statinis prie } \angle
                        A}{\text{įžambinė}}$	\\
            $\tg \angle A = \frac{BC}{AB} =\frac{\text{statinis priešais }	\angle
                        A}{\text{statinis prie } \angle A}$
      \end{minipage}%
}

\end{document}