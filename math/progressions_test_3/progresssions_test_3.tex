\mathchardef\period=\mathcode`.
\documentclass[a4paper]{article}
\usepackage[top=1.45cm, bottom=1cm, left=1cm, right=1cm]{geometry}

\usepackage{parskip} % Package to tweak paragraph skipping
\usepackage{tikz} % Package for drawing
\usepackage{tkz-euclide}
\usepackage{siunitx}
\usepackage{wrapfig}
\usepackage{graphicx}

\usetikzlibrary{fit,positioning}
\usetikzlibrary{arrows.meta}
\usetikzlibrary{patterns,patterns.meta}
\usepackage[inline]{enumitem}
\usepackage{amsmath,amssymb}
\usepackage{tasks}
\usepackage{amsmath}
\usepackage{hyperref}
\usepackage[main=lithuanian, german, shorthands=off]{babel}
\usepackage{tgpagella}
\usepackage[L7x,T1]{fontenc}
\usepackage[utf8]{inputenc}
\usepackage{enumitem}
\usepackage{booktabs} % For better looking tables
\usepackage{venndiagram}
\usepackage{subfig}
\usepackage{multirow}
\usepackage{tabularray}
\usepackage{lipsum}
\usepackage{fancyhdr}

\usepackage{blindtext}
\usepackage{adjustbox}
\AfterEndEnvironment{wrapfigure}{\setlength{\intextsep}{0mm}}

\usepackage{icomma}

% Header | Footer 
\fancyhf{} % clear all header and footer fields
\fancyhead[R]{skaičių seka | aritmetinė progresija | geometrinė progresijų suma
      | kontrolinis darbas}
% L for Left, you can also use R for Right or C for Center
\fancyfoot[R]{skaičių seka | aritmetinė progresija | geometrinė progresijų suma
      | kontrolinis darbas}

% L for Left, you can also use R for Right or C for Center
\setlength{\headheight}{0.5pt} % Adjust the head height
\renewcommand{\headrulewidth}{0.4pt} % Line under the header
\renewcommand{\footrulewidth}{0.4pt} % Line above the footer
% Header | Footer 

\newcommand{\germanqq}[1]{{\selectlanguage{german}\glqq#1\grqq\selectlanguage{english}}}

\DeclareMathOperator{\tg}{tg}
\newcommand{\tgx}{\tg x}

\DeclareMathOperator{\arctg}{arctg}
\newcommand{\arctgx}{\arctg x}

\makeatletter
\newcommand*{\rom}[1]{\expandafter\@slowromancap\romannumeral #1@}
\makeatother

\title{Kontrolinis darbas - progresijos}
\author{Vilius Paliokas}
\date{2023/10/17}

\setlist{after=\vspace{\baselineskip}}

% Title spacing
\usepackage{titlesec}
\titlespacing*{\subsection}{0pt}{\baselineskip}{0.5\baselineskip}
% ------------------------ 

\begin{document}
\thispagestyle{fancy}

\titlespacing*{\subsection}{0pt}{.75ex}{0.75ex}

\subsection*{3 variantas}

\begin{enumerate}
      \item Geometrinės progresijos n-tojo nario formulė yra $b_{n}=4 \cdot
                  2^{n-1}$.

            \begin{tasks}[item-format={\normalfont}, after-item-skip=2mm](1)
                  \task \textit{(1 taškas)} Apskaičiuokite šios progresijos
                  šeštąjį narį;
                  \task \textit{(1 taškas)} Raskite k reikšmę, kai $b_k=1024$;
            \end{tasks}

      \item \textit{(2 taškai)} Nurodykite, ar pateikta skaičių seka yra
            aritmetinė progresija, geometrinė progresija ar nei aritmetinė, nei
            geometrinė progresija:
            \begin{tasks}[item-format={\normalfont}, after-item-skip=2mm](2)
                  \task $\frac{1}{10}; \frac{1}{100}; \frac{1}{1000};
                        \frac{1}{10000}$;
                  \task $2,5;;6,25;15,39,0625$;
                  \task $2; 4; 9; 16; 25$;
                  \task $-1;-2;-3;-4;-5$;
            \end{tasks}

      \item \textit{(1 taškas)} Nei mažėjančios, nei didėjančios geometrinės
            progresijos pirmasis
            narys lygus -2, o trečiasis lygus -72. Kam lygus antrasis šios
            progresijos narys?

      \item \textit{(1 taškas)} Kuri iš žemiau aprašytų sekų yra
            \underline{mažėjanti} aritmetinė progresija:
            \begin{tasks}[item-format={\normalfont}, after-item-skip=2mm,
                        label=\Alph*, label-format={\bfseries}](2)
                  \task $c_n = 3n-1$;
                  \task $-1; -5; -25; -75$;
                  \task $-8; -5; -2; 1; 4$;
                  \task $b_n=b_1\cdot q^{n-1}$, $b_1 = 5$, $q = 5,5$;
            \end{tasks}
      \item \textit{(1 taškas)}  Duota $n$ skirtingų natūraliųjų skaičių,
            sudarančių didėjančią
            aritmetinę progresiją. Skaičius $n$ yra ne mažesnis už 3. Ar šių
            skaičių suma
            gali būti lygi 21? Jeigu yra, pateikite tokią skaičių seką.
      \item \textit{(1 taškas)}  Yra žinomi du pirmieji aritmetinės progresijos
            nariai: $a_1 = 3$ ir $a_2 = 18$. Apskaičiuokite penktąjį šios
            progresijos narį
            $a_5$.
      \item Aritmetinės progresijos $a_1, a_2, a_3, \ldots$ pirmųjų $n$ narių
            suma yra $S_{n} = 4n^2-3n$.
            \begin{tasks}[item-format={\normalfont}, after-item-skip=2mm](2)
                  \task \textit{(1 taškas)} Apskaičiuokite $S_5$ ir $S_6$
                  reikšmes.
                  \task \textit{(1 taškas)} Apskaičiuokite $a_1$ ir $a_6$
                  reikšmes;
            \end{tasks}
      \item \textit{(3 taškai)} Raskite geometrinės progresijos 10 pirmųjų
            narių
            suma, kai sekos pirmieji trys nariai yra $x-12$, $x-4$,
            $-\frac{x}{35}$;
\end{enumerate}

\begin{small}
      \begin{enumerate*}[label={(\arabic*)}]
            \item \textbf{Visur}, išskyrus įrodymus, \textbf{užrašykite
                  atsakymus} ($Ats\ldots$);
            \item Jokio sukčiavimo. Negalima naudotis užrašais, vadovėliais,
            elektroniniais prietaisais;
            \item Jokio kalbėjimo;
            \item Rašyti aiškiai, nedviprasmiškai;
            \item Galima naudotis tik savo skaičiuotuvu ir formulių lapu;
      \end{enumerate*}
\end{small}

\subsection*{3 variantas}

\begin{enumerate}
      \item Geometrinės progresijos n-tojo nario formulė yra $b_{n}=4 \cdot
                  2^{n-1}$.

            \begin{tasks}[item-format={\normalfont}, after-item-skip=2mm](1)
                  \task \textit{(1 taškas)} Apskaičiuokite šios progresijos
                  šeštąjį narį;
                  \task \textit{(1 taškas)} Raskite k reikšmę, kai $b_k=1024$;
            \end{tasks}

      \item \textit{(2 taškai)} Nurodykite, ar pateikta skaičių seka yra
            aritmetinė progresija, geometrinė progresija ar nei aritmetinė, nei
            geometrinė progresija:
            \begin{tasks}[item-format={\normalfont}, after-item-skip=2mm](2)
                  \task $\frac{1}{10}; \frac{1}{100}; \frac{1}{1000};
                        \frac{1}{10000}$;
                  \task $2,5;;6,25;15,39,0625$;
                  \task $2; 4; 9; 16; 25$;
                  \task $-1;-2;-3;-4;-5$;
            \end{tasks}

      \item \textit{(1 taškas)} Nei mažėjančios, nei didėjančios geometrinės
            progresijos pirmasis
            narys lygus -2, o trečiasis lygus -72. Kam lygus antrasis šios
            progresijos narys?

      \item \textit{(1 taškas)} Kuri iš žemiau aprašytų sekų yra
            \underline{mažėjanti} aritmetinė progresija:
            \begin{tasks}[item-format={\normalfont}, after-item-skip=2mm,
                        label=\Alph*, label-format={\bfseries}](2)
                  \task $c_n = 3n-1$;
                  \task $-1; -5; -25; -75$;
                  \task $-8; -5; -2; 1; 4$;
                  \task $b_n=b_1\cdot q^{n-1}$, $b_1 = 5$, $q = 5,5$;
            \end{tasks}
      \item \textit{(1 taškas)}  Duota $n$ skirtingų natūraliųjų skaičių,
            sudarančių didėjančią
            aritmetinę progresiją. Skaičius $n$ yra ne mažesnis už 3. Ar šių
            skaičių suma
            gali būti lygi 21? Jeigu yra, pateikite tokią skaičių seką.
      \item \textit{(1 taškas)}  Yra žinomi du pirmieji aritmetinės progresijos
            nariai: $a_1 = 3$ ir $a_2 = 18$. Apskaičiuokite penktąjį šios
            progresijos narį
            $a_5$.
      \item Aritmetinės progresijos $a_1, a_2, a_3, \ldots$ pirmųjų $n$ narių
            suma yra $S_{n} = 4n^2-3n$.
            \begin{tasks}[item-format={\normalfont}, after-item-skip=2mm](2)
                  \task \textit{(1 taškas)} Apskaičiuokite $S_5$ ir $S_6$
                  reikšmes.
                  \task \textit{(1 taškas)} Apskaičiuokite $a_1$ ir $a_6$
                  reikšmes;
            \end{tasks}
      \item \textit{(3 taškai)} Raskite geometrinės progresijos 10 pirmųjų
            narių
            suma, kai sekos pirmieji trys nariai yra $x-12$, $x-4$,
            $-\frac{x}{35}$;
\end{enumerate}

\begin{small}
      \begin{enumerate*}[label={(\arabic*)}]
            \item \textbf{Visur}, išskyrus įrodymus, \textbf{užrašykite
                  atsakymus} ($Ats\ldots$);
            \item Jokio sukčiavimo. Negalima naudotis užrašais, vadovėliais,
            elektroniniais prietaisais;
            \item Jokio kalbėjimo;
            \item Rašyti aiškiai, nedviprasmiškai;
            \item Galima naudotis tik savo skaičiuotuvu ir formulių lapu;
      \end{enumerate*}
\end{small}

\end{document}