\mathchardef\period=\mathcode`.
\documentclass[a4paper]{article}
\usepackage[top=1.45cm, bottom=1cm, left=1cm, right=1cm]{geometry}

\usepackage{parskip} % Package to tweak paragraph skipping
\usepackage{tikz} % Package for drawing
\usepackage{tkz-euclide}
\usepackage{siunitx}
\usepackage{wrapfig}
\usepackage{graphicx}

\usetikzlibrary{fit,positioning}
\usetikzlibrary{arrows.meta}
\usetikzlibrary{patterns,patterns.meta}
\usepackage[inline]{enumitem}
\usepackage{amsmath,amssymb}
\usepackage{tasks}
\usepackage{amsmath}
\usepackage{hyperref}
\usepackage[main=lithuanian, german, shorthands=off]{babel}
\usepackage{tgpagella}
\usepackage[L7x,T1]{fontenc}
\usepackage[utf8]{inputenc}
\usepackage{enumitem}
\usepackage{booktabs} % For better looking tables
\usepackage{venndiagram}
\usepackage{subfig}
\usepackage{multirow}
\usepackage{tabularray}
\usepackage{lipsum}
\usepackage{fancyhdr}

\usepackage{blindtext}
\usepackage{adjustbox}
\AfterEndEnvironment{wrapfigure}{\setlength{\intextsep}{0mm}}
\usepackage{afterpage}

\usepackage{icomma}

% Header | Footer 
\fancyhf{} % clear all header and footer fields
\fancyhead[R]{pusmečio pabaiga | kontrolinis darbas}
% L for Left, you can also use R for Right or C for Center
\fancyfoot[R]{pusmečio pabaiga | kontrolinis darbas}

% L for Left, you can also use R for Right or C for Center
\setlength{\headheight}{0.5pt} % Adjust the head height
\renewcommand{\headrulewidth}{0.4pt} % Line under the header
\renewcommand{\footrulewidth}{0.4pt} % Line above the footer
% Header | Footer 

\newcommand{\germanqq}[1]{{\selectlanguage{german}\glqq#1\grqq\selectlanguage{english}}}

\DeclareMathOperator{\tg}{tg}
\newcommand{\tgx}{\tg x}

\DeclareMathOperator{\arctg}{arctg}
\newcommand{\arctgx}{\arctg x}

\makeatletter
\newcommand*{\rom}[1]{\expandafter\@slowromancap\romannumeral #1@}
\makeatother

\title{Kontrolinis darbas - progresijos}
\author{Vilius Paliokas}
\date{2023/10/17}

\setlist{after=\vspace{\baselineskip}}

% Title spacing
\usepackage{titlesec}
\titlespacing*{\subsection}{0pt}{\baselineskip}{0.5\baselineskip}
% ------------------------ 

\newcommand\blankpage{%
      \null
      \thispagestyle{empty}%
      \addtocounter{page}{-1}%
      \newpage}

\begin{document}
\thispagestyle{fancy}

\titlespacing*{\subsection}{0pt}{.75ex}{0.75ex}

\subsection*{3 variantas}

\begin{enumerate}
      \item \textit{(1 taškas)} Duots dvi skaičiųs aibės: $A=\{-5; -1; 0; 2; 3;
                  5; 7\}$ ir $B=\{-5; -2; 0; 1; 7; 9\}$. Nustatykite, kiek
            elementų priklauso aibių
            $A$ ir $B$ sankirtai.
            \begin{tasks}[item-format={\normalfont}, after-item-skip=2mm,
                        label=\Alph*, label-format={\bfseries}](4)
                  \task 3
                  \task 7
                  \task 10
                  \task 13
            \end{tasks}

      \item \textit{(1 taškas)} Su kuria $b$ reikšme lygybė yra teisinga
            $\sqrt[6]{33}\cdot\sqrt[3]{33}=33^b$?
            \begin{tasks}[item-format={\normalfont}, after-item-skip=2mm,
                        label=\Alph*, label-format={\bfseries}](4)
                  \task $b=-2$
                  \task $b=2$
                  \task $b=\frac{1}{2}$
                  \task $b=\frac{1}{4}$
            \end{tasks}

      \item \textit{(1 taškas)} Su kuria $a$ reikšme lygybė yra teisinga
            $2\log_{3} 4-\log_{3} 8 + 2= \log_{3} a$?
            \begin{tasks}[item-format={\normalfont}, after-item-skip=2mm,
                        label=\Alph*, label-format={\bfseries}](4)
                  \task $a=12$
                  \task $a=-2$
                  \task $a=18$
                  \task $a=6$
            \end{tasks}

      \item \textit{(1 taškas)} Kam lygu pusė $8^{22}$ skaičiaus? Atsakymą
            pateikite laipsniu $a^n$; čia $n \in \mathbb{N}$.
            \begin{tasks}[item-format={\normalfont}, after-item-skip=2mm,
                        label=\Alph*, label-format={\bfseries}](4)
                  \task $2^{66}$
                  \task $8^{11}$
                  \task $2^{65}$
                  \task $4^{65}$
            \end{tasks}
      \item \textit{(1 taškas)} Skaičių $\sqrt[4]{7^6}$ parašykite
            $a\cdot\sqrt[4]{b}$ pavidalu; čia $a, b \in \mathbb{N}$.
            \begin{tasks}[item-format={\normalfont}, after-item-skip=2mm,
                        label=\Alph*, label-format={\bfseries}](4)
                  \task $4\sqrt[4]{49}$
                  \task $49\sqrt[4]{343}$
                  \task $49\sqrt[4]{343}$
                  \task $7\sqrt[4]{49}$
            \end{tasks}
            % Adjust the width as needed

            \begin{minipage}[t]{0.725\linewidth}
                  \item \textit{(1 taškas)} Paveiksle pavaizduotas vienetinis
                  apskritimas, kurio centras yra taškas $O$. Remdamiesi
                  paveikslu, nustatykite
                  taško $M$ koordinates.
                  \vspace{2mm}
                  \begin{tasks}[item-format={\normalfont},
                              after-item-skip=3mm,
                              label=\Alph*), label-format={\bfseries},
                              column-sep=10pt](2)
                        \task $(-\frac{\sqrt{3}}{2};-\frac{1}{2})$
                        \task $(-\frac{1}{2};-\frac{\sqrt{3}}{2})$
                        \task $(-\frac{1}{2};-\frac{\sqrt{3}}{2})$
                        \task $(-\frac{\sqrt{3}}{2};\frac{1}{2})$
                  \end{tasks}

            \end{minipage}
            \begin{minipage}[t]{0.25\linewidth}
                  \adjustbox{valign=t}{
                        \begin{tikzpicture}[font=\small, scale=1.4]
                              % Coordinate plane
                              \draw[->] (-1.5,0) -- (1.5,0) node[below] {$x$};
                              \draw[->] (0,-1.5) -- (0,1.5) node[left] {$y$};

                              % Unit circle (thicker)
                              \draw[line width=0.6pt] (0,0) circle (1);

                              % Coordinates on the unit circle
                              \node[above left] at (0,0) {$O$};
                              \node[above right, yshift=-.5pt, xshift=-2pt] at
                              (0,1) {$1$};
                              \node[below right, yshift=.5pt, xshift=-2pt] at
                              (1,0) {$1$};
                              \node[below left, yshift=.5pt, xshift=2pt] at
                              (0,-1) {$-1$};
                              \node[below left, yshift=-.5pt, xshift=2pt] at
                              (-1,0) {$-1$};

                              % Radius
                              \draw (0,0) -- (-150:1) node[midway, left] {};

                              % Angle arc with arrow at -120 degrees
                              \draw[->] (0.4,0) arc (0:-150:0.4);

                              % Angle label at -120 degrees
                              \node at (-60:0.6) {-150\textdegree};

                              \draw[fill=black] (-150:1) circle (0.8pt)
                              node[below, yshift=-.5pt, xshift=-2pt] {$M$};
                        \end{tikzpicture}
                  }
            \end{minipage}

      \item \textit{(1 taškas)} Reiškinys $log_{0,3}(3x-2)$ turi prasmę, kai.
            \begin{tasks}[item-format={\normalfont}, after-item-skip=2mm,
                        label=\Alph*, label-format={\bfseries}](4)
                  \task $x\in(\frac{2}{3};\infty)$
                  \task $x\in(2;\infty)$
                  \task $x\in(-\infty;3)$
                  \task $x\in(-\infty;\frac{2}{3})$
            \end{tasks}
      \item \textit{(1 taškas)} Yra žinoma, kad $\cos\alpha=-0,8$ ir
            $180^\circ<\alpha<270^\circ$. Tuomet $\sin\alpha=$
            \begin{tasks}[item-format={\normalfont}, after-item-skip=2mm,
                        label=\Alph*, label-format={\bfseries}](4)
                  \task -0,6
                  \task -0,4
                  \task 0,4
                  \task 0,6
            \end{tasks}
      \item \textit{(1 taškas)} Pagal receptą, varškės spurgoms pagaminti
            reikia $500\,gr.$, $200\,gr.$ miltų, $3\,vnt$ kiaušinių... Klarkas
            turi $300\,gr.$
            varškės. Kiek gramų miltų reikės Klarkui, jeigu jis gamins varškės
            spurgas pagal šį receptą?
            \begin{tasks}[item-format={\normalfont}, after-item-skip=2mm,
                        label=\Alph*, label-format={\bfseries}](5)
                  \task 180 gr.
                  \task 250 gr.
                  \task 120 gr.
                  \task 160 gr.
                  \task 100 gr.
            \end{tasks}
      \item \textit{(1 taškas)} Duotos dvi aibės: $A=\{-3;-2;-1;0;1;3;4\}$ ir
            $B=\{-5;-3;-1;3;5\}$. Kam lygus aibių $B$ ir $A$ skirtumas?
            \vspace{7mm}

      \item \textit{(1 taškas)} Didžiausia galima reiškinio
            $\frac{12}{3+\sin^2\alpha}$ reikmė yra:
            \begin{tasks}[item-format={\normalfont}, after-item-skip=2mm,
                        label=\Alph*, label-format={\bfseries}](6)
                  \task 5
                  \task 6
                  \task 12
                  \task 1
                  \task 3
                  \task 4
            \end{tasks}

      \item \textit{(1 taškas)} Panaikinkite iracionalumą vardiklyje
            $\frac{3}{\sqrt{7}-2}$.
            \vspace{7mm}

      \item \textit{(1 taškas)}   Eksperimento metu išmatuotas šviesos greitis
            buvo $3.00 \cdot 10^{8}$
            m/s, o tikrasis šviesos greitis yra $2.998 \cdot 10^{8}$ m/s.
            Apskaičiuokite matavimo absoliučiąją paklaidą.
            \vspace{7mm}

      \item \textit{(2 taškas)} Kiekvienas sekos narys, pradedant nuo antrojo,
            gaunamas prieš
            tai
            buvusį narį sumažinus $20\%$. Yra žinoma, kad šeštasis šios sekos
            narys lygus
            2048. Raskite septintąjį šios sekos narį.
            \vspace{7mm}

      \item \textit{(2 taškas)} Geometrinės progresijos $6x+2$, $x+3$, $x-3$,
            ... nariai yra
            neteigiami skaičiai. Raskite pirmųjų trijų skaičių sumą.
            \vspace{7mm}

\end{enumerate}

\begin{small}
      \begin{enumerate*}[label={(\arabic*)}, topsep=0pt, partopsep=0pt]
            \item \textbf{Visur}, išskyrus įrodymus, \textbf{užrašykite
                  atsakymus} ($Ats\ldots$);
            \item Jokio sukčiavimo. Negalima naudotis užrašais, vadovėliais,
            elektroniniais prietaisais;
            \item Jokio kalbėjimo;
            \item Rašyti aiškiai, nedviprasmiškai;
            \item Galima naudotis tik savo skaičiuotuvu ir formulių lapu;
      \end{enumerate*}
\end{small}

\pagenumbering{gobble}

\def\width{19}
\def\hauteur{28}

\begin{tikzpicture}[x=1cm, y=1cm, semitransparent]
      % \draw[step=1mm, line width=0.1mm, black!30!white] (0,0) grid (\width,\hauteur);
      \draw[step=4mm, line width=0.2mm, black!60!white] (0,0) grid
      (\width,\hauteur);
      % \draw[step=5cm, line width=0.5mm, black!50!white] (0,0) grid (\width,\hauteur);
      % \draw[step=1cm, line width=0.3mm, black!90!white] (0,0) grid (\width,\hauteur);
\end{tikzpicture}
\end{document}