% !TEX root = ./equation_test_2.tex
\documentclass[a4paper]{article}
\usepackage[top=1.45cm, bottom=1cm, left=1cm, right=1cm]{geometry}
\mathchardef\period=\mathcode`.

\usepackage{parskip} % Package to tweak paragraph skipping
\usepackage{siunitx}

\usepackage[inline]{enumitem}
\usepackage{amsmath,amssymb}
\usepackage{tasks}
\usepackage{amsmath}
\usepackage{hyperref}
\usepackage[main=lithuanian, german, shorthands=off]{babel}
\usepackage{tgpagella}
\usepackage[L7x,T1]{fontenc}
\usepackage[utf8]{inputenc}
\usepackage{enumitem}
\usepackage{lipsum}
\usepackage{fancyhdr}

\usepackage{blindtext}
\usepackage{adjustbox}
\AfterEndEnvironment{wrapfigure}{\setlength{\intextsep}{0mm}}

\usepackage{icomma}

% Header | Footer 
\fancyhf{} % clear all header and footer fields
\fancyhead[R]{kurso kartojimas | reiškinių persitvarkymas | savarankiškas darbas}
% L for Left, you can also use R for Right or C for Center
\fancyfoot[R]{kurso kartojimas | reiškinių persitvarkymas | savarankiškas darbas}

% L for Left, you can also use R for Right or C for Center
\setlength{\headheight}{0.5pt} % Adjust the head height
\renewcommand{\headrulewidth}{0.4pt} % Line under the header
\renewcommand{\footrulewidth}{0.4pt} % Line above the footer
% Header | Footer 

\newcommand{\germanqq}[1]{{\selectlanguage{german}\glqq#1\grqq\selectlanguage{english}}}

\DeclareMathOperator{\tg}{tg}
\newcommand{\tgx}{\tg x}

\DeclareMathOperator{\arctg}{arctg}
\newcommand{\arctgx}{\arctg x}

\makeatletter
\newcommand*{\rom}[1]{\expandafter\@slowromancap\romannumeral #1@}
\makeatother

\title{Kontrolinis darbas - lygtys}
\author{Vilius Paliokas}
\date{2024/05/010}

\setlist{after=\vspace{\baselineskip}}

% Title spacing
\usepackage{titlesec}
\titlespacing*{\subsection}{0pt}{\baselineskip}{0.5\baselineskip}
% ------------------------ 

% Tasjks

\begin{document}
\thispagestyle{fancy}

\titlespacing*{\subsection}{0pt}{.75ex}{0.75ex}

\subsection*{1 variantas}

\textit{Visi uždaviniai verti 1 taško.}

\begin{enumerate}
      \item Suprastinkite reiškinius.
            \begin{tasks}[item-format={\normalfont}, after-item-skip=2mm](4)
                  \task $\frac{x-8}{15x}\cdot \frac{5}{x-8}$;
                  \task $(-2x^4y)^3:(-5xy)^2$;
                  \task $\frac{a^2-25}{10+3a-a^2}$;
                  \task $(\frac{x}{2}-\frac{2}{x}\cdot \frac{10x}{x-2})$;
            \end{tasks}

      \item Subendravardiklinkite trupmenas ir atlikite veiksmus.
            \begin{tasks}[item-format={\normalfont}, after-item-skip=2mm](2)
                  \task $\frac{a}{2a-b}+\frac{3a-b}{b-2a}$;
                  \task $\frac{8}{15x}-\frac{4}{7x^2}$;
            \end{tasks}

      \item Išskaidykite dauginamaisiais.
            \begin{tasks}[item-format={\normalfont}, after-item-skip=2mm](2)
                  \task $-6ab+9b^2$;
                  \task $m^2-2m-15$;
            \end{tasks}

      \item Išspręskite tiesinę lygtį.
            \begin{tasks}[item-format={\normalfont}, after-item-skip=2mm](2)
                  \task $3x-2(x-7)=x+14$;
                  \task $\frac{5x-4}{2}-\frac{2x+1}{3}=-\frac{1}{5}(x-29)$;
            \end{tasks}

      \item Išspręskite nelygybę $\frac{1}{2}(6-4x)>-9-2x$;
      \item Raskite didžiausią sveikąjį skaičių, su kuriuo reiškinio $\frac{x-48}{2}-5x$
            reikšmė yra teigiama;
\end{enumerate}

\vspace*{12mm}

\subsection*{1 variantas}

\textit{Visi uždaviniai verti 1 taško.}


\begin{enumerate}
      \item Suprastinkite reiškinius.
            \begin{tasks}[item-format={\normalfont}, after-item-skip=2mm](4)
                  \task $\frac{x-8}{15x}\cdot \frac{5}{x-8}$;
                  \task $(-2x^4y)^3:(-5xy)^2$;
                  \task $\frac{a^2-25}{10+3a-a^2}$;
                  \task $(\frac{x}{2}-\frac{2}{x}\cdot \frac{10x}{x-2})$;
            \end{tasks}

      \item Subendravardiklinkite trupmenas ir atlikite veiksmus.
            \begin{tasks}[item-format={\normalfont}, after-item-skip=2mm](2)
                  \task $\frac{a}{2a-b}+\frac{3a-b}{b-2a}$;
                  \task $\frac{8}{15x}-\frac{4}{7x^2}$;
            \end{tasks}

      \item Išskaidykite dauginamaisiais.
            \begin{tasks}[item-format={\normalfont}, after-item-skip=2mm](2)
                  \task $-6ab+9b^2$;
                  \task $m^2-2m-15$;
            \end{tasks}

      \item Išspręskite tiesinę lygtį.
            \begin{tasks}[item-format={\normalfont}, after-item-skip=2mm](2)
                  \task $3x-2(x-7)=x+14$;
                  \task $\frac{5x-4}{2}-\frac{2x+1}{3}=-\frac{1}{5}(x-29)$;
            \end{tasks}

      \item Išspręskite nelygybę $\frac{1}{2}(6-4x)>-9-2x$;
      \item Raskite didžiausią sveikąjį skaičių, su kuriuo reiškinio $\frac{x-48}{2}-5x$
            reikšmė yra teigiama;
\end{enumerate}

\vspace*{12mm}

\subsection*{1 variantas}

\textit{Visi uždaviniai verti 1 taško.}


\begin{enumerate}
      \item Suprastinkite reiškinius.
            \begin{tasks}[item-format={\normalfont}, after-item-skip=2mm](4)
                  \task $\frac{x-8}{15x}\cdot \frac{5}{x-8}$;
                  \task $(-2x^4y)^3:(-5xy)^2$;
                  \task $\frac{a^2-25}{10+3a-a^2}$;
                  \task $(\frac{x}{2}-\frac{2}{x}\cdot \frac{10x}{x-2})$;
            \end{tasks}

      \item Subendravardiklinkite trupmenas ir atlikite veiksmus.
            \begin{tasks}[item-format={\normalfont}, after-item-skip=2mm](2)
                  \task $\frac{a}{2a-b}+\frac{3a-b}{b-2a}$;
                  \task $\frac{8}{15x}-\frac{4}{7x^2}$;
            \end{tasks}

      \item Išskaidykite dauginamaisiais.
            \begin{tasks}[item-format={\normalfont}, after-item-skip=2mm](2)
                  \task $-6ab+9b^2$;
                  \task $m^2-2m-15$;
            \end{tasks}

      \item Išspręskite tiesinę lygtį.
            \begin{tasks}[item-format={\normalfont}, after-item-skip=2mm](2)
                  \task $3x-2(x-7)=x+14$;
                  \task $\frac{5x-4}{2}-\frac{2x+1}{3}=-\frac{1}{5}(x-29)$;
            \end{tasks}

      \item Išspręskite nelygybę $\frac{1}{2}(6-4x)>-9-2x$;
      \item Raskite didžiausią sveikąjį skaičių, su kuriuo reiškinio $\frac{x-48}{2}-5x$
            reikšmė yra teigiama;
\end{enumerate}

\end{document}