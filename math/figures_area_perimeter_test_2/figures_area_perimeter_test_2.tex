\mathchardef\period=\mathcode`.
\documentclass[a4paper]{article}
\usepackage[top=1.45cm, bottom=1cm, left=1cm, right=1cm]{geometry}

\usepackage{parskip} % Package to tweak paragraph skipping
\usepackage{tikz} % Package for drawing
\usepackage{tkz-euclide}
\usepackage{siunitx}
\usepackage{wrapfig}
\usepackage{graphicx}
\usepackage{array}
\usepackage{changepage}

\usepackage{pgfplots}
\usetikzlibrary{fit,positioning}
\usetikzlibrary{arrows.meta}
\usetikzlibrary{patterns,patterns.meta}
\usetikzlibrary{calc,intersections}
\usetikzlibrary{angles, quotes}
\usetikzlibrary{shapes}
\usetikzlibrary{intersections,pgfplots.fillbetween}
\usepackage[inline]{enumitem}
\usepackage{amsmath,amssymb}
\usepackage{tasks}
\usepackage{amsmath}
\usepackage{hyperref}
\usepackage[main=lithuanian, german, shorthands=off]{babel}
\usepackage{tgpagella}
\usepackage[L7x,T1]{fontenc}
\usepackage[utf8]{inputenc}
\usepackage{enumitem}
\usepackage{booktabs} % For better looking tables
\usepackage{venndiagram}
\usepackage{subfig}
\usepackage{multirow}
\usepackage{tabularray}
\usepackage{lipsum}
\usepackage{fancyhdr}

\usepackage{blindtext}
\usepackage{adjustbox}
\AfterEndEnvironment{wrapfigure}{\setlength{\intextsep}{0mm}}

\usepackage{icomma}

% Header | Footer 
\fancyhf{} % clear all header and footer fields
\fancyhead[R]{plotas | perimetras | skritulys | trikampis | keturkampiai }
\fancyfoot[R]{plotas | perimetras | skritulys | trikampis | keturkampiai}
\setlength{\headheight}{0.5pt} % Adjust the head height
\renewcommand{\headrulewidth}{0.4pt} % Line under the header
\renewcommand{\footrulewidth}{0.4pt} % Line above the footer
% Header | Footer 

\newcommand{\germanqq}[1]{{\selectlanguage{german}\glqq#1\grqq\selectlanguage{english}}}

\DeclareMathOperator{\tg}{tg}
\newcommand{\tgx}{\tg x}

\DeclareMathOperator{\arctg}{arctg}
\newcommand{\arctgx}{\arctg x}

\makeatletter
\newcommand*{\rom}[1]{\expandafter\@slowromancap\romannumeral #1@}
\makeatother

\title{Planimetrija, Kontrolinis darbas, plotas ir perimtras }
\author{Vilius Paliokas}
\date{2024/01/17}

\setlist{after=\vspace{\baselineskip}}

% Title spacing
\usepackage{titlesec}
\titlespacing*{\subsection}{0pt}{.75ex}{0.75ex}

% ------------------------ 

\begin{document}
\thispagestyle{fancy}

\subsection*{2 variantas}

\begin{itemize}[after=\vspace{-\baselineskip}]

      \item[--] Jeigu reikia, pirmiausia nubrėžiamas brėžinys ir suteikiamos
            raidės viršunėms arba kraštinėmis.

      \item[--] Užrašoma naudojama teorema, formulė ar taisyklė (jeigu
            naudojama teorema, turi būti parašyta: \germanqq{pagal
                  \ldots } arba \germanqq{kadangi \ldots } arba \germanqq{nes
                  \ldots }). Jeigu buvo
            nubrėžtas brėžinys, formulėje, teoremoje ar taisyklėje naudojamos
            brėžinio raides.
\end{itemize}

\vspace{0.25cm}
\par\noindent\rule{\textwidth}{0.5pt}

\begin{enumerate}
      \item \textit{(1 taškai)} Kvadrato įstrižainė lygi 18. Kam lygus šio
            kvadratos plotas?

      \item \textit{(1 taškai)} Puskritulio skersmuo lygus $11\;cm$. Koks šio
            puskritulio plotas?

\end{enumerate}

\begin{minipage}{0.5\textwidth}
      \begin{enumerate}
            \setcounter{enumi}{2} % This continues the numbering from the previous enumerate
            \item \textit{(1 taškas)} Stačiakampio $ABCD$ įstrižainė yra
                  $24\;cm$
                  ilgio, o kampo tarp įstrižainės $AC$ ir  kraštinės $AD$ lygus
                  santykis lygus $\frac{\sqrt{3}}{2}$.
                  Apskaičiuokite stačiakampio plotą.

                  \begin{tikzpicture}[thick, scale=.18]
                        % The diagonal forms a 30-60-90 triangle
                        % The sides of the rectangle are in the ratio 1 : sqrt(3)
                        % If the diagonal (hypotenuse) is 24, the sides are:
                        % shorter side (opposite to 30°) = 24 / 2 = 12
                        % longer side (opposite to 60°) = 12 * sqrt(3)
                        \pgfmathsetmacro{\shorterside}{24 / 2}
                        \pgfmathsetmacro{\longerside}{\shorterside * sqrt(3)}

                        % Define coordinates
                        \coordinate (A) at (0,0);
                        \coordinate (B) at (\longerside,0);
                        \coordinate (C) at (\longerside,\shorterside);
                        \coordinate (D) at (0,\shorterside);

                        % Draw rectangle
                        \draw (A) -- (B) -- (C) -- (D) -- cycle;

                        % Draw diagonal
                        \draw (A) -- (C);

                        % Label points
                        \node at (A) [below left] {$A$};
                        \node at (B) [below right] {$B$};
                        \node at (C) [above right] {$C$};
                        \node at (D) [above left] {$D$};

                        % Label diagonal
                        \draw (A) -- node[sloped, above, midway] {$24\,cm$}
                        (C);

                        % Mark the 60 degree angle
                        \draw (B) rectangle ++(-1, 1);
                        \draw pic[draw=black, -, angle eccentricity=1, angle
                                    radius=.5cm] {angle=C--A--D};
                  \end{tikzpicture}
      \end{enumerate}
\end{minipage}
\begin{minipage}{0.5\textwidth}
      \begin{enumerate}
            \setcounter{enumi}{3} % This continues the numbering from the previous enumerate
            \item \textit{(3 taškai)} Paveiksle pavaizduota lygiašonė trapecija
                  ABCD
                  ($AB||CD$, $AD = BC$). Yra žinoma, kad $AB =
                        12$,
                  $CD = 24$,
                  $\angle DAB = 135^\circ$. Apskaičiuokite trapecijos plotą ir
                  perimetrą.

                  \begin{tikzpicture}[scale=.16, thick]
                        % Define coordinates
                        \coordinate (A) at (0,0);
                        \coordinate (B) at (12,0);
                        \coordinate (C) at (18,-12);
                        % BC = 12
                        \coordinate (D) at (-6,-12);
                        % CD = 16
                        % AB = 25
                        % DA = 15, but this is not directly used because AB || CD

                        % Draw trapezoid
                        \draw (A) -- (B) -- (C) -- (D) -- cycle;

                        % Label points
                        \node at (A) [above left] {$A$};
                        \node at (B) [above right] {$B$};
                        \node at (C) [below right] {$C$};
                        \node at (D) [below left] {$D$};

                        \tkzMarkSegment[color=black,pos=.5,mark=|](A,D)
                        \tkzMarkSegment[color=black,pos=.5,mark=|](B,C)

                        % Label sides
                        % \draw (B) -- node[left] {$12$} (C);
                        \draw (C) -- node[below] {$24$} (D);
                        \draw (A) -- node[above] {$12$} (B);
                        % \draw (D) -- node[below right] {$15$} (A);
                        \pic [draw, -,
                              angle radius=3mm, angle eccentricity=2,
                              "$135^\circ$"] {angle = D--A--B};
                  \end{tikzpicture}
      \end{enumerate}
\end{minipage}

\begin{enumerate}
      \setcounter{enumi}{4} % This continues the numbering from the previous enumerate
      \item \textit{(3 taškai)} Lygiagretainio $ABCD$ aukštinė $BE$ daliją
            kraštinę $AD$ į atkarpas $AE = 18\;cm$ ir $ED = 12\;cm$. Vienas
            lygiagretainio kampas lygus $60^\circ$. Apskaičiuokite
            lygiagretainio:

            \begin{tasks}[item-format={\normalfont}, after-item-skip=2mm](2)
                  \task perimetrą;
                  \task plotą.
            \end{tasks}
\end{enumerate}

\begin{small}
      \begin{enumerate*}[label={(\arabic*)}]
            \item \textbf{Visur}, išskyrus teorijos klausimus ir įrodymus,
            \textbf{užrašykite atsakymus} ($Ats\ldots$);
            \item Jokio sukčiavimo. Negalima naudotis užrašais, vadovėliais,
            elektroniniais prietaisais;
            \item Jokio kalbėjimo;
            \item Rašyti aiškiai, nedviprasmiškai;
            \item Galima naudotis tik savo skaičiuotuvu ir formulių lapu;
      \end{enumerate*}
\end{small}
\subsection*{2 variantas}

\begin{itemize}[after=\vspace{-\baselineskip}]

      \item[--] Jeigu reikia, pirmiausia nubrėžiamas brėžinys ir suteikiamos
            raidės viršunėms arba kraštinėmis.

      \item[--] Užrašoma naudojama teorema, formulė ar taisyklė (jeigu
            naudojama teorema, turi būti parašyta: \germanqq{pagal
                  \ldots } arba \germanqq{kadangi \ldots } arba \germanqq{nes
                  \ldots }). Jeigu buvo
            nubrėžtas brėžinys, formulėje, teoremoje ar taisyklėje naudojamos
            brėžinio raides.
\end{itemize}

\vspace{0.25cm}
\par\noindent\rule{\textwidth}{0.5pt}

\begin{enumerate}
      \item \textit{(1 taškai)} Kvadrato įstrižainė lygi 18. Kam lygus šio
            kvadratos plotas?

      \item \textit{(1 taškai)} Puskritulio skersmuo lygus $11\;cm$. Koks šio
            puskritulio plotas?

\end{enumerate}

\begin{minipage}{0.5\textwidth}
      \begin{enumerate}
            \setcounter{enumi}{2} % This continues the numbering from the previous enumerate
            \item \textit{(1 taškas)} Stačiakampio $ABCD$ įstrižainė yra
                  $24\;cm$
                  ilgio, o kampo tarp įstrižainės $AC$ ir  kraštinės $AD$ lygus
                  santykis lygus $\frac{\sqrt{3}}{2}$.
                  Apskaičiuokite stačiakampio plotą.

                  \begin{tikzpicture}[thick, scale=.18]
                        % The diagonal forms a 30-60-90 triangle
                        % The sides of the rectangle are in the ratio 1 : sqrt(3)
                        % If the diagonal (hypotenuse) is 24, the sides are:
                        % shorter side (opposite to 30°) = 24 / 2 = 12
                        % longer side (opposite to 60°) = 12 * sqrt(3)
                        \pgfmathsetmacro{\shorterside}{24 / 2}
                        \pgfmathsetmacro{\longerside}{\shorterside * sqrt(3)}

                        % Define coordinates
                        \coordinate (A) at (0,0);
                        \coordinate (B) at (\longerside,0);
                        \coordinate (C) at (\longerside,\shorterside);
                        \coordinate (D) at (0,\shorterside);

                        % Draw rectangle
                        \draw (A) -- (B) -- (C) -- (D) -- cycle;

                        % Draw diagonal
                        \draw (A) -- (C);

                        % Label points
                        \node at (A) [below left] {$A$};
                        \node at (B) [below right] {$B$};
                        \node at (C) [above right] {$C$};
                        \node at (D) [above left] {$D$};

                        % Label diagonal
                        \draw (A) -- node[sloped, above, midway] {$24\,cm$}
                        (C);

                        % Mark the 60 degree angle
                        \draw (B) rectangle ++(-1, 1);
                        \draw pic[draw=black, -, angle eccentricity=1, angle
                                    radius=.5cm] {angle=C--A--D};
                  \end{tikzpicture}
      \end{enumerate}
\end{minipage}
\begin{minipage}{0.5\textwidth}
      \begin{enumerate}
            \setcounter{enumi}{3} % This continues the numbering from the previous enumerate
            \item \textit{(3 taškai)} Paveiksle pavaizduota lygiašonė trapecija
                  ABCD
                  ($AB||CD$, $AD = BC$). Yra žinoma, kad $AB =
                        12$,
                  $CD = 24$,
                  $\angle DAB = 135^\circ$. Apskaičiuokite trapecijos plotą ir
                  perimetrą.

                  \begin{tikzpicture}[scale=.16, thick]
                        % Define coordinates
                        \coordinate (A) at (0,0);
                        \coordinate (B) at (12,0);
                        \coordinate (C) at (18,-12);
                        % BC = 12
                        \coordinate (D) at (-6,-12);
                        % CD = 16
                        % AB = 25
                        % DA = 15, but this is not directly used because AB || CD

                        % Draw trapezoid
                        \draw (A) -- (B) -- (C) -- (D) -- cycle;

                        % Label points
                        \node at (A) [above left] {$A$};
                        \node at (B) [above right] {$B$};
                        \node at (C) [below right] {$C$};
                        \node at (D) [below left] {$D$};

                        \tkzMarkSegment[color=black,pos=.5,mark=|](A,D)
                        \tkzMarkSegment[color=black,pos=.5,mark=|](B,C)

                        % Label sides
                        % \draw (B) -- node[left] {$12$} (C);
                        \draw (C) -- node[below] {$24$} (D);
                        \draw (A) -- node[above] {$12$} (B);
                        % \draw (D) -- node[below right] {$15$} (A);
                        \pic [draw, -,
                              angle radius=3mm, angle eccentricity=2,
                              "$135^\circ$"] {angle = D--A--B};
                  \end{tikzpicture}
      \end{enumerate}
\end{minipage}

\begin{enumerate}
      \setcounter{enumi}{4} % This continues the numbering from the previous enumerate
      \item \textit{(3 taškai)} Lygiagretainio $ABCD$ aukštinė $BE$ daliją
            kraštinę $AD$ į atkarpas $AE = 18\;cm$ ir $ED = 12\;cm$. Vienas
            lygiagretainio kampas lygus $60^\circ$. Apskaičiuokite
            lygiagretainio:

            \begin{tasks}[item-format={\normalfont}, after-item-skip=2mm](2)
                  \task perimetrą;
                  \task plotą.
            \end{tasks}
\end{enumerate}

\begin{small}
      \begin{enumerate*}[label={(\arabic*)}]
            \item \textbf{Visur}, išskyrus teorijos klausimus ir įrodymus,
            \textbf{užrašykite atsakymus} ($Ats\ldots$);
            \item Jokio sukčiavimo. Negalima naudotis užrašais, vadovėliais,
            elektroniniais prietaisais;
            \item Jokio kalbėjimo;
            \item Rašyti aiškiai, nedviprasmiškai;
            \item Galima naudotis tik savo skaičiuotuvu ir formulių lapu;
      \end{enumerate*}
\end{small}

\end{document}