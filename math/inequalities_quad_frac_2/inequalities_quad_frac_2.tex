\mathchardef\period=\mathcode`.
\documentclass[a4paper]{article}
\usepackage[top=1.45cm, bottom=1cm, left=1cm, right=1cm]{geometry}

\usepackage{parskip} % Package to tweak paragraph skipping
\usepackage{tikz} % Package for drawing
\usepackage{tkz-euclide}
\usepackage{siunitx}
\usepackage{wrapfig}
\usepackage{graphicx}

\usetikzlibrary{fit,positioning}
\usetikzlibrary{arrows.meta}
\usetikzlibrary{patterns,patterns.meta}
\usepackage[inline]{enumitem}
\usepackage{amsmath,amssymb}
\usepackage{tasks}
\usepackage{amsmath}
\usepackage{hyperref}
\usepackage[main=lithuanian, german, shorthands=off]{babel}
\usepackage{tgpagella}
\usepackage[L7x,T1]{fontenc}
\usepackage[utf8]{inputenc}
\usepackage{enumitem}
\usepackage{booktabs} % For better looking tables
\usepackage{venndiagram}
\usepackage{subfig}
\usepackage{multirow}
\usepackage{tabularray}
\usepackage{lipsum}
\usepackage{fancyhdr}

\usepackage{blindtext}
\usepackage{adjustbox}
\AfterEndEnvironment{wrapfigure}{\setlength{\intextsep}{0mm}}

\usepackage{icomma}

% Header | Footer 
\fancyhf{} % clear all header and footer fields
\fancyhead[R]{lygtys | kontrolinis darbas}
% L for Left, you can also use R for Right or C for Center
\fancyfoot[R]{lygtys | kontrolinis darbas}

% L for Left, you can also use R for Right or C for Center
\setlength{\headheight}{0.5pt} % Adjust the head height
\renewcommand{\headrulewidth}{0.4pt} % Line under the header
\renewcommand{\footrulewidth}{0.4pt} % Line above the footer
% Header | Footer 

\newcommand{\germanqq}[1]{{\selectlanguage{german}\glqq#1\grqq\selectlanguage{english}}}

\DeclareMathOperator{\tg}{tg}
\newcommand{\tgx}{\tg x}

\DeclareMathOperator{\arctg}{arctg}
\newcommand{\arctgx}{\arctg x}

\makeatletter
\newcommand*{\rom}[1]{\expandafter\@slowromancap\romannumeral #1@}
\makeatother

\title{Kontrolinis darbas - progresijos}
\author{Vilius Paliokas}
\date{2023/10/17}

\setlist{after=\vspace{\baselineskip}}

% Title spacing
\usepackage{titlesec}
\titlespacing*{\subsection}{0pt}{\baselineskip}{0.5\baselineskip}
% ------------------------ 

\begin{document}
\thispagestyle{fancy}

\titlespacing*{\subsection}{0pt}{.75ex}{0.75ex}

\subsection*{2 variantas}

\begin{enumerate}
      \item Išspręskite nelygybes (\textit{po 1 tašką}).

            \begin{tasks}[item-format={\normalfont}, after-item-skip=2mm](2)
                  \task $\frac{x+2}{3-x} \geqslant 0$;
                  \task $x^2<16x$;
                  \task $\frac{2x+2}{x-3}<2$;
                  \task $\frac{3}{4-x}<0$;
                  \task $-x^2+4x-4 \leqslant 0$;
                  \task $(5x+4)(5-x)>0$;
                  \task $\frac{25}{10-x} \leqslant x$;
                  \task $-(x-5)^2 \leqslant 0$;
                  \task $9x^2+9x \leq 0$;
            \end{tasks}

      \item \textit{(1 taškas)} Raskite didžiausią sveikąjį skaičių, su kuriuo reikšinio $\frac{4-2x}{5}-x$ reikšmė yra neigiama;
      \item \textit{(1 taškas)} Raskite nelygybės $x^2<13$ mažiausią neigiamą sveikąjį sprendinį.
\end{enumerate}

\begin{small}
      \begin{enumerate*}[label={(\arabic*)}]
            \item \textbf{Visur} \textbf{nurodykite atsakymus} ($Ats\ldots$);
            \item Jokio sukčiavimo. Negalima naudotis užrašais, vadovėliais,
            elektroniniais prietaisais;
            \item Jokio kalbėjimo;
            \item Rašyti aiškiai, nedviprasmiškai;
            \item Galima naudotis tik savo skaičiuotuvu ir formulių lapu;
      \end{enumerate*}
\end{small}

\vspace*{12mm}

\subsection*{2 variantas}

\begin{enumerate}
      \item Išspręskite nelygybes (\textit{po 1 tašką}).

            \begin{tasks}[item-format={\normalfont}, after-item-skip=2mm](2)
                  \task $\frac{x+2}{3-x} \geqslant 0$;
                  \task $x^2<16x$;
                  \task $\frac{2x+2}{x-3}<2$;
                  \task $\frac{3}{4-x}<0$;
                  \task $-x^2+4x-4 \leqslant 0$;
                  \task $(5x+4)(5-x)>0$;
                  \task $\frac{25}{10-x} \leqslant x$;
                  \task $-(x-5)^2 \leqslant 0$;
                  \task $9x^2+9x \leq 0$;
            \end{tasks}

      \item \textit{(1 taškas)} Raskite didžiausią sveikąjį skaičių, su kuriuo reikšinio $\frac{4-2x}{5}-x$ reikšmė yra neigiama;
      \item \textit{(1 taškas)} Raskite nelygybės $x^2<13$ mažiausią neigiamą sveikąjį sprendinį.
\end{enumerate}

\begin{small}
      \begin{enumerate*}[label={(\arabic*)}]
            \item \textbf{Visur} \textbf{nurodykite atsakymus} ($Ats\ldots$);
            \item Jokio sukčiavimo. Negalima naudotis užrašais, vadovėliais,
            elektroniniais prietaisais;
            \item Jokio kalbėjimo;
            \item Rašyti aiškiai, nedviprasmiškai;
            \item Galima naudotis tik savo skaičiuotuvu ir formulių lapu;
      \end{enumerate*}
\end{small}

\vspace*{12mm}

\subsection*{2 variantas}

\begin{enumerate}
      \item Išspręskite nelygybes (\textit{po 1 tašką}).

            \begin{tasks}[item-format={\normalfont}, after-item-skip=2mm](2)
                  \task $\frac{x+2}{3-x} \geqslant 0$;
                  \task $x^2<16x$;
                  \task $\frac{2x+2}{x-3}<2$;
                  \task $\frac{3}{4-x}<0$;
                  \task $-x^2+4x-4 \leqslant 0$;
                  \task $(5x+4)(5-x)>0$;
                  \task $\frac{25}{10-x} \leqslant x$;
                  \task $-(x-5)^2 \leqslant 0$;
                  \task $9x^2+9x \leq 0$;
            \end{tasks}

      \item \textit{(1 taškas)} Raskite didžiausią sveikąjį skaičių, su kuriuo reikšinio $\frac{4-2x}{5}-x$ reikšmė yra neigiama;
      \item \textit{(1 taškas)} Raskite nelygybės $x^2<13$ mažiausią neigiamą sveikąjį sprendinį.
\end{enumerate}

\begin{small}
      \begin{enumerate*}[label={(\arabic*)}]
            \item \textbf{Visur} \textbf{nurodykite atsakymus} ($Ats\ldots$);
            \item Jokio sukčiavimo. Negalima naudotis užrašais, vadovėliais,
            elektroniniais prietaisais;
            \item Jokio kalbėjimo;
            \item Rašyti aiškiai, nedviprasmiškai;
            \item Galima naudotis tik savo skaičiuotuvu ir formulių lapu;
      \end{enumerate*}
\end{small}

\end{document}