% !TEX root = ./equation_test_2.tex
\documentclass[a4paper]{article}
\usepackage[top=1.45cm, bottom=1cm, left=1cm, right=1cm]{geometry}
\mathchardef\period=\mathcode`.

\usepackage{parskip} % Package to tweak paragraph skipping
\usepackage{siunitx}

\usepackage[inline]{enumitem}
\usepackage{amsmath,amssymb}
\usepackage{tasks}
\usepackage{amsmath}
\usepackage{hyperref}
\usepackage[main=lithuanian, german, shorthands=off]{babel}
\usepackage{tgpagella}
\usepackage[L7x,T1]{fontenc}
\usepackage[utf8]{inputenc}
\usepackage{enumitem}
\usepackage{lipsum}
\usepackage{fancyhdr}

\usepackage{blindtext}
\usepackage{adjustbox}
\AfterEndEnvironment{wrapfigure}{\setlength{\intextsep}{0mm}}

\usepackage{icomma}

% Header | Footer 
\fancyhf{} % clear all header and footer fields
\fancyhead[R]{lygtys | kontrolinis darbas}
% L for Left, you can also use R for Right or C for Center
\fancyfoot[R]{lygtys | kontrolinis darbas}

% L for Left, you can also use R for Right or C for Center
\setlength{\headheight}{0.5pt} % Adjust the head height
\renewcommand{\headrulewidth}{0.4pt} % Line under the header
\renewcommand{\footrulewidth}{0.4pt} % Line above the footer
% Header | Footer 

\newcommand{\germanqq}[1]{{\selectlanguage{german}\glqq#1\grqq\selectlanguage{english}}}

\DeclareMathOperator{\tg}{tg}
\newcommand{\tgx}{\tg x}

\DeclareMathOperator{\arctg}{arctg}
\newcommand{\arctgx}{\arctg x}

\makeatletter
\newcommand*{\rom}[1]{\expandafter\@slowromancap\romannumeral #1@}
\makeatother

\title{Kontrolinis darbas - progresijos}
\author{Vilius Paliokas}
\date{2024/03/11}

\setlist{after=\vspace{\baselineskip}}

% Title spacing
\usepackage{titlesec}
\titlespacing*{\subsection}{0pt}{\baselineskip}{0.5\baselineskip}
% ------------------------ 

\begin{document}
\thispagestyle{fancy}

\titlespacing*{\subsection}{0pt}{.75ex}{0.75ex}

\subsection*{2 variantas}

\begin{enumerate}
      \item Išspręskite lygtį.
            \begin{tasks}[item-format={\normalfont}, after-item-skip=2mm](2)
                  \task \textit{(1 taškas)} $2x^3+2=2\frac{1}{2}$;
                  \task \textit{(1 taškas)} $-4(2x+6)^3=-\frac{1}{2}$;
                  \task \textit{(1 taškas)} $2\sqrt{x}-6=-1$;
                  \task \textit{(2 taškai)} $\frac{1}{2}\sqrt{3x^2-x+20}=5$;
                  \task \textit{(1 taškas)} $5^{25-x^2}=1$;
                  \task \textit{(2 taškai)} $5^x\cdot5^{5x-4}=25$;
                  \task \textit{(1 taškai)} $\log_3(2x-4)=\log_3(4x)$;
                  \task \textit{(2 taškai)} $\log _2\left(x+1\right)+\log _2\left(2x-4\right)=3\log _23+2$;
                  \task \textit{(1 taškai)} $\frac{1}{3}\cdot 3^{x+1}=22$;
            \end{tasks}

      \item Duota funkcija $f(x)=-3\sqrt{x+4}-1$.
            \begin{tasks}[item-format={\normalfont}, after-item-skip=2mm](1)
                  \task \textit{(1 taškas)} Raskite taško, kuriame funkcijos grafikas kerta $Ox$ ašį, abcisę;
                  \task \textit{(1 taškas)} Raskite taško, kuriame funkcijos grafikas kerta $Oy$ ašį, oordinatę;
            \end{tasks}
\end{enumerate}

\begin{small}
      \begin{enumerate*}[label={(\arabic*)}]
            \item \textbf{Visur} \textbf{nurodykite atsakymus} ($Ats\ldots$);
            \item Jokio sukčiavimo. Negalima naudotis užrašais, vadovėliais,
            elektroniniais prietaisais;
            \item Jokio kalbėjimo;
            \item Rašyti aiškiai, nedviprasmiškai;
            \item Galima naudotis tik savo skaičiuotuvu ir formulių lapu;
      \end{enumerate*}
\end{small}

\vspace*{12mm}
\subsection*{2 variantas}

\begin{enumerate}
	\item Išspręskite lygtį.
	\begin{tasks}[item-format={\normalfont}, after-item-skip=2mm](2)
		\task \textit{(1 taškas)} $2x^3+2=2\frac{1}{2}$;
		\task \textit{(1 taškas)} $-4(2x+6)^3=-\frac{1}{2}$;
		\task \textit{(1 taškas)} $2\sqrt{x}-6=-1$;
		\task \textit{(2 taškai)} $\frac{1}{2}\sqrt{3x^2-x+20}=5$;
		\task \textit{(1 taškas)} $5^{25-x^2}=1$;
		\task \textit{(2 taškai)} $5^x\cdot5^{5x-4}=25$;
		\task \textit{(1 taškai)} $\log_3(2x-4)=\log_3(4x)$;
		\task \textit{(2 taškai)} $\log _2\left(x+1\right)+\log _2\left(2x-4\right)=3\log _23+2$;
		\task \textit{(1 taškai)} $\frac{1}{3}\cdot 3^{x+1}=22$;
	\end{tasks}
	
	\item Duota funkcija $f(x)=-3\sqrt{x+4}-1$.
	\begin{tasks}[item-format={\normalfont}, after-item-skip=2mm](1)
		\task \textit{(1 taškas)} Raskite taško, kuriame funkcijos grafikas kerta $Ox$ ašį, abcisę;
		\task \textit{(1 taškas)} Raskite taško, kuriame funkcijos grafikas kerta $Oy$ ašį, oordinatę;
	\end{tasks}
\end{enumerate}

\begin{small}
      \begin{enumerate*}[label={(\arabic*)}]
            \item \textbf{Visur} \textbf{nurodykite atsakymus} ($Ats\ldots$);
            \item Jokio sukčiavimo. Negalima naudotis užrašais, vadovėliais,
            elektroniniais prietaisais;
            \item Jokio kalbėjimo;
            \item Rašyti aiškiai, nedviprasmiškai;
            \item Galima naudotis tik savo skaičiuotuvu ir formulių lapu;
      \end{enumerate*}
\end{small}

\vspace*{12mm}
\subsection*{2 variantas}

\begin{enumerate}
	\item Išspręskite lygtį.
	\begin{tasks}[item-format={\normalfont}, after-item-skip=2mm](2)
		\task \textit{(1 taškas)} $2x^3+2=2\frac{1}{2}$;
		\task \textit{(1 taškas)} $-4(2x+6)^3=-\frac{1}{2}$;
		\task \textit{(1 taškas)} $2\sqrt{x}-6=-1$;
		\task \textit{(2 taškai)} $\frac{1}{2}\sqrt{3x^2-x+20}=5$;
		\task \textit{(1 taškas)} $5^{25-x^2}=1$;
		\task \textit{(2 taškai)} $5^x\cdot5^{5x-4}=25$;
		\task \textit{(1 taškai)} $\log_3(2x-4)=\log_3(4x)$;
		\task \textit{(2 taškai)} $\log _2\left(x+1\right)+\log _2\left(2x-4\right)=3\log _23+2$;
		\task \textit{(1 taškai)} $\frac{1}{3}\cdot 3^{x+1}=22$;
	\end{tasks}
	
	\item Duota funkcija $f(x)=-3\sqrt{x+4}-1$.
	\begin{tasks}[item-format={\normalfont}, after-item-skip=2mm](1)
		\task \textit{(1 taškas)} Raskite taško, kuriame funkcijos grafikas kerta $Ox$ ašį, abcisę;
		\task \textit{(1 taškas)} Raskite taško, kuriame funkcijos grafikas kerta $Oy$ ašį, oordinatę;
	\end{tasks}
\end{enumerate}

\begin{small}
      \begin{enumerate*}[label={(\arabic*)}]
            \item \textbf{Visur} \textbf{nurodykite atsakymus} ($Ats\ldots$);
            \item Jokio sukčiavimo. Negalima naudotis užrašais, vadovėliais,
            elektroniniais prietaisais;
            \item Jokio kalbėjimo;
            \item Rašyti aiškiai, nedviprasmiškai;
            \item Galima naudotis tik savo skaičiuotuvu ir formulių lapu;
      \end{enumerate*}
\end{small}


\end{document}