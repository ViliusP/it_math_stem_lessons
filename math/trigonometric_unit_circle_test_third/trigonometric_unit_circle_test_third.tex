\documentclass[a4paper]{article}
\usepackage[top=1cm, bottom=1cm, left=1cm, right=1cm]{geometry}

\usepackage{parskip} % Package to tweak paragraph skipping
\usepackage{tikz} % Package for drawing
\usepackage{tkz-euclide}
\usetikzlibrary{fit,positioning}
\usetikzlibrary{arrows.meta}
\usetikzlibrary{patterns,patterns.meta}
\usepackage[inline]{enumitem}
\usepackage{amsmath,amssymb}
\usepackage{tasks}
\usepackage{amsmath}
\usepackage{hyperref}
\usepackage[main=lithuanian, german, shorthands=off]{babel}
\usepackage{tgpagella}
\usepackage[L7x,T1]{fontenc}
\usepackage[utf8]{inputenc}
\usepackage{enumitem}
\usepackage{booktabs} % For better looking tables
\usepackage{venndiagram}
\usepackage{subfig}
\usepackage{multirow}
\usepackage{tabularray}
\newcommand{\germanqq}[1]{{\selectlanguage{german}\glqq#1\grqq\selectlanguage{english}}}

\DeclareMathOperator{\tg}{tg}
\newcommand{\tgx}{\tg x}

\DeclareMathOperator{\arctg}{arctg}
\newcommand{\arctgx}{\arctg x}

\title{Kontrolinis darbas nr. 2}
\author{Vilius Paliokas}
\date{2023/10/17}

\setlist{after=\vspace{\baselineskip}}

\begin{document}
\thispagestyle{empty}
\subsection*{3 variantas}

\begin{enumerate}
      \item Apskaičiuokite:

            \begin{tasks}[item-format={\normalfont}, after-item-skip=4mm](3)
                  \task $\arccos{\frac{1}{2}} + \arccos{\frac{\sqrt{2}}{2}} $;
                  \task $\arcsin{(-\frac{\sqrt{3}}{2})} + 2\arccos{1} $;
                  \task $\sin({4\arctg{\sqrt{3}} +
                              \arctg{\frac{3}{3\sqrt{3}}}})  $;

            \end{tasks}

      \item Kuriame ketvirtje yra posūkio kampas $\alpha$, jeigu:
            \begin{tasks}[item-format={\normalfont}, after-item-skip=4mm](2)
                  \task $\cos \alpha = \frac{1}{9}$, o $\sin \alpha < 0$;
                  \task $\tg \alpha = -\frac{2}{4}$, o $\cos \alpha < 0$;
            \end{tasks}
      \item Apskaičiuokite $\sin \alpha$, $\cos \alpha$, $\tg \alpha$, kai
            stačiakampėje koordinačių plokštumoje pasukus spindulį $OX$ kampu
            $\alpha$ spindulio taškas $A(1; 0)$ perėjo į tašką $A_{1}$, kurio
            koordinatės
            yra:
            \begin{tasks}[item-format={\normalfont}, after-item-skip=4mm](4)
                  \task $(-1; 0)$;
                  \task $(0; -1)$;
                  \task $(\frac{1}{2}; -\frac{\sqrt{3}}{2})$;
                  \task $(\frac{\sqrt{2}}{2}; -\frac{\sqrt{2}}{2})$;
            \end{tasks}

      \item Su kuriomis $x$ reikšmėmis reiškinys turi prasmę::
            \begin{tasks}[item-format={\normalfont},
                        after-item-skip=4mm](2)
                  \task $\arctg(\sqrt{x-5})$;
                  \task $\arcsin(9-4x)$;
            \end{tasks}

      \item Supaprastinkite reiškinį, kad posūkio kampas būtų nuo
            \textbf{$\boldsymbol{-90^\circ}$
                  iki  $\boldsymbol{0^\circ}$} ir tada apskaičiuokite jo
            reikšmę
            (\textbf{būtina parodyti veiksmų seką}):
            \begin{tasks}[item-format={\normalfont},
                        after-item-skip=4mm](4)
                  \task $\tg 510^\circ$;
                  \task $\tg 330^\circ$;
                  \task $\sin (-780^\circ$);
                        \task $\cos (-660^\circ$);
            \end{tasks}

      \item Kuriame koordinačių plokštumos OXY ketvirtyje ar ašyje yra
            posukūkio kampas $\alpha$, jei:

            \begin{tasks}[item-format={\normalfont},
                        after-item-skip=4mm](3)
                  \task $\alpha = -900^\circ$
                  \task $\alpha = 2000^\circ$
                  \task $\alpha = -1845^\circ$
            \end{tasks}

\end{enumerate}

\begin{table}[!htpb]
      \centering
      \begin{tblr}{
                  cell{1}{1} = {c=18}{},
                  cell{2}{1} = {c=3}{},
                  cell{2}{4} = {c=2}{},
                  cell{2}{6} = {c=4}{},
                  cell{2}{10} = {c=2}{},
                  cell{2}{12} = {c=4}{},
                  cell{2}{16} = {c=3}{},
                  hlines,
                  vlines,
            }
            Užduočių vertė &	   &	     &	       &	 &	   &
            &	      & 	&	 &	  &	   &
            &	     &	      &        &	&	 \\
            1		 &	   &	     & 2       &	 & 3	   &
            &	      & 	& 4	 &	  & 5	   &
            &	     &	      & 6      &	&	 \\
            {a\\ 1}	 & {b\\ 1} & {c\\ 2} & {a\\ 2} & {b\\ 2} & {a\\ 3} &
            {b\\			    4} & {c\\ 4} & {d\\ 4} & {a\\3} & {b\\3} & {a\\2} &
            {b\\2} & {c\\2} & {d\\2} & {a\\1} & {b\\1} & {c\\1}
      \end{tblr}
\end{table}


\begin{small}
      \begin{enumerate*}[label={(\arabic*)}]
            \item \textbf{Visur užrašykite atsakymus} ($Ats\ldots$);
            \item Jokio sukčiavimo. Negalima naudotis užrašais, vadovėliais,
            elektroniniais prietaisais;
            \item Jokio kalbėjimo;
            \item Rašyti aiškiai, nedviprasmiškai;
            \item Galima naudotis tik savo skaičiuotuvu ir formulių lapu;
      \end{enumerate*}
\end{small}
\subsection*{3 variantas}
\begin{enumerate}
      \item Apskaičiuokite:

            \begin{tasks}[item-format={\normalfont}, after-item-skip=4mm](3)
                  \task $\arccos{\frac{1}{2}} + \arccos{\frac{\sqrt{2}}{2}} $;
                  \task $\arcsin{(-\frac{\sqrt{3}}{2})} + 2\arccos{1} $;
                  \task $\sin({4\arctg{\sqrt{3}} +
                              \arctg{\frac{3}{3\sqrt{3}}}})  $;

            \end{tasks}

      \item Kuriame ketvirtje yra posūkio kampas $\alpha$, jeigu:
            \begin{tasks}[item-format={\normalfont}, after-item-skip=4mm](2)
                  \task $\cos \alpha = \frac{1}{9}$, o $\sin \alpha < 0$;
                  \task $\tg \alpha = -\frac{2}{4}$, o $\cos \alpha < 0$;
            \end{tasks}
      \item Apskaičiuokite $\sin \alpha$, $\cos \alpha$, $\tg \alpha$, kai
            stačiakampėje koordinačių plokštumoje pasukus spindulį $OX$ kampu
            $\alpha$ spindulio taškas $A(1; 0)$ perėjo į tašką $A_{1}$, kurio
            koordinatės
            yra:
            \begin{tasks}[item-format={\normalfont}, after-item-skip=4mm](4)
                  \task $(-1; 0)$;
                  \task $(0; -1)$;
                  \task $(\frac{1}{2}; -\frac{\sqrt{3}}{2})$;
                  \task $(\frac{\sqrt{2}}{2}; -\frac{\sqrt{2}}{2})$;
            \end{tasks}

      \item Su kuriomis $x$ reikšmėmis reiškinys turi prasmę::
            \begin{tasks}[item-format={\normalfont},
                        after-item-skip=4mm](2)
                  \task $\arctg(\sqrt{x-5})$;
                  \task $\arcsin(9-4x)$;
            \end{tasks}

      \item Supaprastinkite reiškinį, kad posūkio kampas būtų nuo
            \textbf{$\boldsymbol{-90^\circ}$
                  iki  $\boldsymbol{0^\circ}$} ir tada apskaičiuokite jo
            reikšmę
            (\textbf{būtina parodyti veiksmų seką}):
            \begin{tasks}[item-format={\normalfont},
                        after-item-skip=4mm](4)
                  \task $\tg 510^\circ$;
                  \task $\tg 330^\circ$;
                  \task $\sin (-780^\circ$);
                        \task $\cos (-660^\circ$);
            \end{tasks}

      \item Kuriame koordinačių plokštumos OXY ketvirtyje ar ašyje yra
            posukūkio kampas $\alpha$, jei:

            \begin{tasks}[item-format={\normalfont},
                        after-item-skip=4mm](3)
                  \task $\alpha = -900^\circ$
                  \task $\alpha = 2000^\circ$
                  \task $\alpha = -1845^\circ$
            \end{tasks}

\end{enumerate}
\begin{table}[!htpb]
      \centering
      \begin{tblr}{
                  cell{1}{1} = {c=18}{},
                  cell{2}{1} = {c=3}{},
                  cell{2}{4} = {c=2}{},
                  cell{2}{6} = {c=4}{},
                  cell{2}{10} = {c=2}{},
                  cell{2}{12} = {c=4}{},
                  cell{2}{16} = {c=3}{},
                  hlines,
                  vlines,
            }
            Užduočių vertė &	   &	     &	       &	 &	   &
            &	      & 	&	 &	  &	   &
            &	     &	      &        &	&	 \\
            1		 &	   &	     & 2       &	 & 3	   &
            &	      & 	& 4	 &	  & 5	   &
            &	     &	      & 6      &	&	 \\
            {a\\ 1}	 & {b\\ 1} & {c\\ 2} & {a\\ 2} & {b\\ 2} & {a\\ 3} &
            {b\\			    4} & {c\\ 4} & {d\\ 4} & {a\\3} & {b\\3} & {a\\2} &
            {b\\2} & {c\\2} & {d\\2} & {a\\1} & {b\\1} & {c\\1}
      \end{tblr}
\end{table}

\begin{small}
      \begin{enumerate*}[label={(\arabic*)}]
            \item \textbf{Visur užrašykite atsakymus} ($Ats\ldots$);
            \item Jokio sukčiavimo. Negalima naudotis užrašais, vadovėliais,
            elektroniniais prietaisais;
            \item Jokio kalbėjimo;
            \item Rašyti aiškiai, nedviprasmiškai;
            \item Galima naudotis tik savo skaičiuotuvu ir formulių lapu;
      \end{enumerate*}
\end{small}

\end{document}