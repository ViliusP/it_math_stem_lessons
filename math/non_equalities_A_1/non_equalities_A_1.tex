% !TEX root = ./equation_test_2.tex
\documentclass[a4paper]{article}
\usepackage[top=1.45cm, bottom=1cm, left=1cm, right=1cm]{geometry}
\mathchardef\period=\mathcode`.

\usepackage{parskip} % Package to tweak paragraph skipping
\usepackage{siunitx}

\usepackage[inline]{enumitem}
\usepackage{amsmath,amssymb}
\usepackage{tasks}
\usepackage{amsmath}
\usepackage{hyperref}
\usepackage[main=lithuanian, german, shorthands=off]{babel}
\usepackage{tgpagella}
\usepackage[L7x,T1]{fontenc}
\usepackage[utf8]{inputenc}
\usepackage{enumitem}
\usepackage{lipsum}
\usepackage{fancyhdr}

\usepackage{blindtext}
\usepackage{adjustbox}
\AfterEndEnvironment{wrapfigure}{\setlength{\intextsep}{0mm}}

\usepackage{icomma}

% Header | Footer 
\fancyhf{} % clear all header and footer fields
\fancyhead[R]{lygtys | kontrolinis darbas}
% L for Left, you can also use R for Right or C for Center
\fancyfoot[R]{lygtys | kontrolinis darbas}

% L for Left, you can also use R for Right or C for Center
\setlength{\headheight}{0.5pt} % Adjust the head height
\renewcommand{\headrulewidth}{0.4pt} % Line under the header
\renewcommand{\footrulewidth}{0.4pt} % Line above the footer
% Header | Footer 

\newcommand{\germanqq}[1]{{\selectlanguage{german}\glqq#1\grqq\selectlanguage{english}}}

\DeclareMathOperator{\tg}{tg}
\newcommand{\tgx}{\tg x}

\DeclareMathOperator{\arctg}{arctg}
\newcommand{\arctgx}{\arctg x}

\makeatletter
\newcommand*{\rom}[1]{\expandafter\@slowromancap\romannumeral #1@}
\makeatother

\title{Kontrolinis darbas - lygtys}
\author{Vilius Paliokas}
\date{2024/05/010}

\setlist{after=\vspace{\baselineskip}}

% Title spacing
\usepackage{titlesec}
\titlespacing*{\subsection}{0pt}{\baselineskip}{0.5\baselineskip}
% ------------------------ 

\begin{document}
\thispagestyle{fancy}

\titlespacing*{\subsection}{0pt}{.75ex}{0.75ex}

\subsection*{1 variantas}

Išspręskite lygtį (\textit{po 1 tašką}):
\begin{tasks}[item-format={\normalfont}, after-item-skip=2mm](2)
      \task $\frac{x-\sqrt{2}}{x^2-2}\leqslant 0$;
      \task $\frac{x-1}{x}-\frac{x+1}{x-1} \leqslant 2$;
      \task $(\frac{1}{2})^{-3x} \geqslant (\frac{1}{2})^9$;
      \task $4^{5-x} \geqslant \frac{1}{64}$;
      \task $3^{\frac{-5x}{x-1}}-9^{\frac{x-12}{2}} \leqslant 0$;
      \task $\log_{\frac{1}{2}}(4x-1) \geq -2$;
      \task $\log_{4}x-\log_{4}27 \leqslant \log_{4}\frac{1}{9}$;
      \task $\log_{6}(-x^2+9x-14) \geq 1$;
      \task $\sqrt{(x+\sqrt{2})^2} \geqslant \sqrt{50}$;
      \task Su kuriomis $y=f(x)=3|x+2|-1$ reikšmės yra mažesnės už 8;
\end{tasks}
\subsection*{1 variantas}

Išspręskite lygtį (\textit{po 1 tašką}):
\begin{tasks}[item-format={\normalfont}, after-item-skip=2mm](2)
      \task $\frac{x-\sqrt{2}}{x^2-2}\leqslant 0$;
      \task $\frac{x-1}{x}-\frac{x+1}{x-1} \leqslant 2$;
      \task $(\frac{1}{2})^{-3x} \geqslant (\frac{1}{2})^9$;
      \task $4^{5-x} \geqslant \frac{1}{64}$;
      \task $3^{\frac{-5x}{x-1}}-9^{\frac{x-12}{2}} \leqslant 0$;
      \task $\log_{\frac{1}{2}}(4x-1) \geq -2$;
      \task $\log_{4}x-\log_{4}27 \leqslant \log_{4}\frac{1}{9}$;
      \task $\log_{6}(-x^2+9x-14) \geq 1$;
      \task $\sqrt{(x+\sqrt{2})^2} \geqslant \sqrt{50}$;
      \task Su kuriomis $y=f(x)=3|x+2|-1$ reikšmės yra mažesnės už 8;
\end{tasks}

\subsection*{1 variantas}

Išspręskite lygtį (\textit{po 1 tašką}):
\begin{tasks}[item-format={\normalfont}, after-item-skip=2mm](2)
      \task $\frac{x-\sqrt{2}}{x^2-2}\leqslant 0$;
      \task $\frac{x-1}{x}-\frac{x+1}{x-1} \leqslant 2$;
      \task $(\frac{1}{2})^{-3x} \geqslant (\frac{1}{2})^9$;
      \task $4^{5-x} \geqslant \frac{1}{64}$;
      \task $3^{\frac{-5x}{x-1}}-9^{\frac{x-12}{2}} \leqslant 0$;
      \task $\log_{\frac{1}{2}}(4x-1) \geq -2$;
      \task $\log_{4}x-\log_{4}27 \leqslant \log_{4}\frac{1}{9}$;
      \task $\log_{6}(-x^2+9x-14) \geq 1$;
      \task $\sqrt{(x+\sqrt{2})^2} \geqslant \sqrt{50}$;
      \task Su kuriomis $y=f(x)=3|x+2|-1$ reikšmės yra mažesnės už 8;
\end{tasks}

\subsection*{1 variantas}

Išspręskite lygtį (\textit{po 1 tašką}):
\begin{tasks}[item-format={\normalfont}, after-item-skip=2mm](2)
      \task $\frac{x-\sqrt{2}}{x^2-2}\leqslant 0$;
      \task $\frac{x-1}{x}-\frac{x+1}{x-1} \leqslant 2$;
      \task $(\frac{1}{2})^{-3x} \geqslant (\frac{1}{2})^9$;
      \task $4^{5-x} \geqslant \frac{1}{64}$;
      \task $3^{\frac{-5x}{x-1}}-9^{\frac{x-12}{2}} \leqslant 0$;
      \task $\log_{\frac{1}{2}}(4x-1) \geq -2$;
      \task $\log_{4}x-\log_{4}27 \leqslant \log_{4}\frac{1}{9}$;
      \task $\log_{6}(-x^2+9x-14) \geq 1$;
      \task $\sqrt{(x+\sqrt{2})^2} \geqslant \sqrt{50}$;
      \task Su kuriomis $y=f(x)=3|x+2|-1$ reikšmės yra mažesnės už 8;
\end{tasks}

\subsection*{1 variantas}

Išspręskite lygtį (\textit{po 1 tašką}):
\begin{tasks}[item-format={\normalfont}, after-item-skip=2mm](2)
      \task $\frac{x-\sqrt{2}}{x^2-2}\leqslant 0$;
      \task $\frac{x-1}{x}-\frac{x+1}{x-1} \leqslant 2$;
      \task $(\frac{1}{2})^{-3x} \geqslant (\frac{1}{2})^9$;
      \task $4^{5-x} \geqslant \frac{1}{64}$;
      \task $3^{\frac{-5x}{x-1}}-9^{\frac{x-12}{2}} \leqslant 0$;
      \task $\log_{\frac{1}{2}}(4x-1) \geq -2$;
      \task $\log_{4}x-\log_{4}27 \leqslant \log_{4}\frac{1}{9}$;
      \task $\log_{6}(-x^2+9x-14) \geq 1$;
      \task $\sqrt{(x+\sqrt{2})^2} \geqslant \sqrt{50}$;
      \task Su kuriomis $y=f(x)=3|x+2|-1$ reikšmės yra mažesnės už 8;
\end{tasks}

\end{document}