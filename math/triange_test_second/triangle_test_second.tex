\documentclass[a4paper]{article}
\usepackage[top=1.45cm, bottom=1.45cm, left=1cm, right=1cm]{geometry}

\usepackage{parskip} % Package to tweak paragraph skipping
\usepackage{tikz} % Package for drawing
\usepackage{tkz-euclide}
\usepackage{siunitx}
\usepackage{wrapfig}
\usepackage{graphicx}

\usetikzlibrary{fit,positioning}
\usetikzlibrary{arrows.meta}
\usetikzlibrary{patterns,patterns.meta}
\usepackage[inline]{enumitem}
\usepackage{amsmath,amssymb}
\usepackage{tasks}
\usepackage{amsmath}
\usepackage{hyperref}
\usepackage[main=lithuanian, german, shorthands=off]{babel}
\usepackage{tgpagella}
\usepackage[L7x,T1]{fontenc}
\usepackage[utf8]{inputenc}
\usepackage{enumitem}
\usepackage{booktabs} % For better looking tables
\usepackage{venndiagram}
\usepackage{subfig}
\usepackage{multirow}
\usepackage{tabularray}
\usepackage{lipsum}
\usepackage{fancyhdr}

\usepackage{blindtext}
\usepackage{adjustbox}
\AfterEndEnvironment{wrapfigure}{\setlength{\intextsep}{0mm}}

% Header | Footer 
\fancyhf{} % clear all header and footer fields
\fancyhead[R]{pagrindinė trigonometrinė tapatybė | sinusų teorema | kosinusių
      teorema | trikampio plotas | planimetrija}
% L for Left, you can also use R for Right or C for Center
\fancyfoot[R]{pagrindinė trigonometrinė tapatybė | sinusų teorema | kosinusių
      teorema | trikampio plotas | planimetrija}
% L for Left, you can also use R for Right or C for Center
\setlength{\headheight}{0.5pt} % Adjust the head height
\renewcommand{\headrulewidth}{0.4pt} % Line under the header
\renewcommand{\footrulewidth}{0.4pt} % Line above the footer
% Header | Footer 

\newcommand{\germanqq}[1]{{\selectlanguage{german}\glqq#1\grqq\selectlanguage{english}}}

\DeclareMathOperator{\tg}{tg}
\newcommand{\tgx}{\tg x}

\DeclareMathOperator{\arctg}{arctg}
\newcommand{\arctgx}{\arctg x}

\makeatletter
\newcommand*{\rom}[1]{\expandafter\@slowromancap\romannumeral #1@}
\makeatother

\title{Kontrolinis darbas nr. 2}
\author{Vilius Paliokas}
\date{2023/10/17}

\setlist{after=\vspace{\baselineskip}}

% Title spacing
\usepackage{titlesec}
\titlespacing*{\subsection}{0pt}{\baselineskip}{0.5\baselineskip}
% ------------------------ 

\begin{document}
\thispagestyle{fancy}
\subsection*{2 variantas}

\begin{enumerate}
      \item Užduočių sprendimo pavyzdys. Stačiojo trikampio dviejų statinių
            ilgiai yra 3 ir 4. Apskaičiuokite įžambinės ilgį.

            \begin{enumerate}

                  \item

                        \begin{adjustbox}{minipage={\linewidth}, valign=t}

                              \begin{wrapfigure}{t}{0.17\linewidth}

                                    \begin{resizebox}{0.17\textwidth}{!}{
                                                \begin{tikzpicture}
    \coordinate (a) at (0,0);
    \coordinate (b) at (4,0);
    \coordinate (c) at (4,5);

    \draw (a) -- (b) node[midway, below]{\large 3} -- (c) node[midway, right]{\large 4} -- (a); % Larger labels

    \draw (a) node[anchor=east, align=center] {\Large L};
    \draw (b) node[anchor=west, align=center] {\Large I};
    \draw (c) node[anchor=south] {\Large V};
\end{tikzpicture}
}
                                    \end{resizebox}
                                    \vspace{-2\baselineskip}

                              \end{wrapfigure}

                              \vspace*{0.15em}

                              Pirmiausia nusibrėžiamas trikampis ir suteikiamos
                              raidės viršunėms arba kraštinėmis naudojantis
                              jūsų vardo ar pavardės raidėmis.
                        \end{adjustbox}

                  \item \parbox{0.8\textwidth}{
                              Tada užrašoma naudojama teorema ar formulė (jeigu
                              naudojama
                              teorema, turi būti parašyta: \germanqq{pagal
                                    ...}) :
                        }

                        \begin{minipage}{0.5\textwidth}
                              Pagal Pitagoro teorema: $VL^{2}=LI^{2}+VI^{2}$.

                        \end{minipage}

                  \item \parbox{0.7\textwidth}{
                              Toliau galima išsireikšti kraštinę iš
                              raidinio reiškinio arba sustatyti turimas
                              reikšmes:
                        }

                        $$VL^{2}=3^{2}+4^{2} \Rightarrow=
                              VL^{2}=9+16 \Rightarrow VL=\sqrt{25}=5$$
                        $$\text{Ats.:} \; 25;$$

                  \item Už teisingą teoremos ar formulės parinkimą,
                        pritaikymą
                        ir užrašymą skiriamas 1 taškas.
            \end{enumerate}

      \item Apskaičiuokite dviejų pasirinktų kampų dydžius ($0,1^\circ$
            tikslumu), kai trikampio kraštinių ilgiai - 5, 10, 8 (3 taškai).

      \item Apskaičiuokite $\cos\beta$ ir $\tg \beta$, kai $sin \beta =
                  \frac{3}{11}$ ir $\beta \in$ \rom{2} ketvirčiui (3 taškai).

      \item Apskaičiuokite trikampio plotą, kai jo dvi kraštinės lygios
            $20\:cm.$ ir $30\sqrt{2}\:cm.$, o kampas tarp šių kraštinių lygus
            $150^\circ$ (2 taškai).

      \item Duotas trikampis, kurio viena kraštinė lygi 20, o kampai prie jos
            $45^\circ$ ir $30^\circ$. Suskaičiuokite nežinomo kampo dydį
            ($0,1^\circ$ tikslumu) ir
            nežinomų kraštinių ilgius dešimtųjų tikslumu (3 taškai).

\end{enumerate}

\begin{small}
      \begin{enumerate*}[label={(\arabic*)}]
            \item \textbf{Visur užrašykite atsakymus} ($Ats\ldots$);
            \item Jokio sukčiavimo. Negalima naudotis užrašais, vadovėliais,
            elektroniniais prietaisais;
            \item Jokio kalbėjimo;
            \item Rašyti aiškiai, nedviprasmiškai;
            \item Galima naudotis tik savo skaičiuotuvu ir formulių lapu;
      \end{enumerate*}
\end{small}


\subsection*{2 variantas}
  
\begin{enumerate}
      \item Užduočių sprendimo pavyzdys. Stačiojo trikampio dviejų statinių
            ilgiai yra 3 ir 4. Apskaičiuokite įžambinės ilgį.

            \begin{enumerate}

                  \item

                        \begin{adjustbox}{minipage={\linewidth}, valign=t}

                              \begin{wrapfigure}{t}{0.17\linewidth}

                                    \begin{resizebox}{0.17\textwidth}{!}{
                                                \begin{tikzpicture}
    \coordinate (a) at (0,0);
    \coordinate (b) at (4,0);
    \coordinate (c) at (4,5);

    \draw (a) -- (b) node[midway, below]{\large 3} -- (c) node[midway, right]{\large 4} -- (a); % Larger labels

    \draw (a) node[anchor=east, align=center] {\Large L};
    \draw (b) node[anchor=west, align=center] {\Large I};
    \draw (c) node[anchor=south] {\Large V};
\end{tikzpicture}
}
                                    \end{resizebox}
                                    \vspace{-2\baselineskip}

                              \end{wrapfigure}

                              \vspace*{0.15em}

                              Pirmiausia nusibrėžiamas trikampis ir suteikiamos
                              raidės viršunėms arba kraštinėmis naudojantis
                              jūsų vardo ar pavardės raidėmis.
                        \end{adjustbox}

                  \item \parbox{0.8\textwidth}{
                              Tada užrašoma naudojama teorema ar formulė (jeigu
                              naudojama
                              teorema, turi būti parašyta: \germanqq{pagal
                                    ...}) :
                        }

                        \begin{minipage}{0.5\textwidth}
                              Pagal Pitagoro teorema: $VL^{2}=LI^{2}+VI^{2}$.

                        \end{minipage}

                  \item \parbox{0.7\textwidth}{
                              Toliau galima išsireikšti kraštinę iš
                              raidinio reiškinio arba sustatyti turimas
                              reikšmes:
                        }

                        $$VL^{2}=3^{2}+4^{2} \Rightarrow=
                              VL^{2}=9+16 \Rightarrow VL=\sqrt{25}=5$$
                        $$\text{Ats.:} \; 25;$$

                  \item Už teisingą teoremos ar formulės parinkimą,
                        pritaikymą
                        ir užrašymą skiriamas 1 taškas.
            \end{enumerate}

      \item Apskaičiuokite dviejų pasirinktų kampų dydžius ($0,1^\circ$
            tikslumu), kai trikampio kraštinių ilgiai - 5, 10, 8 (3 taškai).

      \item Apskaičiuokite $\cos\beta$ ir $\tg \beta$, kai $sin \beta =
                  \frac{3}{11}$ ir $\beta \in$ \rom{2} ketvirčiui (3 taškai).

      \item Apskaičiuokite trikampio plotą, kai jo dvi kraštinės lygios
            $20\:cm.$ ir $30\sqrt{2}\:cm.$, o kampas tarp šių kraštinių lygus
            $150^\circ$ (2 taškai).

      \item Duotas trikampis, kurio viena kraštinė lygi 20, o kampai prie jos
            $45^\circ$ ir $30^\circ$. Suskaičiuokite nežinomo kampo dydį
            ($0,1^\circ$ tikslumu) ir
            nežinomų kraštinių ilgius dešimtųjų tikslumu (3 taškai).

\end{enumerate}

\begin{small}
      \begin{enumerate*}[label={(\arabic*)}]
            \item \textbf{Visur užrašykite atsakymus} ($Ats\ldots$);
            \item Jokio sukčiavimo. Negalima naudotis užrašais, vadovėliais,
            elektroniniais prietaisais;
            \item Jokio kalbėjimo;
            \item Rašyti aiškiai, nedviprasmiškai;
            \item Galima naudotis tik savo skaičiuotuvu ir formulių lapu;
      \end{enumerate*}
\end{small}


\end{document}