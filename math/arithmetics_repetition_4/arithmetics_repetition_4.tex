% !TEX root = ./equation_test_2.tex
\documentclass[a4paper]{article}
\usepackage[top=1.45cm, bottom=1cm, left=1cm, right=1cm]{geometry}
\mathchardef\period=\mathcode`.

\usepackage{parskip} % Package to tweak paragraph skipping
\usepackage{siunitx}

\usepackage[inline]{enumitem}
\usepackage{amsmath,amssymb}
\usepackage{tasks}
\usepackage{amsmath}
\usepackage{hyperref}
\usepackage[main=lithuanian, german, shorthands=off]{babel}
\usepackage{tgpagella}
\usepackage[L7x,T1]{fontenc}
\usepackage[utf8]{inputenc}
\usepackage{enumitem}
\usepackage{lipsum}
\usepackage{fancyhdr}

\usepackage{blindtext}
\usepackage{adjustbox}
\AfterEndEnvironment{wrapfigure}{\setlength{\intextsep}{0mm}}

\usepackage{icomma}

% Header | Footer 
\fancyhf{} % clear all header and footer fields
\fancyhead[R]{kurso kartojimas | reiškinių persitvarkymas | savarankiškas darbas}
% L for Left, you can also use R for Right or C for Center
\fancyfoot[R]{kurso kartojimas | reiškinių persitvarkymas | savarankiškas darbas}

% L for Left, you can also use R for Right or C for Center
\setlength{\headheight}{0.5pt} % Adjust the head height
\renewcommand{\headrulewidth}{0.4pt} % Line under the header
\renewcommand{\footrulewidth}{0.4pt} % Line above the footer
% Header | Footer 

\newcommand{\germanqq}[1]{{\selectlanguage{german}\glqq#1\grqq\selectlanguage{english}}}

\DeclareMathOperator{\tg}{tg}
\newcommand{\tgx}{\tg x}

\DeclareMathOperator{\arctg}{arctg}
\newcommand{\arctgx}{\arctg x}

\makeatletter
\newcommand*{\rom}[1]{\expandafter\@slowromancap\romannumeral #1@}
\makeatother

\title{Kontrolinis darbas - lygtys}
\author{Vilius Paliokas}
\date{2024/05/010}

\setlist{after=\vspace{\baselineskip}}

% Title spacing
\usepackage{titlesec}
\titlespacing*{\subsection}{0pt}{\baselineskip}{0.5\baselineskip}
% ------------------------ 

% Tasjks

\begin{document}
\thispagestyle{fancy}

\titlespacing*{\subsection}{0pt}{.75ex}{0.75ex}

\subsection*{4 variantas}

\textit{Visi uždaviniai verti 1 taško.}

\begin{enumerate}
      \item Suprastinkite reiškinius.
            \begin{tasks}[item-format={\normalfont}, after-item-skip=2mm](4)
                  \task $\frac{10a}{-22y}$;
                  \task $(2-7x)(7x+2)-8$;
                  \task $\frac{4x^2-9y^2}{2x+3y}$;
                  \task $\frac{2m-9}{21}:\frac{4m^2-81}{7m^2}$;
            \end{tasks}

      \item Subendravardiklinkite trupmenas ir atlikite veiksmus.
            \begin{tasks}[item-format={\normalfont}, after-item-skip=2mm](2)
                  \task $\frac{5}{2x-2}+\frac{3}{4x-4}$;
                  \task $\frac{a}{a+2}-\frac{a}{a-2}$;
            \end{tasks}

      \item Išskaidykite dauginamaisiais.
            \begin{tasks}[item-format={\normalfont}, after-item-skip=2mm](2)
                  \task $-2ax+4a$;
                  \task $25x^2-4$;
            \end{tasks}

      \item Išspręskite tiesinę lygtį.
            \begin{tasks}[item-format={\normalfont}, after-item-skip=2mm](2)
                  \task $\frac{3x}{2}=\frac{4x}{7}-14$;
                  \task $0,7(3-\frac{3}{4}x)=5-2(3,1x-0,5)$;
            \end{tasks}

      \item Išspręskite nelygybę $5x-3>\frac{x}{2}$.
      \item Raskite nelygybės $3-\frac{5(1-4x)}{3}\geq2(10x-4)$ didžiausią sveikąjį sprendinį.
\end{enumerate}

\vspace*{12mm}

\subsection*{4 variantas}

\textit{Visi uždaviniai verti 1 taško.}

\begin{enumerate}
      \item Suprastinkite reiškinius.
            \begin{tasks}[item-format={\normalfont}, after-item-skip=2mm](4)
                  \task $\frac{10a}{-22y}$;
                  \task $(2-7x)(7x+2)-8$;
                  \task $\frac{4x^2-9y^2}{2x+3y}$;
                  \task $\frac{2m-9}{21}:\frac{4m^2-81}{7m^2}$;
            \end{tasks}

      \item Subendravardiklinkite trupmenas ir atlikite veiksmus.
            \begin{tasks}[item-format={\normalfont}, after-item-skip=2mm](2)
                  \task $\frac{5}{2x-2}+\frac{3}{4x-4}$;
                  \task $\frac{a}{a+2}-\frac{a}{a-2}$;
            \end{tasks}

      \item Išskaidykite dauginamaisiais.
            \begin{tasks}[item-format={\normalfont}, after-item-skip=2mm](2)
                  \task $-2ax+4a$;
                  \task $25x^2-4$;
            \end{tasks}

      \item Išspręskite tiesinę lygtį.
            \begin{tasks}[item-format={\normalfont}, after-item-skip=2mm](2)
                  \task $\frac{3x}{2}=\frac{4x}{7}-14$;
                  \task $0,7(3-\frac{3}{4}x)=5-2(3,1x-0,5)$;
            \end{tasks}

      \item Išspręskite nelygybę $5x-3>\frac{x}{2}$.
      \item Raskite nelygybės $3-\frac{5(1-4x)}{3}\geq2(10x-4)$ didžiausią sveikąjį sprendinį.
\end{enumerate}

\vspace*{12mm}

\subsection*{4 variantas}

\textit{Visi uždaviniai verti 1 taško.}

\begin{enumerate}
      \item Suprastinkite reiškinius.
            \begin{tasks}[item-format={\normalfont}, after-item-skip=2mm](4)
                  \task $\frac{10a}{-22y}$;
                  \task $(2-7x)(7x+2)-8$;
                  \task $\frac{4x^2-9y^2}{2x+3y}$;
                  \task $\frac{2m-9}{21}:\frac{4m^2-81}{7m^2}$;
            \end{tasks}

      \item Subendravardiklinkite trupmenas ir atlikite veiksmus.
            \begin{tasks}[item-format={\normalfont}, after-item-skip=2mm](2)
                  \task $\frac{5}{2x-2}+\frac{3}{4x-4}$;
                  \task $\frac{a}{a+2}-\frac{a}{a-2}$;
            \end{tasks}

      \item Išskaidykite dauginamaisiais.
            \begin{tasks}[item-format={\normalfont}, after-item-skip=2mm](2)
                  \task $-2ax+4a$;
                  \task $25x^2-4$;
            \end{tasks}

      \item Išspręskite tiesinę lygtį.
            \begin{tasks}[item-format={\normalfont}, after-item-skip=2mm](2)
                  \task $\frac{3x}{2}=\frac{4x}{7}-14$;
                  \task $0,7(3-\frac{3}{4}x)=5-2(3,1x-0,5)$;
            \end{tasks}

      \item Išspręskite nelygybę $5x-3>\frac{x}{2}$.
      \item Raskite nelygybės $3-\frac{5(1-4x)}{3}\geq2(10x-4)$ didžiausią sveikąjį sprendinį.
\end{enumerate}

\end{document}