\documentclass[a4paper]{article}

\usepackage{fullpage} % Package to use full page
\usepackage{parskip} % Package to tweak paragraph skipping
\usepackage{tikz} % Package for drawing
\usepackage{tkz-euclide}
\usetikzlibrary{fit,positioning}
\usepackage{amsmath,amssymb}

\usepackage{amsmath}
\usepackage{hyperref}
\usepackage[main=lithuanian, german, shorthands=off]{babel}
\usepackage{tgpagella}
\usepackage[L7x,T1]{fontenc}
\usepackage[utf8]{inputenc}
\usepackage{enumitem}
\usepackage{booktabs} % For better looking tables
\usepackage{venndiagram}

\newcommand{\germanqq}[1]{{\selectlanguage{german}\glqq#1\grqq\selectlanguage{english}}}

\tikzset{
      venn box/.style={
                  draw=black, very thick,
                  rounded corners=10,
                  inner xsep=10pt, inner ysep=15pt, outer ysep=5pt
            },
      venn numbers/.style={
                  %    draw,
                  inner ysep=0pt,
                  align=center
            },
      venn title/.style={
                  fill=black, text=white
            }
}

\title{Savarankiško darbo refleksija}
\author{Vilius Paliokas}
\date{2023/09/29}

\begin{document}

\maketitle

\section{Lygtys}

\textbf{Lygtis}: matematinis teiginys, teigiantis dviejų reiškinių lygybę.

\textbf{Sprendinys}: reikšmė (arba reikšmių rinkinys), dėl kurios lygtis yra
teisinga, kai jos kintamasis (dažniausiai $x$) pakeičiamas ja (reikšme).

\subsection{Lygties sprendimas}

Pagrindiniai žingsniai:

\begin{enumerate}
      \item \textbf{Supaprastinimas}: suprastinamos abi lygties pusės (panašių
            narių jungimas, perkėlimai, skliaustų atskleidimai ir kt.);
      \item \textbf{Izoliuojamas kintamasis}: Naudojami aritmetiniai veiksmai
            ir atvirkštinės operacijos (jeigu lygybė, tai atimtis; jeigu
            kėlimas laipsniu, tai šaknies traukimas ir t.t.), kad kintamasis
            (dažniausiai
            $x$) būtų vienintelis
            kažkurioje tai lygties pusėje.
      \item \textbf{Atsakymo pasitikrinimas}: gavus sprendinį, įdedamas vietoje
            kintamojo ir patikrinama, kad abi pusės lygios.

\end{enumerate}

Pagrindiniai aspektai:
\begin{enumerate}
      \item \textbf{Atvirkštinės operacijos}:
            naudojamos operacijos, kurios atšaukia viena kitą (pvz., sudėjimas
            ir atėmimas, daugyba ir padalijimas).

            \begin{table}[h]
                  \centering
                  \begin{tabular}{cc}
                        \toprule
                        Operacija                                & Atvirkštinė
                        operacija
                        \\
                        \midrule
                        Sudėtis $(+a)$                           & Atimtis
                        $(-a)$
                        \\
                        Atimtis $(-a)$                           & Sudėtis
                        $(+a)$
                        \\
                        Daugyba $(\times a)$                     & Dalyba
                        $(\div a)$
                        \\
                        Dalyba $(\div a)$                        & Daugyba
                        $(\times
                              a)$
                        \\
                        Kėlimas kvadratu $(x^2)$                 & Kvadratinė
                        šaknis
                        $(\sqrt{x})$
                        \\
                        Kėlimas kubu $(x^3)$                     & Kubinė
                        šaknis
                        $(\sqrt[3]{x})$
                        \\
                        Kėlimas laipsniu $(x^a)$                 & Šaknis
                        \((\sqrt[a]{x})\)
                        \\
                        Logaritmas pagrindu  \(b\) $(\log_b{x})$ & Kėlimas, kai
                        pagrindas konstanta \(b\) $(b^x)$
                        \\
                        \bottomrule
                  \end{tabular}
                  \caption{Operacijos ir jų atvirkštinės operacijos}
                  \label{tab:inverse_operations}
            \end{table}

      \item \textbf{Panašieji nariai}: Atliekamos operacijos su panašiais
            nariais. Panašieji nariai - tai tie, kurie turi tą patį kintamąjį
            ir pakelti
            tuo pačiu laipsniu (daugiau žiūrėti nelygybių temoje).
      \item \textbf{Lygties balansas}: Kad ir ką darytumėte vienai lygties
            pusei, turite daryti su kita.
\end{enumerate}

Lygybėms galioja veiksmų eiliškumas - taisyklių rinkinys, kuris nurodo, kokius
veiksmus reikia atlikti pirmiausia, kad būtų tinkamai apskaičiuota matematinė
išraiška.
Žemiau pateikiama operacijų atlikimo tvarka:

\begin{enumerate}
      \item Skliaustai;
      \item Kėlimas laipsniu, šaknies traukimas, logaritmavimas;
      \item Daugyba, dalyba (iš kairės į dešinę);
      \item Atimtis, sudėtis (iš kairės į dešinę).
\end{enumerate}

\subsection{Kaip išspręsti $ ax^{2}+bx=0 $}

\subsubsection{Teorinis sprendimas}

Žingsniai:

\begin{enumerate}
      \item Turime nepilną kvadratinę lygtį.

            $ ax^{2}+ bx = 0; $
      \item Išskaidome dauginamaisiais - iškeliame $ x $ prieš skliaustus:

            $ x(ax + b) = 0 $
      \item Iškėlus prieš skliaustus, jau turime vieną sprendinį ($ x $), kitą
            dar reikia susirasti:

            $ ax+b = 0 $ $\;\;\;$ arba $\;\;\;$ $ x=0 $
      \item Susitvarkome lygtį taip, kad vienoje pusėje atsirastų nariai su $ x
            $, o kitoje tik skaičiai. Tai padarysime atėmę iš abiejų pusių
            skaičių $ b $:

            $ ax+b = 0 | - b $
      \item  $ ax+b-b = 0-b $
      \item Reikia pasidaryti, kad kintamasis $ x $ būtų plikas - be dauginio $
                  a
            $. Tai padarysime padalinę lygtį iš to dauginio $ a $:

            $ ax = -b | : a $
      \item $ \frac{ax}{a} = -\frac{b}{a} $
      \item $ x = -\frac{b}{a} $
\end{enumerate}
Po 9 žingsnio turime du lygties sprendinius $ x = -\frac{b}{a} $ ir $ x=0 $ (3
žingsnis).

\subsubsection{Pavyzdys \#1}

Turime $ 2x^{2}-4x = 0; $

Pagal formulę $ ax^{2}+ bx = 0 $:

\begin{itemize}
      \item $ a = 2; $
      \item $ b = -4 $.
\end{itemize}

\begin{figure}[!htbp]
      \begin{center}
            \begin{tikzpicture}
                  \foreach \x in {-3, -2,...,-1,1,2} \draw (\x,2pt) --++
                  (0,-4pt)
                  node [below] {\x};
                  \foreach \y in {-3,...,-1,1,2} \draw (2pt,\y) --++ (-4pt,0)
                  node
                  [left] {\y};
                  \draw[domain=-.5:2.5, color=blue] plot (\x, {2*
                              (\x)^2-4*(\x)})
                  node[above = .5cm, right, color=blue] {$f(x)=2x^2-4x$};
                  \draw [thick, ->] (-3,0) -- (3,0) node [above] {$x$};
                  \draw [thick, ->] (0,-3) -- (0,3) node [right] {$y$};
                  \node at (0,0) {\textbullet};
                  \node at (2,0) {\textbullet};
            \end{tikzpicture}
      \end{center}
      \caption{$f(x)=2x^2-4x$ grafikas su sprendiniais $2x^2-4x=0$
      }\label{fx=2x2-4x}
\end{figure}

Žingsniai:
\begin{enumerate}

      \item  Išskaidome dauginamaisiais - iškeliame $ x $ prieš skliaustus:

            $ x(2x - 4) = 0 $;
      \item  Iškėlus prieš skliaustus, jau turime vieną sprendinį ($ x $), kitą
            dar reikia susirasti:

            $ 2x-4 = 0 $ $\;\;\;$ arba $\;\;\;$ $ \boldsymbol{x=0} $;
      \item  Susitvarkome lygtį taip, kad vienoje pusėje atsirastų nariai su $
                  x
            $, o kitoje tik skaičiai. Tai padarysime pridėję abiem pusėms
            skaičių
            $ 4 $:

            $ 2x-4 = 0 | + 4 $;
      \item  $ 2-4+4 = 0+4 $;
      \item Reikia pasidaryti, kad kintamasis $ x $ būtų plikas - be dauginio $
                  2
            $. Tai padarysime padalinę lygtį iš to dauginio $ 2 $:

            $ 2x = 4 | : 2 $;
      \item $ \textcolor{blue}{\frac{2x}{2}} = \textcolor{red}{\frac{4}{2}} $;

            $ \textcolor{blue}{\frac{2x}{2}}=\textcolor{blue}{x} $;

            $ \textcolor{red}{-\frac{4}{2}}=\textcolor{red}{2} $;

      \item $ \textcolor{blue}{x} = \textcolor{red}{2} $;
\end{enumerate}

Po 7 žingsnio turime du lygties sprendinius $ x = 2 $ ir $ x=0 $ (2 žingsnis).

\subsubsection{Pavyzdys \#2}

Turime $ 2x^2 + 3x^2 - 5x = 4x $.

Šis reiškinys neatitinka $ ax^{2}+ bx = 0 $ formulės. Todėl pirmiausia reikia
bandyti susitvarkyti.

\begin{enumerate}
      \item Visus narius perkeliame į vieną pusę:

            $ 2x^2 + 3x^2 - 5x = 4x | - 4x; $

            $ 2x^2 + 3x^2 - 5x - 4x = 4x - 4x; $

            $ 2x^2 + 3x^2 - 5x - 4x = 0; $

      \item Sutraukiame panašius narius:

            $ \textcolor{blue}{2x^2} + \textcolor{blue}{3x^2} \textcolor{red}{-
                        5x} \textcolor{red}{- 4x} = 0; $

            $ \textcolor{blue}{5x^2} \textcolor{red}{- 9x} = 0; $

      \item Dabar jau reiškinys atitinka $ ax^{2}+ bx = 0 $ formulę. Galima
            išskaidyti dauginamaisiais - iškeliame prieš skliaustus $ x $:

            $ x(5x - 9) = 0 $;

      \item Iš čia gauname vieną sprendinį:

            $ 5x-9 = 0 $ $\;\;\;$ arba $\;\;\;$ $ \boldsymbol{x=0} $;

      \item Toliau sprendžiame pirmąją lygtį:

            $ 5x-9 = 0 | + 9 $;

            $ 5x-9+9 = 0+9 $;

            $ 5x = 9 $;

            $ 5x = 9|:5 $ arba $ 5x = 9|\cdot \frac{1}{5}$;

            $ \frac{5x}{5} = \frac{9}{5} $ arba $ 5x\cdot\frac{1}{5} = 9\cdot
                  \frac{1}{5}$;

            abiejais atvejais $ x = 1.8 $.

\end{enumerate}

Gauname, kad $ 2x^2 + 3x^2 - 5x = 4x $ lygties sprendiniai yra $x=0$ ir $ x =
      1.8 $ (galima dar rašyti $ x \in \{0, 1.8\} $).

\subsection{Kaip išspręsti $ ax^{2}+b=0 $}

\subsubsection{Teorinis sprendimas}

Žingsniai:
\begin{enumerate}

      \item  Išskiriame $ ax^{2} $ (paliekame kairėje pusėje be $ b $):

            $ ax^{2}+b=0 | -b $;

            $ ax^{2}+b-b=0-b $;

            $ ax^{2}=-b $;

      \item Kairėje pusėje reikia palikt $ x^2 $ - abi puses padaliname iš $ a
            $:

            $ ax^{2}=-b |:a $;

            $ \frac{ax^{2}}{a}=\frac{-b}{a}$;

            Kairėje pusėje galima suprastinti skaitiklyje ir vardiklyje
            esančius
            $ a $:

            $ x^{2}=\frac{-b}{a}$;

      \item Ištraukiame šaknį iš abiejų pusių:

            Visos kvadratinės lygtys turi du sprendinius (išskyrus, $ x^2=0 $),
            tai ištraukus šaknį:

            $ \sqrt{x^{2}}=\sqrt{\frac{-b}{a}}$;

            $ x=\sqrt{\frac{-b}{a}}$;

            ir

            $ \sqrt{x^{2}}=-\sqrt{\frac{-b}{a}}$;

            $ x=-\sqrt{\frac{-b}{a}}$;

\end{enumerate}

Šis sprendimas turi prasmę, kol $ x \neq 0 $ (dalijimas iš nulio neturi
reikšmės) ir $ \frac{-b}{a} \ge 0 $ (traukiant šaknį iš neigiamo skaičiaus
gaunamas kompleksinis skaičius - mokykloje to nesimokoma).

\subsubsection{Pavyzdys \#1}

Turime $ 2x^{2}+8=0 $. Ši atitinka $ ax^{2}+b=0 $ formą. Sprendžiame pagal
auksčiau duotą teorinį sprendimą:

\begin{enumerate}
      \item  Išskiriame $ ax^{2} $ (paliekame kairėje pusėje be $ b $):

            $ 2x^{2}+8=0 | -8 $;

            $ 2x^{2}+8-8=0-8 $;

            $ 2x^{2}=-8 $;

      \item Kairėje pusėje reikia palikt $ x^2 $ - abi puses padaliname iš $ 2
            $:

            $ 2x^{2}=-8 |:2 $;

            $ \frac{2x^{2}}{2}=\frac{-8}{2}$;

            Kairėje pusėje galima suprastinti skaitiklyje ir vardiklyje
            esančius
            $ a $, o dešinėje padalinti skaičius:

            $ x^{2}=-4$;

      \item Ištraukiame šaknį iš abiejų pusių:

            Visos kvadratinės lygtys turi du sprendinius (išskyrus, $ x^2=0 $),
            tai ištraukus šaknį:

            $ \sqrt{x^{2}}=\sqrt{-4}$;

            $ x=\sqrt{-4}$;

            ir

            $ \sqrt{x^{2}}=-\sqrt{-4}$;

            $ x=-\sqrt{-4}$;

\end{enumerate}

Kadangi dešinėje pusėje esantis skaičius yra mažiau už nulį (-4<0), tai ši
lygtis neturi realiųjų sprendinių.

\subsubsection{Pavyzdys \#2}

Turime $ 6x^{2}=3x^{2}+12 $. Ši lygtis neatitinka $ ax^{2}\pm b=0 $ formos.
Todėl pirmiausia reikia bandyti susitvarkyti.

\begin{enumerate}

      \item Persikeliame narius su $ x^2 $ į vieną pusę (pasirenkame kairę),
            tai
            galima padaryti atėmus abi puses iš $ 3x^{2} $:

            $ 6x^{2}=3x^{2}+12|-3x^{2} $;

            $ 6x^{2}-3x^{2}=3x^{2}+12-3x^{2} $;

            $ 3x^{2}=12 $;

      \item Dabar reiškinys atitinka $ ax^{2}-b=0 $, nes tai yra tas pats kas $
                  ax^{2}=b $. Toliau sprendžiame pagal taisykles, reikia $ x^2
            $
            palikti be
            skaičiaus esančio priekyje, tai padarysime padaline iš skaičiaus
            esančio prieš
            $ x^{2} $:

            $ 3x^{2}=12 |: 3 $;

            $ \frac{3x^{2}}{3}=\frac{12}{3}$;

            Kairėje pusėje galima padalinti 3 iš 3, o dešinėje 12 iš 3:

            $ x^{2}=4 $;

      \item Dabar galima iš abiejų pusių ištraukti šaknį:

            $ \sqrt{x^{2}}=\sqrt{4}$;

            $ x=2$;

            ir

            $ \sqrt{x^{2}}=-\sqrt{4}$;

            $ x=-2$;

\end{enumerate}

Lygtis $ 6x^{2}=3x^{2}+12 $ turi du sprendinius: $ x=2 $ ir $x=-2$. Sprendinius
visada galima pasitikrinti įdėjus atgal į lygtį.

\subsubsection{Pavyzdys \#3}

Nevisada išeis ištraukti šaknį \germanqq{gražiai} sprendžiant $ ax^{2}+b=0 $
lygtį. Pavyzdžiui turime paprastą lygtį $ x^{2}=40 $.

\begin{enumerate}

      \item Iš karto galime ištraukti šaknį iš abiejų pusių:

            $ \sqrt{x^{2}}=\sqrt{40}$;

            $ x=\sqrt{40}$;

            arba

            $ \sqrt{x^{2}}=-\sqrt{40}$;

            $ x=-\sqrt{40}$;

      \item Nors galėtume čia ir baigti spręsti, bet dar galime išskaidyti
            dauginamaisiai ir dalinai ištraukti šaknį:

            Šiam tikslui naudosime vieną iš šaknų savybių (žiūrėti bendrojo
            kurso
            brandos egzamino formulyną):
            $ \sqrt[n]{a\cdot b} = \sqrt[n]{a} \cdot \sqrt[n]{b}$.

            $ x=\sqrt{4\cdot10} = \sqrt{4}\cdot\sqrt{10} = 2\sqrt{10}$, nes $40
                  =
                  4 \cdot 10$;

            arba

            $ x=-\sqrt{4\cdot10} = -\sqrt{4}\cdot\sqrt{10} = -2\sqrt{10}$;

\end{enumerate}

Lygtis $ x^{2}=40 $ turi du sprendinius: $ \pm2\sqrt{10} $. Sprendinius visada
galima pasitikrinti įdėjus atgal į lygtį.

\section{Nelygybės}

Nelygybės išreiškia ryšį tarp dviejų dydžių, kurie nėra lygūs. Jose naudojami
kintamieji ir konstantos, o nelygybės simboliais parodoma, kad viena teiginio
pusė yra didesnė arba mažesnė už kitą.

Naudojami simboliai:

\begin{enumerate}
      \item Daugiau už ($>$), pavyzdžiui: $ x>3 $ (skaitoma $x$ daugiau už 3);
      \item Mažiau už ($<$), pavyzdžiui:	$ x<5 $, (skaitoma $x$ mažiau
            už 5);
      \item Daugiau už arba lygu ($ \geq $), pavyzdžiui: $ x \geq 4$ (skaitoma
            $x$ daugiau arba lygu už 4);

            Vietoje \germanqq{daugiau už arba lygu} galima vartoti \germanqq{ne
                  mažiau}.
      \item Mažiau už arba lygu ($ \leq $), pavyzdžiui: $ x \leq 6$ (skaitoma
            $x$
            mažiau arba lygu už 4).

            Vietoje \germanqq{mažiau už arba lygu} galima vartoti \germanqq{ne
                  daugiau}.

\end{enumerate}

\subsection{Pagrindiniai principai sprendžiant nelygybes}

Sprendžiant nelygybes, pritaikomi tokie pat principai, kaip ir lygtyse (ir
atvirkščiai).
Prisideda tik nelygybės apvertimas dauginant ar dalinant iš negiamo skaičiaus.

\begin{enumerate}
      \item Kad ir ką darytumėte vienai nelygybės pusei, turite padaryti kitai,
            kad išlaikytumėte nelygybę;

            $x+3>5$ tampa $x>2$ atėmus 3 iš abiejų pusių.

      \item Nelygybės apvertimas:

            \begin{enumerate}[label*=\arabic*.]

                  \item Kai padauginate arba padalijate abi nelygybės puses iš
                        neigiamo skaičiaus, nelygybės ženklas turi būti
                        apverstas.

                        $-2x>6$ tampa $x<-3x$ padalinus nelygybę iš $-2$
                        ($\boldsymbol{>} \rightarrow \boldsymbol{<}$).

                  \item Jeigu yra perkeliamas narys iš vienos nelygybės pusės į
                        kitą, tai reiktų laikyti tai, kaip to nario pridėjimą
                        ar
                        atėmimą iš abiejų
                        pusių. Būtina atkreipti dėmesį į ženklą:

                        $3\boldsymbol{\textcolor{red}{>}}x$ tampa
                        $x\boldsymbol{\textcolor{blue}{<}}3$ (atkreipkite
                        dėmesį į
                        ženklą), nes

                        $3\boldsymbol{\textcolor{red}{>}}x |-3 \Rightarrow$

                        $\Rightarrow 0\boldsymbol{\textcolor{red}{>}}x-3|-x
                              \Rightarrow$

                        $\Rightarrow
                              -x\boldsymbol{\textcolor{red}{>}}-3|\cdot-1
                              \Rightarrow$

                        $\Rightarrow x\boldsymbol{\textcolor{blue}{<}}3$.

            \end{enumerate}

      \item Panašių narių tvarkymas.

            Panašūs nariai turi tuos pačius kintamuosius ($x$, $y$, $z$
            ir kt.), kuriuo pakelti tais pačiais laipsniais.
            Iš esmės jie atrodo taip pat, išskyrus koeficientą prie jo
            (skaičius prieš kintamąjį).

            Pavyzdžiai:
            \begin{itemize}
                  \item $5x$ ir $3x$ yra panašūs nariai, nes abu turi kintamąjį
                        $x$
                        ir jie pakilti pirmuoju ($x^{1}=x$), nors ir
                        koeficientai
                        (5 ir 3) prie šių
                        kintamųjų skirtingi.
                  \item $7y^{2}$ ir $-y^2$ yra panašūs nariai, nes abu turi
                        kintamąjį $y$ ir jie pakilti antruoju laipsniu, nors ir
                        koeficientai (5 ir
                        -1) prie šių kintamųjų skirtingi.
                  \item $-4ab$ ir $5ab$ yra panašūs nariai, nes abu turi
                        kintamuosius $a$ ir $b$, bei jie pakilti pirmuoju
                        laipsniu.
            \end{itemize}

            Atvirkštiniai pavyzdžiai:
            \begin{itemize}
                  \item $3x$ ir $3y$ nėra panašūs nariai, nes abu turi
                        skirtingus kintamuosius
                        $x$ ir $y$.
                  \item $x^{2}$ ir $x$ nėra panašūs nariai, nes abu kintamieji
                        pakelti skirtingais laipsniais (2 ir 1);
                  \item $4xy$ ir $4xy^{2}$ nėra panašūs nariai, antrojo nario
                        kintamasis $y$ pakeltas kvadratu.
                        laipsniu.
            \end{itemize}

            \begin{enumerate}[label*=\arabic*.]

                  \item Konstantos ir kintamieji turi būti suprastinti, jeigu
                        tai įmanoma:

                        $2x+5>x+8$ tampa $x>3$ atėmus abiu puses iš $x$ ir $5$.

                  \item Panašūs nariai gali būti sudėti arba atimti:

                        $3x+2x>10$ tampa $5x>10$, o po to ir $x>2$.

            \end{enumerate}

\end{enumerate}

\subsection{Atsakymo pasitikrinimas}

Visada galima pasitikrinti nelygybės atsakymą. Pavyzdžiui turime nelygybę
$2x+3<11$:

\begin{enumerate}
      \item Atimame abi pusęs iš 3

            $2x+3-3<11-3$;

            $2x<8$;

      \item Padaliname abi pusęs iš 2

            $\frac{2x}{2}<\frac{8}{2}$;

            $x<4$;
\end{enumerate}

Radome, kad nelygybės sprendinys yra $x<4$ arba $x\in (-\infty;4)$. Galime
pasitikrinti šį sprendinį įstatydami skaičių mažesnį negu 4, pavyzdžiui 3.
Įstačius į pradinę nelygybę gauname, kad $2\cdot3+3<11$. Atlikus aritmetinius
veiksmus gauname, kad $9<11$, kas yra tiesa ir tai reiškia, kad sprendinys yra
teisingas.

\subsection{Kaip spręsti $ax-b<0$ nelygybę?}

\subsubsection{Teorinis sprendimas}

Žingsniai, kad išspręstume $ax-b<0$ nelygybę:

\begin{enumerate}
      \item Prie abiejų pusių pridedame $b$:

            $ax-b+b<0+b$;

            $ax<b$;

      \item Padaliname iš $a$, kad paliktumę kintąmjį $x$ be koeficiento
            (daugiklio):

            $\frac{ax}{a}<\frac{b}{a}$;

            $x<\frac{b}{a}$;

      \item Neužmirškite, jeigu skaičius $a$ yra neigiamas, reikia apversti
            nelygybės ženklą:

            $x>-\frac{b}{a}$;

\end{enumerate}

\subsubsection{Pavyzdys \#1}

Turime $3x-5 < 0$. Sprendimas:

\begin{enumerate}
      \item Prie abiejų pusių pridedame $5$:

            $3x-5+5<0+5$;

            $3x<5$;

      \item Padaliname abi puses iš $3$, kad paliktumę kintąmjį $x$ be
            koeficiento (daugiklio):

            $\frac{3x}{3}<\frac{5}{3}$;

            $x<\frac{5}{3}$;

\end{enumerate}

Nelygybės sprendinys: $x\in (-\infty;\frac{5}{3})$.

\subsubsection{Pavyzdys \#2}

Turime $-3x+2\geq 5x-8$. Sprendimas:

\begin{enumerate}
      \item Visus narius su $x$ kintamuoju perkeliame į vieną pusę. Aš
            pasirenku kelti į dešinę:

            $-3x+2\geq 5x-8 | -5x$;

            $-3x+2-5x \geq 5x-8-5x$;

            $-8x+2 \geq -8$.

      \item Visus skaičius be kintamųjų (konstantas) perkeliame į kitą pusę.
            Šiuo atveju į kairę:

            $-8x+2 \geq -8|-2$;

            $-8x+2-2 \geq -8-2$;

            $-8x \geq -10$;

      \item Panaikiname skaičių prie $x$ padalindami abi puses iš jo:

            $-8x \geq -10|:(-8)$;

            $\frac{-8x}{-8} \leq \frac{-10}{-8}$ (atkreipkite dėmesį į ženklo
            pasikeitimą);

            $x \leq \frac{5}{4} $;

\end{enumerate}

Nelygybės sprendinys: $x\in (\infty;\frac{5}{4}]$.

\section{Aibės}

Aibė yra skirtingų elementų rinkinys. Jeigu elementas $a$ yra aibės $A$
elementas, tai rašoma, kad $a \in A$. Jeigu elementas $b$ nėra aibės $A$
elementas, tai rašoma, kad $b \notin A$. Aibės žymimos didžiąją raide, o jos
elementai mažosiomis. Matematikos šaka nagrinėjanti aibes
vadinama aibių teorija.

Aibės pavyzdžiai:

\begin{itemize}
      \item mokyklos mokinių aibė;
      \item saulės sistemos planetų aibė;
      \item visų natūraliųjų skaičių aibė;
      \item lygties sprendinių aibė;
      \item ...
\end{itemize}

\subsection{Būdai užrašyti aibę}

Pagal elementų skaičių, yra dviejų tipų aibės: baigtinės ir begalinės.
Baigtines aibes galima lengvai išrašyti. Tokiam būdui naudojami figūriniai
skliaustai $\{...\}$. Pavyzdžiui:

$$ A=\{1;5;6;7;8;10;30\}; \quad B=\{c;b;e\}; $$

Dvi aibės yra vienodos, jeigu šių elementai nesiskiria, nors ir skiriasi jų
išdėstymo tvarka. Pavyzdžiui, $\{1,2,3\}=\{2,3,1\}$. Bet matematikoje sutarta,
jeigu aibės elementai yra skaičiai, tai jie užrašomi didėjimo tvarka.

Taip pat baigtines ir begalines aibes galima užrašyti tam tikromis taisyklėmis:

\begin{itemize}
      \item taisyklėmis.

            $ B=\{x | x \text{ yra pirminis skaičius mažesnis už 10}\} $ - toks
            užrašymas reikštų, kad aibę $ B $ sudaro pirminiai skaičiai mažesni
            už 10. Tokią aibę dar būtų galima užrašyti šitaip $ B=\{2;3;5;7\}
            $.

      \item žodiniu apibūdinimu.

            Tegul, aibė $ C $ yra sudaryta iš visų sveikųjų skaičių mažesnių už
            100.

      \item intervalu.

            $ D = (2;5)$ - tokia aibė yra sudaryta iš visų realių skaičių nuo 2
            neįskaitant iki 5 neįskaitant.

            $ E = [2;5]$ - tokia aibė yra sudaryta iš visų realių skaičių nuo 2
            įskaitant iki 5 įskaitant.
      \item Veno diagramomis.

            Nors tai nėra tekstinis aibės apibrėžimo būdas, Veno diagrama
            vizualiai vaizduoja aibes ir jų ryšius (įprastai apskritimais).

            \begin{venndiagram3sets}[labelOnlyA={1}, labelOnlyB={2},
                        labelOnlyC={}, labelOnlyAB= {}, labelOnlyAC={},
                        labelOnlyBC={9}, labelABC={3},
                        vgap=.75cm]

            \end{venndiagram3sets}
\end{itemize}

\subsection{Realiųjų skaičių aibė}

Matematikoje, yra skaičių rinkiniai, kurie naudojami taip dažnai, kad jie turi
specialius pavadinimus ir simbolius:

\begin{enumerate}
      \item Natūralūs ($\mathbb{N}$);
      \item Sveikieji ($\mathbb{Z}$);
      \item Racionalieji ($\mathbb{Q}$);
      \item Iracionalieji ($\mathbb{I}$);
      \item Realūs ($\mathbb{R}$);
      \item ir kt.
\end{enumerate}

\begin{figure}[!htbp]
      \centering
      \begin{tikzpicture}[node distance=12pt]
            \node[venn box] (N) {%
                  $0 \quad 1 \quad 12 \quad 350 \quad 48 \quad 5 \quad 16$
            };

            \node[venn numbers, below=of N] (Z-N) {
                  $-1 \quad {-5} \quad {-10} \quad {-39}$
            };
            \node[venn box, fit=(Z-N) (N)] (Z) {};

            \node[venn numbers, below=of Z, align=center] (Q-Z) {
            $25{,}3401401401\dots \quad 48{,}259$ \\[5pt]
            $-0{,}101232323\dots \quad \dfrac52 \quad {-\dfrac73}$
            };
            \node[venn box, fit=(Q-Z) (Z)] (Q) {};

            \node[venn numbers, below=of Q, inner sep=8pt, align=center, ] (Q')
            {
            $6{,}1010010001\dots \quad {-0{,}1234567\dots}$ \\[5pt]
            $\sqrt{2} \quad \pi$
            };

            \node [venn box, inner sep=8pt, fit=(Q')] (Q') {};

            \node [venn box, fit=(Q') (Q)] (R) {};

            \tikzset{every node/.style=venn title}
            \foreach \i/\j/\k in {N/Natūralūs/0, Z/Sveikieji/10pt,
                        Q/Racionalūs/10pt,
                        Q'/Iracionalūs/0pt, R/Realūs/15pt} {
                        \draw node[anchor=north west] at ([shift={(10pt,
                                          3pt)}]\i.north west) {\j}
                        node[rounded corners] at ([yshift=-\k]\i.east)
                        {$\mathbb{\i}$};
                  }
      \end{tikzpicture}
      \label{fig:real_number_set}
      \caption{Realių skaičių aibė ir jos poaibiai}
\end{figure}

Mokyliniame kurse yra mokomi tik 5 pagrindinės skaičių aibės. Bet reiktų
žinoti, kad jų yra ir daugiau, pavyzdžiui menamasis vienetas ir kompleksniai
skaičiai. Toliau apibūdinama realiųjų skaičių aibė ir jos poaibiai.

\subsubsection{Natūralūs skaičiai}

\begin{itemize}
      \item \textbf{Apibrėžimas}: skaičiai naudojami skaičiuoti, nuo 1 iki
            begalybės (kartais įtraukiamas ir 0).
      \item \textbf{Simbolis}: $\mathbb{N}$.
      \item \textbf{Pavyzdys}: $\mathbb{N}=\{1;2;3;4;5;6;7;8;9;\cdots\}$.
\end{itemize}

\subsubsection{Sveikieji skaičiai}

\begin{itemize}
      \item \textbf{Apibrėžimas}: 0, visi natūralieji skaičiai ir jiems
            atvirkštiniai skaičiai (natūralieji skaičiai su minuso ženklu).
      \item \textbf{Simbolis}: $\mathbb{Z}$.
      \item \textbf{Pavyzdys}:

            $\mathbb{Z}=\{\cdots;-9;-8;-7;-6;-5;-4;-3;-2;-1;0;1;2;3;4;5;6;7;8;9;\cdots\}$.
\end{itemize}

\subsubsection{Racionalieji skaičiai}

\begin{itemize}
      \item \textbf{Apibrėžimas}: Skaičiai, kurie gali būti užrašyti trupmena
            $\frac{a}{b}$, kur $a$ ir $b$ sveikieji skaičiai, o $b \neq 0$.
      \item \textbf{Simbolis}: $\mathbb{Q}$.
      \item \textbf{Pavyzdys}: $\mathbb{Q}=\{\cdots;-\frac{8}{6}, -\frac{1}{1},
                  \frac{0}{1}, \frac{1}{2}, \frac{22}{7}\cdots\}$.

            Ankščiau aprašytą sveikųjų skaičių aibę $\mathbb{Z}$, taip pat
            galima
            išreikšti per racionaliųjų skaičių aibę: sveikieji skaičiai yra
            tie,
            racionalieji skaičiai $\frac{a}{b}$, kurių vardiklis $b$ yra lygus
            $1$.
\end{itemize}

\subsubsection{Iracionalieji skaičiai}

\begin{itemize}
      \item \textbf{Apibrėžimas}: Skaičiai, kurių negalima išreikšti trupmena
            $\frac{a}{b}$, kur $a$ ir $b$ sveikieji skaičiai. Šių skaičių
            dešimtainė dalis yra nesikartojanti ir nesibaigianti.
      \item \textbf{Simbolis}: $\mathbb{Q'}$ arba $\mathbb{R} \setminus
                  \mathbb{Q}$ (realiųjų skaičių ir racionaliųjų skaičių aibės
            skirtumas).
      \item \textbf{Pavyzdys}: $\sqrt{2}$, $\pi$, $e$.
\end{itemize}

\subsubsection{Realių skaičių aibės ir poaibių hierachija}
Visos aukščiau nurodytos aibės yra kažkokios tai kitos aibės poaibis. Šį ryšį
galima pamatyti veno diagramose \ref{fig:real_number_set} paveiksle.

\begin{itemize}
      \item $\mathbb{N}$ yra sveikųjų skaičių aibės $\mathbb{Z}$ poaibis.
      \item $\mathbb{Z}$ yra racionaliųjų skaičių aibės $\mathbb{Q}$ poaibis.
      \item $\mathbb{Q}$ yra realiųjų skaičių aibės $\mathbb{R}$ poaibis.
      \item Iracionalieji skaičiai taip pat yra realiųjų skaičių aibės $\mathbb{R}$ poaibis.
\end{itemize}

Šiuos ryšius galima taip pat pavaizduoti su simboliais:
\[ \mathbb{N} \subset \mathbb{Z} \subset \mathbb{Q} \subset \mathbb{R}; \]


\bibliographystyle{plain}
\bibliography{bibliography.bib}
\end{document}