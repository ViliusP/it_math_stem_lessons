\documentclass[a4paper]{article}

\usepackage{fullpage} % Package to use full page
\usepackage{parskip} % Package to tweak paragraph skipping
\usepackage{tikz} % Package for drawing
\usepackage{tkz-euclide}
\usepackage{amsmath}
\usepackage{hyperref}
\usepackage[main=lithuanian, german]{babel}
\usepackage{tgpagella}
\usepackage[L7x,T1]{fontenc}
\usepackage[utf8]{inputenc}
\usepackage{enumitem}

\newcommand{\germanqq}[1]{{\selectlanguage{german}\glqq#1\grqq\selectlanguage{english}}}

\title{Savarankiško darbo refleksija}
\author{Vilius Paliokas}
\date{2023/09/29}

\begin{document}

\maketitle

\section{Lygtys}

Differentiation is a concept of Mathematics studied in Calculus. There is an
ongoing discussion as to who was the first to define differentiation: Leibniz
or Newton \cite{bardi2006calculus}.

Differentiation allows for the calculation of the slope of the tangent of a
curve at any given point as shown in Figure \ref{exampleplot}.

\begin{figure}[!htbp]
      \begin{center}
            \begin{tikzpicture}
                  \draw[domain=-2:2, color=blue] plot (\x, {1 - (\x)^2})
                  node[above =
                              .5cm, right, color=blue] {$f(x)=1-x^2$};
                  \draw[domain=-2:2, color=red] plot(\x,-1 * \x + 1.25)
                  node[above =
                              .5cm, right, color=red] {Tangent at $x=.5$};
                  \draw [thick, ->] (-3,0) -- (3,0) node [above] {$x$};
                  \draw [thick, ->] (0,-3) -- (0,3) node [right] {$y$};
                  \node at (.5,.75) {\textbullet};
            \end{tikzpicture}
      \end{center}
      \caption{The plot of $f(x)=1-x^2$ with a tangent at
            $x=.5$.}\label{exampleplot}
\end{figure}

\subsection{Kaip išspręsti $ ax^{2}+bx=0 $}

\subsubsection{Teorinis sprendimas}

Žingsniai:

\begin{enumerate}
      \item Turime nepilną kvadratinę lygtį.

            $ ax^{2}+ bx = 0; $
      \item Išskaidome dauginamaisiais - iškeliame $ x $ prieš skliaustus:

            $ x(ax + b) = 0 $
      \item Iškėlus prieš skliaustus, jau turime vieną sprendinį ($ x $), kitą
            dar reikia susirasti:

            $ ax+b = 0 $ $\;\;\;$ arba $\;\;\;$ $ x=0 $
      \item Susitvarkome lygtį taip, kad vienoje pusėje atsirastų nariai su $ x
            $, o kitoje tik skaičiai. Tai padarysime atėmę iš abiejų pusių
            skaičių $ b $:

            $ ax+b = 0 | - b $
      \item  $ ax+b-b = 0-b $
      \item Reikia pasidaryti, kad kintamasis $ x $ būtų plikas - be dauginio $
                  a
            $. Tai padarysime padalinę lygtį iš to dauginio $ a $:

            $ ax = -b | : a $
      \item $ \frac{ax}{a} = -\frac{b}{a} $
      \item $ x = -\frac{b}{a} $
\end{enumerate}
Po 9 žingsnio turime du lygties sprendinius $ x = -\frac{b}{a} $ ir $ x=0 $ (3
žingsnis).

\subsubsection{Pavyzdys \#1}

Turime $ 2x^{2}-4x = 0; $

Pagal formulę $ ax^{2}+ bx = 0 $:

\begin{itemize}
      \item $ a = 2; $
      \item $ b = -4 $.
\end{itemize}

\begin{figure}[!htbp]
      \begin{center}
            \begin{tikzpicture}
                  \foreach \x in {-3, -2,...,-1,1,2} \draw (\x,2pt) --++
                  (0,-4pt)
                  node [below] {\x};
                  \foreach \y in {-3,...,-1,1,2} \draw (2pt,\y) --++ (-4pt,0)
                  node
                  [left] {\y};
                  \draw[domain=-.5:2.5, color=blue] plot (\x, {2*
                              (\x)^2-4*(\x)})
                  node[above = .5cm, right, color=blue] {$f(x)=2x^2-4x$};
                  \draw [thick, ->] (-3,0) -- (3,0) node [above] {$x$};
                  \draw [thick, ->] (0,-3) -- (0,3) node [right] {$y$};
                  \node at (0,0) {\textbullet};
                  \node at (2,0) {\textbullet};
            \end{tikzpicture}
      \end{center}
      \caption{$f(x)=2x^2-4x$ grafikas su sprendiniais $2x^2-4x=0$
      }\label{fx=2x2-4x}
\end{figure}

Žingsniai:
\begin{enumerate}

      \item  Išskaidome dauginamaisiais - iškeliame $ x $ prieš skliaustus:

            $ x(2x - 4) = 0 $;
      \item  Iškėlus prieš skliaustus, jau turime vieną sprendinį ($ x $), kitą
            dar reikia susirasti:

            $ 2x-4 = 0 $ $\;\;\;$ arba $\;\;\;$ $ \boldsymbol{x=0} $;
      \item  Susitvarkome lygtį taip, kad vienoje pusėje atsirastų nariai su $
                  x
            $, o kitoje tik skaičiai. Tai padarysime pridėję abiem pusėms
            skaičių
            $ 4 $:

            $ 2x-4 = 0 | + 4 $;
      \item  $ 2-4+4 = 0+4 $;
      \item Reikia pasidaryti, kad kintamasis $ x $ būtų plikas - be dauginio $
                  2
            $. Tai padarysime padalinę lygtį iš to dauginio $ 2 $:

            $ 2x = 4 | : 2 $;
      \item $ \textcolor{blue}{\frac{2x}{2}} = \textcolor{red}{\frac{4}{2}} $;

            $ \textcolor{blue}{\frac{2x}{2}}=\textcolor{blue}{x} $;

            $ \textcolor{red}{-\frac{4}{2}}=\textcolor{red}{2} $;

      \item $ \textcolor{blue}{x} = \textcolor{red}{2} $;
\end{enumerate}

Po 7 žingsnio turime du lygties sprendinius $ x = 2 $ ir $ x=0 $ (2 žingsnis).

\subsubsection{Pavyzdys \#2}

Turime $ 2x^2 + 3x^2 - 5x = 4x $.

Šis reiškinys neatitinka $ ax^{2}+ bx = 0 $ formulės. Todėl pirmiausia reikia
bandyti susitvarkyti.

\begin{enumerate}
      \item Visus narius perkeliame į vieną pusę:

            $ 2x^2 + 3x^2 - 5x = 4x | - 4x; $

            $ 2x^2 + 3x^2 - 5x - 4x = 4x - 4x; $

            $ 2x^2 + 3x^2 - 5x - 4x = 0; $

      \item Sutraukiame panašius narius:

            $ \textcolor{blue}{2x^2} + \textcolor{blue}{3x^2} \textcolor{red}{-
                        5x} \textcolor{red}{- 4x} = 0; $

            $ \textcolor{blue}{5x^2} \textcolor{red}{- 9x} = 0; $

      \item Dabar jau reiškinys atitinka $ ax^{2}+ bx = 0 $ formulę. Galima
            išskaidyti dauginamaisiais - iškeliame prieš skliaustus $ x $:

            $ x(5x - 9) = 0 $;

      \item Iš čia gauname vieną sprendinį:

            $ 5x-9 = 0 $ $\;\;\;$ arba $\;\;\;$ $ \boldsymbol{x=0} $;

      \item Toliau sprendžiame pirmąją lygtį:

            $ 5x-9 = 0 | + 9 $;

            $ 5x-9+9 = 0+9 $;

            $ 5x = 9 $;

            $ 5x = 9|:5 $ arba $ 5x = 9|\cdot \frac{1}{5}$;

            $ \frac{5x}{5} = \frac{9}{5} $ arba $ 5x\cdot\frac{1}{5} = 9\cdot
                  \frac{1}{5}$;

            abiejais atvejais $ x = 1.8 $.

\end{enumerate}

Gauname, kad $ 2x^2 + 3x^2 - 5x = 4x $ lygties sprendiniai yra $x=0$ ir $ x =
      1.8 $ (galima dar rašyti $ x \in \{0, 1.8\} $).

\subsection{Kaip išspręsti $ ax^{2}+b=0 $}

\subsubsection{Teorinis sprendimas}

Žingsniai:
\begin{enumerate}

      \item  Išskiriame $ ax^{2} $ (paliekame kairėje pusėje be $ b $):

            $ ax^{2}+b=0 | -b $;

            $ ax^{2}+b-b=0-b $;

            $ ax^{2}=-b $;

      \item Kairėje pusėje reikia palikt $ x^2 $ - abi puses padaliname iš $ a
            $:

            $ ax^{2}=-b |:a $;

            $ \frac{ax^{2}}{a}=\frac{-b}{a}$;

            Kairėje pusėje galima suprastinti skaitiklyje ir vardiklyje
            esančius
            $ a $:

            $ x^{2}=\frac{-b}{a}$;

      \item Ištraukiame šaknį iš abiejų pusių:

            Visos kvadratinės lygtys turi du sprendinius (išskyrus, $ x^2=0 $),
            tai ištraukus šaknį:

            $ \sqrt{x^{2}}=\sqrt{\frac{-b}{a}}$;

            $ x=\sqrt{\frac{-b}{a}}$;

            ir

            $ \sqrt{x^{2}}=-\sqrt{\frac{-b}{a}}$;

            $ x=-\sqrt{\frac{-b}{a}}$;

\end{enumerate}

Šis sprendimas turi prasmę, kol $ x \neq 0 $ (dalijimas iš nulio neturi
reikšmės) ir $ \frac{-b}{a} \ge 0 $ (traukiant šaknį iš neigiamo skaičiaus
gaunamas kompleksinis skaičius - mokykloje to nesimokoma).

\subsubsection{Pavyzdys \#1}

Turime $ 2x^{2}+8=0 $. Ši atitinka $ ax^{2}+b=0 $ formą. Sprendžiame pagal
auksčiau duotą teorinį sprendimą:

\begin{enumerate}
      \item  Išskiriame $ ax^{2} $ (paliekame kairėje pusėje be $ b $):

            $ 2x^{2}+8=0 | -8 $;

            $ 2x^{2}+8-8=0-8 $;

            $ 2x^{2}=-8 $;

      \item Kairėje pusėje reikia palikt $ x^2 $ - abi puses padaliname iš $ 2
            $:

            $ 2x^{2}=-8 |:2 $;

            $ \frac{2x^{2}}{2}=\frac{-8}{2}$;

            Kairėje pusėje galima suprastinti skaitiklyje ir vardiklyje
            esančius
            $ a $, o dešinėje padalinti skaičius:

            $ x^{2}=-4$;

      \item Ištraukiame šaknį iš abiejų pusių:

            Visos kvadratinės lygtys turi du sprendinius (išskyrus, $ x^2=0 $),
            tai ištraukus šaknį:

            $ \sqrt{x^{2}}=\sqrt{-4}$;

            $ x=\sqrt{-4}$;

            ir

            $ \sqrt{x^{2}}=-\sqrt{-4}$;

            $ x=-\sqrt{-4}$;

\end{enumerate}

Kadangi dešinėje pusėje esantis skaičius yra mažiau už nulį (-4<0), tai ši
lygtis neturi realiųjų sprendinių.

\subsubsection{Pavyzdys \#2}

Turime $ 6x^{2}=3x^{2}+12 $. Ši lygtis neatitinka $ ax^{2}\pm b=0 $ formos.
Todėl pirmiausia reikia bandyti susitvarkyti.

\begin{enumerate}

      \item Persikeliame narius su $ x^2 $ į vieną pusę (pasirenkame kairę),
            tai
            galima padaryti atėmus abi puses iš $ 3x^{2} $:

            $ 6x^{2}=3x^{2}+12|-3x^{2} $;

            $ 6x^{2}-3x^{2}=3x^{2}+12-3x^{2} $;

            $ 3x^{2}=12 $;

      \item Dabar reiškinys atitinka $ ax^{2}-b=0 $, nes tai yra tas pats kas $
                  ax^{2}=b $. Toliau sprendžiame pagal taisykles, reikia $ x^2
            $
            palikti be
            skaičiaus esančio priekyje, tai padarysime padaline iš skaičiaus
            esančio prieš
            $ x^{2} $:

            $ 3x^{2}=12 |: 3 $;

            $ \frac{3x^{2}}{3}=\frac{12}{3}$;

            Kairėje pusėje galima padalinti 3 iš 3, o dešinėje 12 iš 3:

            $ x^{2}=4 $;

      \item Dabar galima iš abiejų pusių ištraukti šaknį:

            $ \sqrt{x^{2}}=\sqrt{4}$;

            $ x=2$;

            ir

            $ \sqrt{x^{2}}=-\sqrt{4}$;

            $ x=-2$;

\end{enumerate}

Lygtis $ 6x^{2}=3x^{2}+12 $ turi du sprendinius: $ x=2 $ ir $x=-2$. Sprendinius
visada galima pasitikrinti įdėjus atgal į lygtį.

\subsubsection{Pavyzdys \#3}

Nevisada išeis ištraukti šaknį \germanqq{gražiai} sprendžiant $ ax^{2}+b=0 $
lygtį. Pavyzdžiui turime paprastą lygtį $ x^{2}=40 $.

\begin{enumerate}

      \item Iš karto galime ištraukti šaknį iš abiejų pusių:

            $ \sqrt{x^{2}}=\sqrt{40}$;

            $ x=\sqrt{40}$;

            arba

            $ \sqrt{x^{2}}=-\sqrt{40}$;

            $ x=-\sqrt{40}$;

      \item Nors galėtume čia ir baigti spręsti, bet dar galime išskaidyti
            dauginamaisiai ir dalinai ištraukti šaknį:

            Šiam tikslui naudosime vieną iš šaknų savybių (žiūrėti bendrojo
            kurso
            brandos egzamino formulyną):
            $ \sqrt[n]{a\cdot b} = \sqrt[n]{a} \cdot \sqrt[n]{b}$.

            $ x=\sqrt{4\cdot10} = \sqrt{4}\cdot\sqrt{10} = 2\sqrt{10}$, nes $40
                  =
                  4 \cdot 10$;

            arba

            $ x=-\sqrt{4\cdot10} = -\sqrt{4}\cdot\sqrt{10} = -2\sqrt{10}$;

\end{enumerate}

Lygtis $ x^{2}=40 $ turi du sprendinius: $ \pm2\sqrt{10} $. Sprendinius visada
galima pasitikrinti įdėjus atgal į lygtį.

\section{Nelygybės}

Nelygybės išreiškia ryšį tarp dviejų dydžių, kurie nėra lygūs. Jose naudojami
kintamieji ir konstantos, o nelygybės simboliais parodoma, kad viena teiginio
pusė yra didesnė arba mažesnė už kitą.

Naudojami simboliai:

\begin{enumerate}
      \item Daugiau už ($>$), pavyzdžiui: $ x>3 $ (skaitoma $x$ daugiau už 3);
      \item Mažiau už ($<$), pavyzdžiui:	$ x<5 $, (skaitoma $x$ mažiau
            už 5);
      \item Daugiau už arba lygu ($ \geq $), pavyzdžiui: $ x \geq 4$ (skaitoma
            $x$ daugiau arba lygu už 4);

            Vietoje \germanqq{daugiau už arba lygu} galima vartoti \germanqq{ne
                  mažiau}.
      \item Mažiau už arba lygu ($ \leq $), pavyzdžiui: $ x \leq 6$ (skaitoma
            $x$
            mažiau arba lygu už 4).

            Vietoje \germanqq{mažiau už arba lygu} galima vartoti \germanqq{ne
                  daugiau}.

\end{enumerate}

Pagrindiniai principai sprendžiant nelygybes:

\begin{enumerate}
      \item Kad ir ką darytumėte vienai nelygybės pusei, turite padaryti kitai,
            kad išlaikytumėte nelygybę;

            $x+3>5$ tampa $x>2$ atėmus 3 iš abiejų pusių.

      \item Nelygybės apvertimas:

            \begin{enumerate}[label*=\arabic*.]

                  \item Kai padauginate arba padalijate abi nelygybės puses iš
                        neigiamo skaičiaus, nelygybės ženklas turi būti
                        apverstas.

                        $-2x>6$ tampa $x<-3x$ padalinus nelygybę iš $-2$
                        ($\boldsymbol{>} \rightarrow \boldsymbol{<}$).

                  \item Jeigu yra perkeliamas narys iš vienos nelygybės pusės į
                        kitą, tai reiktų laikyti tai, kaip to nario pridėjimą
                        ar
                        atėmimą iš abiejų
                        pusių. Būtina atkreipti dėmesį į ženklą:

                        $3\boldsymbol{\textcolor{red}{>}}x$ tampa
                        $x\boldsymbol{\textcolor{blue}{<}}3$ (atkreipkite
                        dėmesį į
                        ženklą), nes

                        $3\boldsymbol{\textcolor{red}{>}}x |-3 \Rightarrow$

                        $\Rightarrow 0\boldsymbol{\textcolor{red}{>}}x-3|-x
                              \Rightarrow$

                        $\Rightarrow
                              -x\boldsymbol{\textcolor{red}{>}}-3|\cdot-1
                              \Rightarrow$

                        $\Rightarrow x\boldsymbol{\textcolor{blue}{<}}3$.

            \end{enumerate}

      \item Panašių narių tvarkymas.

            Panašūs nariai turi tuos pačius kintamuosius ($x$, $y$, $z$,
            skaičius
            ir kt.), kuriuo pakelti tais pačiais laipsniais.
            Iš esmės jie atrodo taip pat, išskyrus koeficientą prie jo
            (skaičius
            prieš kintamąjį).

            Pavyzdžiai:
            \begin{itemize}
                  \item $5x$ ir $3x$ yra panašūs nariai, nes abu turi kintamąjį
                        $x$
                        ir jie pakilti pirmuoju ($x^{1}=x$), nors ir
                        koeficientai
                        (5 ir 3) prie šių
                        kintamųjų skirtingi.
                  \item $7y^{2}$ ir $-y^2$ yra panašūs nariai, nes abu turi
                        kintamąjį $y$ ir jie pakilti antruoju laipsniu, nors ir
                        koeficientai (5 ir
                        -1) prie šių kintamųjų skirtingi.
                  \item $-4ab$ ir $5ab$ yra panašūs nariai, nes abu turi
                        kintamuosius $a$ ir $b$, bei jie pakilti pirmuoju
                        laipsniu.
            \end{itemize}

            Atvirkštiniai pavyzdžiai:
            \begin{itemize}
                  \item $3x$ ir $3y$ nėra panašūs nariai, nes abu turi
                        skirtingus kintamuosius
                        $x$ ir $y$.
                  \item $x^{2}$ ir $x$ nėra panašūs nariai, nes abu kintamieji pakelti skirtingais laipsniais (2 ir 1);
                  \item $4xy$ ir $4xy^{2}$ nėra panašūs nariai, antrojo nario kintamasis $y$ pakeltas kvadratu.
                        laipsniu.
            \end{itemize}
\end{enumerate}

\bibliographystyle{plain}
\bibliography{bibliography.bib}
\end{document}