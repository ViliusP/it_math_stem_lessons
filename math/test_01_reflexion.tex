\documentclass[a4paper]{article}

\usepackage{fullpage} % Package to use full page
\usepackage{parskip} % Package to tweak paragraph skipping
\usepackage{tikz} % Package for drawing
\usepackage{tkz-euclide}
\usetikzlibrary{fit,positioning}
\usetikzlibrary {arrows.meta}
\usetikzlibrary {patterns,patterns.meta}

\usepackage{amsmath,amssymb}

\usepackage{amsmath}
\usepackage{hyperref}
\usepackage[main=lithuanian, german, shorthands=off]{babel}
\usepackage{tgpagella}
\usepackage[L7x,T1]{fontenc}
\usepackage[utf8]{inputenc}
\usepackage{enumitem}
\usepackage{booktabs} % For better looking tables
\usepackage{venndiagram}
\usepackage{subfig}
\newcommand{\germanqq}[1]{{\selectlanguage{german}\glqq#1\grqq\selectlanguage{english}}}

% Definition of circles
\def\firstcircle{(0,0) circle (1.5cm)}
\def\secondcircle{(0:2cm) circle (1.5cm)}

\colorlet{circle edge}{blue!50}
\colorlet{circle area}{blue!20}

\tikzset{filled/.style={fill=circle area, draw=circle edge, thick},
      outline/.style={draw=circle edge, thick}}

\setlength{\parskip}{5mm}

\tikzset{
      venn box/.style={
                  draw=black, very thick,
                  rounded corners=10,
                  inner xsep=10pt, inner ysep=15pt, outer ysep=5pt
            },
      venn numbers/.style={
                  %    draw,
                  inner ysep=0pt,
                  align=center
            },
      venn title/.style={
                  fill=black, text=white
            }
}

\title{Savarankiško darbo refleksija}
\author{Vilius Paliokas}
\date{2023/09/29}

\begin{document}

\maketitle

\section{Lygtys}

\textbf{Lygtis}: matematinis teiginys, teigiantis dviejų reiškinių lygybę.

\textbf{Sprendinys}: reikšmė (arba reikšmių rinkinys), dėl kurios lygtis yra
teisinga, kai jos kintamasis (dažniausiai $x$) pakeičiamas ja (reikšme).

\subsection{Lygties sprendimas}

Pagrindiniai žingsniai:

\begin{enumerate}
      \item \textbf{Supaprastinimas}: suprastinamos abi lygties pusės (panašių
            narių jungimas, perkėlimai, skliaustų atskleidimai ir kt.);
      \item \textbf{Izoliuojamas kintamasis}: Naudojami aritmetiniai veiksmai
            ir atvirkštinės operacijos (jeigu lygybė, tai atimtis; jeigu
            kėlimas laipsniu, tai šaknies traukimas ir t.t.), kad kintamasis
            (dažniausiai
            $x$) būtų vienintelis
            kažkurioje tai lygties pusėje.
      \item \textbf{Atsakymo pasitikrinimas}: gavus sprendinį, įdedamas vietoje
            kintamojo ir patikrinama, kad abi pusės lygios.

\end{enumerate}

Pagrindiniai aspektai:
\begin{enumerate}
      \item \textbf{Atvirkštinės operacijos}:
            naudojamos operacijos, kurios atšaukia viena kitą (pvz., sudėjimas
            ir atėmimas, daugyba ir padalijimas).

            \begin{table}[h]
                  \centering
                  \begin{tabular}{cc}
                        \toprule
                        Operacija                                & Atvirkštinė
                        operacija
                        \\
                        \midrule
                        Sudėtis $(+a)$                           & Atimtis
                        $(-a)$
                        \\
                        Atimtis $(-a)$                           & Sudėtis
                        $(+a)$
                        \\
                        Daugyba $(\times a)$                     & Dalyba
                        $(\div a)$
                        \\
                        Dalyba $(\div a)$                        & Daugyba
                        $(\times
                              a)$
                        \\
                        Kėlimas kvadratu $(x^2)$                 & Kvadratinė
                        šaknis
                        $(\sqrt{x})$
                        \\
                        Kėlimas kubu $(x^3)$                     & Kubinė
                        šaknis
                        $(\sqrt[3]{x})$
                        \\
                        Kėlimas laipsniu $(x^a)$                 & Šaknis
                        \((\sqrt[a]{x})\)
                        \\
                        Logaritmas pagrindu  \(b\) $(\log_b{x})$ & Kėlimas, kai
                        pagrindas konstanta \(b\) $(b^x)$
                        \\
                        \bottomrule
                  \end{tabular}
                  \caption{operacijos ir jų atvirkštinės operacijos}
                  \label{tab:inverse_operations}
            \end{table}

      \item \textbf{Panašieji nariai}: Atliekamos operacijos su panašiais
            nariais. Panašieji nariai - tai tie, kurie turi tą patį kintamąjį
            ir pakelti
            tuo pačiu laipsniu (daugiau žiūrėti nelygybių temoje).
      \item \textbf{Lygties balansas}: Kad ir ką darytumėte vienai lygties
            pusei, turite daryti su kita.
\end{enumerate}

Lygybėms galioja veiksmų eiliškumas - taisyklių rinkinys, kuris nurodo, kokius
veiksmus reikia atlikti pirmiausia, kad būtų tinkamai apskaičiuota matematinė
išraiška.
Žemiau pateikiama operacijų atlikimo tvarka:

\begin{enumerate}
      \item Skliaustai;
      \item Kėlimas laipsniu, šaknies traukimas, logaritmavimas;
      \item Daugyba, dalyba (iš kairės į dešinę);
      \item Atimtis, sudėtis (iš kairės į dešinę).
\end{enumerate}

\subsection{Kaip išspręsti $ ax^{2}+bx=0 $}

\subsubsection{Teorinis sprendimas}

Žingsniai:

\begin{enumerate}
      \item Turime nepilną kvadratinę lygtį.

            $ ax^{2}+ bx = 0; $
      \item Išskaidome dauginamaisiais - iškeliame $ x $ prieš skliaustus:

            $ x(ax + b) = 0 $
      \item Iškėlus prieš skliaustus, jau turime vieną sprendinį ($ x $), kitą
            dar reikia susirasti:

            $ ax+b = 0 $ $\;\;\;$ arba $\;\;\;$ $ x=0 $
      \item Susitvarkome lygtį taip, kad vienoje pusėje atsirastų nariai su $ x
            $, o kitoje tik skaičiai. Tai padarysime atėmę iš abiejų pusių
            skaičių $ b $:

            $ ax+b = 0 | - b $
      \item  $ ax+b-b = 0-b $
      \item Reikia pasidaryti, kad kintamasis $ x $ būtų plikas - be dauginio $
                  a
            $. Tai padarysime padalinę lygtį iš to dauginio $ a $:

            $ ax = -b | : a $
      \item $ \frac{ax}{a} = -\frac{b}{a} $
      \item $ x = -\frac{b}{a} $
\end{enumerate}
Po 9 žingsnio turime du lygties sprendinius $ x = -\frac{b}{a} $ ir $ x=0 $ (3
žingsnis).

\subsubsection{Pavyzdys \#1}

Turime $ 2x^{2}-4x = 0; $

Pagal formulę $ ax^{2}+ bx = 0 $:

\begin{itemize}
      \item $ a = 2; $
      \item $ b = -4 $.
\end{itemize}

\begin{figure}[!htbp]
      \begin{center}
            \begin{tikzpicture}
                  \foreach \x in {-3, -2,...,-1,1,2} \draw (\x,2pt) --++
                  (0,-4pt)
                  node [below] {\x};
                  \foreach \y in {-3,...,-1,1,2} \draw (2pt,\y) --++ (-4pt,0)
                  node
                  [left] {\y};
                  \draw[domain=-.5:2.5, color=blue] plot (\x, {2*
                              (\x)^2-4*(\x)})
                  node[above = .5cm, right, color=blue] {$f(x)=2x^2-4x$};
                  \draw [thick, ->] (-3,0) -- (3,0) node [above] {$x$};
                  \draw [thick, ->] (0,-3) -- (0,3) node [right] {$y$};
                  \node at (0,0) {\textbullet};
                  \node at (2,0) {\textbullet};
            \end{tikzpicture}
      \end{center}
      \caption{$f(x)=2x^2-4x$ grafikas su sprendiniais $2x^2-4x=0$
      }\label{fx=2x2-4x}
\end{figure}

Žingsniai:
\begin{enumerate}

      \item  Išskaidome dauginamaisiais - iškeliame $ x $ prieš skliaustus:

            $ x(2x - 4) = 0 $;
      \item  Iškėlus prieš skliaustus, jau turime vieną sprendinį ($ x $), kitą
            dar reikia susirasti:

            $ 2x-4 = 0 $ $\;\;\;$ arba $\;\;\;$ $ \boldsymbol{x=0} $;
      \item  Susitvarkome lygtį taip, kad vienoje pusėje atsirastų nariai su $
                  x
            $, o kitoje tik skaičiai. Tai padarysime pridėję abiem pusėms
            skaičių
            $ 4 $:

            $ 2x-4 = 0 | + 4 $;
      \item  $ 2-4+4 = 0+4 $;
      \item Reikia pasidaryti, kad kintamasis $ x $ būtų plikas - be dauginio $
                  2
            $. Tai padarysime padalinę lygtį iš to dauginio $ 2 $:

            $ 2x = 4 | : 2 $;
      \item $ \textcolor{blue}{\frac{2x}{2}} = \textcolor{red}{\frac{4}{2}} $;

            $ \textcolor{blue}{\frac{2x}{2}}=\textcolor{blue}{x} $;

            $ \textcolor{red}{-\frac{4}{2}}=\textcolor{red}{2} $;

      \item $ \textcolor{blue}{x} = \textcolor{red}{2} $;
\end{enumerate}

Po 7 žingsnio turime du lygties sprendinius $ x = 2 $ ir $ x=0 $ (2 žingsnis).

\subsubsection{Pavyzdys \#2}

Turime $ 2x^2 + 3x^2 - 5x = 4x $.

Šis reiškinys neatitinka $ ax^{2}+ bx = 0 $ formulės. Todėl pirmiausia reikia
bandyti susitvarkyti.

\begin{enumerate}
      \item Visus narius perkeliame į vieną pusę:

            $ 2x^2 + 3x^2 - 5x = 4x | - 4x; $

            $ 2x^2 + 3x^2 - 5x - 4x = 4x - 4x; $

            $ 2x^2 + 3x^2 - 5x - 4x = 0; $

      \item Sutraukiame panašius narius:

            $ \textcolor{blue}{2x^2} + \textcolor{blue}{3x^2} \textcolor{red}{-
                        5x} \textcolor{red}{- 4x} = 0; $

            $ \textcolor{blue}{5x^2} \textcolor{red}{- 9x} = 0; $

      \item Dabar jau reiškinys atitinka $ ax^{2}+ bx = 0 $ formulę. Galima
            išskaidyti dauginamaisiais - iškeliame prieš skliaustus $ x $:

            $ x(5x - 9) = 0 $;

      \item Iš čia gauname vieną sprendinį:

            $ 5x-9 = 0 $ $\;\;\;$ arba $\;\;\;$ $ \boldsymbol{x=0} $;

      \item Toliau sprendžiame pirmąją lygtį:

            $ 5x-9 = 0 | + 9 $;

            $ 5x-9+9 = 0+9 $;

            $ 5x = 9 $;

            $ 5x = 9|:5 $ arba $ 5x = 9|\cdot \frac{1}{5}$;

            $ \frac{5x}{5} = \frac{9}{5} $ arba $ 5x\cdot\frac{1}{5} = 9\cdot
                  \frac{1}{5}$;

            abiejais atvejais $ x = 1.8 $.

\end{enumerate}

Gauname, kad $ 2x^2 + 3x^2 - 5x = 4x $ lygties sprendiniai yra $x=0$ ir $ x =
      1.8 $ (galima dar rašyti $ x \in \{0, 1.8\} $).

\subsection{Kaip išspręsti $ ax^{2}+b=0 $}\label{sec:ax_square_equal_number}

\subsubsection{Teorinis sprendimas}

Žingsniai:
\begin{enumerate}

      \item  Išskiriame $ ax^{2} $ (paliekame kairėje pusėje be $ b $):

            $ ax^{2}+b=0 | -b $;

            $ ax^{2}+b-b=0-b $;

            $ ax^{2}=-b $;

      \item Kairėje pusėje reikia palikt $ x^2 $ - abi puses padaliname iš $ a
            $:

            $ ax^{2}=-b |:a $;

            $ \frac{ax^{2}}{a}=\frac{-b}{a}$;

            Kairėje pusėje galima suprastinti skaitiklyje ir vardiklyje
            esančius
            $ a $:

            $ x^{2}=\frac{-b}{a}$;

      \item Ištraukiame šaknį iš abiejų pusių:

            Visos kvadratinės lygtys turi du sprendinius (išskyrus, $ x^2=0 $),
            tai ištraukus šaknį:

            $ \sqrt{x^{2}}=\sqrt{\frac{-b}{a}}$;

            $ x=\sqrt{\frac{-b}{a}}$;

            ir

            $ \sqrt{x^{2}}=-\sqrt{\frac{-b}{a}}$;

            $ x=-\sqrt{\frac{-b}{a}}$;

\end{enumerate}

Šis sprendimas turi prasmę, kol $ x \neq 0 $ (dalijimas iš nulio neturi
reikšmės) ir $ \frac{-b}{a} \ge 0 $ (traukiant šaknį iš neigiamo skaičiaus
gaunamas kompleksinis skaičius - mokykloje to nesimokoma).

\subsubsection{Pavyzdys \#1}

Turime $ 2x^{2}+8=0 $. Ši atitinka $ ax^{2}+b=0 $ formą. Sprendžiame pagal
auksčiau duotą teorinį sprendimą:

\begin{enumerate}
      \item  Išskiriame $ ax^{2} $ (paliekame kairėje pusėje be $ b $):

            $ 2x^{2}+8=0 | -8 $;

            $ 2x^{2}+8-8=0-8 $;

            $ 2x^{2}=-8 $;

      \item Kairėje pusėje reikia palikt $ x^2 $ - abi puses padaliname iš $ 2
            $:

            $ 2x^{2}=-8 |:2 $;

            $ \frac{2x^{2}}{2}=\frac{-8}{2}$;

            Kairėje pusėje galima suprastinti skaitiklyje ir vardiklyje
            esančius
            $ a $, o dešinėje padalinti skaičius:

            $ x^{2}=-4$;

      \item Ištraukiame šaknį iš abiejų pusių:

            Visos kvadratinės lygtys turi du sprendinius (išskyrus, $ x^2=0 $),
            tai ištraukus šaknį:

            $ \sqrt{x^{2}}=\sqrt{-4}$;

            $ x=\sqrt{-4}$;

            ir

            $ \sqrt{x^{2}}=-\sqrt{-4}$;

            $ x=-\sqrt{-4}$;

\end{enumerate}

Kadangi dešinėje pusėje esantis skaičius yra mažiau už nulį (-4<0), tai ši
lygtis neturi realiųjų sprendinių.

\subsubsection{Pavyzdys \#2}

Turime $ 6x^{2}=3x^{2}+12 $. Ši lygtis neatitinka $ ax^{2}\pm b=0 $ formos.
Todėl pirmiausia reikia bandyti susitvarkyti.

\begin{enumerate}

      \item Persikeliame narius su $ x^2 $ į vieną pusę (pasirenkame kairę),
            tai
            galima padaryti atėmus abi puses iš $ 3x^{2} $:

            $ 6x^{2}=3x^{2}+12|-3x^{2} $;

            $ 6x^{2}-3x^{2}=3x^{2}+12-3x^{2} $;

            $ 3x^{2}=12 $;

      \item Dabar reiškinys atitinka $ ax^{2}-b=0 $, nes tai yra tas pats kas $
                  ax^{2}=b $. Toliau sprendžiame pagal taisykles, reikia $ x^2
            $
            palikti be
            skaičiaus esančio priekyje, tai padarysime padaline iš skaičiaus
            esančio prieš
            $ x^{2} $:

            $ 3x^{2}=12 |: 3 $;

            $ \frac{3x^{2}}{3}=\frac{12}{3}$;

            Kairėje pusėje galima padalinti 3 iš 3, o dešinėje 12 iš 3:

            $ x^{2}=4 $;

      \item Dabar galima iš abiejų pusių ištraukti šaknį:

            $ \sqrt{x^{2}}=\sqrt{4}$;

            $ x=2$;

            ir

            $ \sqrt{x^{2}}=-\sqrt{4}$;

            $ x=-2$;

\end{enumerate}

Lygtis $ 6x^{2}=3x^{2}+12 $ turi du sprendinius: $ x=2 $ ir $x=-2$. Sprendinius
visada galima pasitikrinti įdėjus atgal į lygtį.

\subsubsection{Pavyzdys \#3}

Nevisada išeis ištraukti šaknį \germanqq{gražiai} sprendžiant $ ax^{2}+b=0 $
lygtį. Pavyzdžiui turime paprastą lygtį $ x^{2}=40 $.

\begin{enumerate}

      \item Iš karto galime ištraukti šaknį iš abiejų pusių:

            $ \sqrt{x^{2}}=\sqrt{40}$;

            $ x=\sqrt{40}$;

            arba

            $ \sqrt{x^{2}}=-\sqrt{40}$;

            $ x=-\sqrt{40}$;

      \item Nors galėtume čia ir baigti spręsti, bet dar galime išskaidyti
            dauginamaisiai ir dalinai ištraukti šaknį:

            Šiam tikslui naudosime vieną iš šaknų savybių (žiūrėti bendrojo
            kurso
            brandos egzamino formulyną):
            $ \sqrt[n]{a\cdot b} = \sqrt[n]{a} \cdot \sqrt[n]{b}$.

            $ x=\sqrt{4\cdot10} = \sqrt{4}\cdot\sqrt{10} = 2\sqrt{10}$, nes $40
                  =
                  4 \cdot 10$;

            arba

            $ x=-\sqrt{4\cdot10} = -\sqrt{4}\cdot\sqrt{10} = -2\sqrt{10}$;

\end{enumerate}

Lygtis $ x^{2}=40 $ turi du sprendinius: $ \pm2\sqrt{10} $. Sprendinius visada
galima pasitikrinti įdėjus atgal į lygtį.

\section{Nelygybės}

Nelygybės išreiškia ryšį tarp dviejų dydžių, kurie nėra lygūs. Jose naudojami
kintamieji ir konstantos, o nelygybės simboliais parodoma, kad viena teiginio
pusė yra didesnė arba mažesnė už kitą.

Naudojami simboliai:

\begin{enumerate}
      \item Daugiau už ($>$), pavyzdžiui: $ x>3 $ (skaitoma $x$ daugiau už 3);
      \item Mažiau už ($<$), pavyzdžiui:	$ x<5 $, (skaitoma $x$ mažiau
            už 5);
      \item Daugiau už arba lygu ($ \geq $), pavyzdžiui: $ x \geq 4$ (skaitoma
            $x$ daugiau arba lygu už 4);

            Vietoje \germanqq{daugiau už arba lygu} galima vartoti \germanqq{ne
                  mažiau}.
      \item Mažiau už arba lygu ($ \leq $), pavyzdžiui: $ x \leq 6$ (skaitoma
            $x$
            mažiau arba lygu už 4).

            Vietoje \germanqq{mažiau už arba lygu} galima vartoti \germanqq{ne
                  daugiau}.

\end{enumerate}

\subsection{Pagrindiniai principai sprendžiant nelygybes}

Sprendžiant nelygybes, pritaikomi tokie pat principai, kaip ir lygtyse (ir
atvirkščiai).
Prisideda tik nelygybės apvertimas dauginant ar dalinant iš negiamo skaičiaus.

\begin{enumerate}
      \item Kad ir ką darytumėte vienai nelygybės pusei, turite padaryti kitai,
            kad išlaikytumėte nelygybę;

            $x+3>5$ tampa $x>2$ atėmus 3 iš abiejų pusių.

      \item Nelygybės apvertimas:

            \begin{enumerate}[label*=\arabic*.]

                  \item Kai padauginate arba padalijate abi nelygybės puses iš
                        neigiamo skaičiaus, nelygybės ženklas turi būti
                        apverstas.

                        $-2x>6$ tampa $x<-3x$ padalinus nelygybę iš $-2$
                        ($\boldsymbol{>} \rightarrow \boldsymbol{<}$).

                  \item Jeigu yra perkeliamas narys iš vienos nelygybės pusės į
                        kitą, tai reiktų laikyti tai, kaip to nario pridėjimą
                        ar
                        atėmimą iš abiejų
                        pusių. Būtina atkreipti dėmesį į ženklą:

                        $3\boldsymbol{\textcolor{red}{>}}x$ tampa
                        $x\boldsymbol{\textcolor{blue}{<}}3$ (atkreipkite
                        dėmesį į
                        ženklą), nes

                        $3\boldsymbol{\textcolor{red}{>}}x |-3 \Rightarrow$

                        $\Rightarrow 0\boldsymbol{\textcolor{red}{>}}x-3|-x
                              \Rightarrow$

                        $\Rightarrow
                              -x\boldsymbol{\textcolor{red}{>}}-3|\cdot-1
                              \Rightarrow$

                        $\Rightarrow x\boldsymbol{\textcolor{blue}{<}}3$.

            \end{enumerate}

      \item Panašių narių tvarkymas.

            Panašūs nariai turi tuos pačius kintamuosius ($x$, $y$, $z$
            ir kt.), kuriuo pakelti tais pačiais laipsniais.
            Iš esmės jie atrodo taip pat, išskyrus koeficientą prie jo
            (skaičius prieš kintamąjį).

            Pavyzdžiai:
            \begin{itemize}
                  \item $5x$ ir $3x$ yra panašūs nariai, nes abu turi kintamąjį
                        $x$
                        ir jie pakilti pirmuoju ($x^{1}=x$), nors ir
                        koeficientai
                        (5 ir 3) prie šių
                        kintamųjų skirtingi.
                  \item $7y^{2}$ ir $-y^2$ yra panašūs nariai, nes abu turi
                        kintamąjį $y$ ir jie pakilti antruoju laipsniu, nors ir
                        koeficientai (5 ir
                        -1) prie šių kintamųjų skirtingi.
                  \item $-4ab$ ir $5ab$ yra panašūs nariai, nes abu turi
                        kintamuosius $a$ ir $b$, bei jie pakilti pirmuoju
                        laipsniu.
            \end{itemize}

            Atvirkštiniai pavyzdžiai:
            \begin{itemize}
                  \item $3x$ ir $3y$ nėra panašūs nariai, nes abu turi
                        skirtingus kintamuosius
                        $x$ ir $y$.
                  \item $x^{2}$ ir $x$ nėra panašūs nariai, nes abu kintamieji
                        pakelti skirtingais laipsniais (2 ir 1);
                  \item $4xy$ ir $4xy^{2}$ nėra panašūs nariai, antrojo nario
                        kintamasis $y$ pakeltas kvadratu.
                        laipsniu.
            \end{itemize}

            \begin{enumerate}[label*=\arabic*.]

                  \item Konstantos ir kintamieji turi būti suprastinti, jeigu
                        tai įmanoma:

                        $2x+5>x+8$ tampa $x>3$ atėmus abiu puses iš $x$ ir $5$.

                  \item Panašūs nariai gali būti sudėti arba atimti:

                        $3x+2x>10$ tampa $5x>10$, o po to ir $x>2$.

            \end{enumerate}

\end{enumerate}

\subsection{Atsakymo pasitikrinimas}

Visada galima pasitikrinti nelygybės atsakymą. Pavyzdžiui turime nelygybę
$2x+3<11$:

\begin{enumerate}
      \item Atimame abi pusęs iš 3

            $2x+3-3<11-3$;

            $2x<8$;

      \item Padaliname abi pusęs iš 2

            $\frac{2x}{2}<\frac{8}{2}$;

            $x<4$;
\end{enumerate}

Radome, kad nelygybės sprendinys yra $x<4$ arba $x\in (-\infty;4)$. Galime
pasitikrinti šį sprendinį įstatydami skaičių mažesnį negu 4, pavyzdžiui 3.
Įstačius į pradinę nelygybę gauname, kad $2\cdot3+3<11$. Atlikus aritmetinius
veiksmus gauname, kad $9<11$, kas yra tiesa ir tai reiškia, kad sprendinys yra
teisingas.

\subsection{Kaip spręsti $ax-b<0$ nelygybę?}\label{sec:ax_minus_b_inequality}

\subsubsection{Teorinis sprendimas}

Žingsniai, kad išspręstume $ax-b<0$ nelygybę:

\begin{enumerate}
      \item Prie abiejų pusių pridedame $b$:

            $ax-b+b<0+b$;

            $ax<b$;

      \item Padaliname iš $a$, kad paliktumę kintąmjį $x$ be koeficiento
            (daugiklio):

            $\frac{ax}{a}<\frac{b}{a}$;

            $x<\frac{b}{a}$;

      \item Neužmirškite, jeigu skaičius $a$ yra neigiamas, reikia apversti
            nelygybės ženklą:

            $x>-\frac{b}{a}$;

\end{enumerate}

\subsubsection{Pavyzdys \#1}

Turime $3x-5 < 0$. Sprendimas:

\begin{enumerate}
      \item Prie abiejų pusių pridedame $5$:

            $3x-5+5<0+5$;

            $3x<5$;

      \item Padaliname abi puses iš $3$, kad paliktumę kintąmjį $x$ be
            koeficiento (daugiklio):

            $\frac{3x}{3}<\frac{5}{3}$;

            $x<\frac{5}{3}$;

\end{enumerate}

Nelygybės sprendinys: $x\in (-\infty;\frac{5}{3})$.

\subsubsection{Pavyzdys \#2}

Turime $-3x+2\geq 5x-8$. Sprendimas:

\begin{enumerate}
      \item Visus narius su $x$ kintamuoju perkeliame į vieną pusę. Aš
            pasirenku kelti į dešinę:

            $-3x+2\geq 5x-8 | -5x$;

            $-3x+2-5x \geq 5x-8-5x$;

            $-8x+2 \geq -8$.

      \item Visus skaičius be kintamųjų (konstantas) perkeliame į kitą pusę.
            Šiuo atveju į kairę:

            $-8x+2 \geq -8|-2$;

            $-8x+2-2 \geq -8-2$;

            $-8x \geq -10$;

      \item Panaikiname skaičių prie $x$ padalindami abi puses iš jo:

            $-8x \geq -10|:(-8)$;

            $\frac{-8x}{-8} \leq \frac{-10}{-8}$ (atkreipkite dėmesį į ženklo
            pasikeitimą);

            $x \leq \frac{5}{4} $;

\end{enumerate}

Nelygybės sprendinys: $x\in (\infty;\frac{5}{4}]$.

\section{Aibės}

Aibė yra skirtingų elementų rinkinys. Jeigu elementas $a$ yra aibės $A$
elementas, tai rašoma, kad $a \in A$. Jeigu elementas $b$ nėra aibės $A$
elementas, tai rašoma, kad $b \notin A$. Aibės žymimos didžiąją raide, o jos
elementai mažosiomis. Matematikos šaka nagrinėjanti aibes
vadinama aibių teorija.

Aibės pavyzdžiai:

\begin{itemize}
      \item mokyklos mokinių aibė;
      \item saulės sistemos planetų aibė;
      \item visų natūraliųjų skaičių aibė;
      \item lygties sprendinių aibė;
      \item ...
\end{itemize}

\subsection{Būdai užrašyti aibę}

Pagal elementų skaičių, yra dviejų tipų aibės: baigtinės ir begalinės.
Baigtines aibes galima lengvai išrašyti. Tokiam būdui naudojami figūriniai
skliaustai $\{...\}$. Pavyzdžiui:

$$ A=\{1;5;6;7;8;10;30\}; \quad B=\{c;b;e\}; $$

Dvi aibės yra vienodos, jeigu šių elementai nesiskiria, nors ir skiriasi jų
išdėstymo tvarka. Pavyzdžiui, $\{1,2,3\}=\{2,3,1\}$. Bet matematikoje sutarta,
jeigu aibės elementai yra skaičiai, tai jie užrašomi didėjimo tvarka.

Taip pat baigtines ir begalines aibes galima užrašyti tam tikromis taisyklėmis:

\begin{itemize}
      \item taisyklėmis.

            $ B=\{x | x \text{ yra pirminis skaičius mažesnis už 10}\} $ - toks
            užrašymas reikštų, kad aibę $ B $ sudaro pirminiai skaičiai mažesni
            už 10. Tokią aibę dar būtų galima užrašyti šitaip $ B=\{2;3;5;7\}
            $.

      \item žodiniu apibūdinimu.

            Tegul, aibė $ C $ yra sudaryta iš visų sveikųjų skaičių mažesnių už
            100.

      \item intervalu.

            $ D = (2;5)$ - tokia aibė yra sudaryta iš visų realių skaičių nuo 2
            neįskaitant iki 5 neįskaitant.

            $ E = [2;5]$ - tokia aibė yra sudaryta iš visų realių skaičių nuo 2
            įskaitant iki 5 įskaitant.
      \item Veno diagramomis.

            Nors tai nėra tekstinis aibės apibrėžimo būdas, Veno diagrama
            vizualiai vaizduoja aibes ir jų ryšius (įprastai apskritimais).

            \begin{venndiagram3sets}[labelOnlyA={1}, labelOnlyB={2},
                        labelOnlyC={}, labelOnlyAB= {}, labelOnlyAC={},
                        labelOnlyBC={9}, labelABC={3},
                        vgap=.75cm]

            \end{venndiagram3sets}
\end{itemize}

\subsection{Realiųjų skaičių aibė}

Matematikoje, yra skaičių rinkiniai, kurie naudojami taip dažnai, kad jie turi
specialius pavadinimus ir simbolius:

\begin{enumerate}
      \item Natūralūs ($\mathbb{N}$);
      \item Sveikieji ($\mathbb{Z}$);
      \item Racionalieji ($\mathbb{Q}$);
      \item Iracionalieji ($\mathbb{I}$);
      \item Realūs ($\mathbb{R}$);
      \item ir kt.
\end{enumerate}

\begin{figure}[!htbp]
      \centering
      \begin{tikzpicture}[node distance=12pt]
            \node[venn box] (N) {%
                  $0 \quad 1 \quad 12 \quad 350 \quad 48 \quad 5 \quad 16$
            };

            \node[venn numbers, below=of N] (Z-N) {
                  $-1 \quad {-5} \quad {-10} \quad {-39}$
            };
            \node[venn box, fit=(Z-N) (N)] (Z) {};

            \node[venn numbers, below=of Z, align=center] (Q-Z) {
            $25{,}3401401401\dots \quad 48{,}259$ \\[5pt]
            $-0{,}101232323\dots \quad \dfrac52 \quad {-\dfrac73}$
            };
            \node[venn box, fit=(Q-Z) (Z)] (Q) {};

            \node[venn numbers, below=of Q, inner sep=8pt, align=center, ] (Q')
            {
            $6{,}1010010001\dots \quad {-0{,}1234567\dots}$ \\[5pt]
            $\sqrt{2} \quad \pi$
            };

            \node [venn box, inner sep=8pt, fit=(Q')] (Q') {};

            \node [venn box, fit=(Q') (Q)] (R) {};

            \tikzset{every node/.style=venn title}
            \foreach \i/\j/\k in {N/Natūralūs/0, Z/Sveikieji/10pt,
                        Q/Racionalūs/10pt,
                        Q'/Iracionalūs/0pt, R/Realūs/15pt} {
                        \draw node[anchor=north west] at ([shift={(10pt,
                                          3pt)}]\i.north west) {\j}
                        node[rounded corners] at ([yshift=-\k]\i.east)
                        {$\mathbb{\i}$};
                  }
      \end{tikzpicture}
      \caption{realių skaičių aibė ir jos poaibiai}\label{fig:real_number_set}
\end{figure}

Mokyliniame kurse yra mokomi tik 5 pagrindinės skaičių aibės. Bet reiktų
žinoti, kad jų yra ir daugiau, pavyzdžiui menamasis vienetas ir kompleksniai
skaičiai. Toliau apibūdinama realiųjų skaičių aibė ir jos poaibiai.

\subsubsection{Natūralūs skaičiai}

\begin{itemize}
      \item \textbf{Apibrėžimas}: skaičiai naudojami skaičiuoti, nuo 1 iki
            begalybės (kartais įtraukiamas ir 0).
      \item \textbf{Simbolis}: $\mathbb{N}$.
      \item \textbf{Pavyzdys}: $\mathbb{N}=\{1;2;3;4;5;6;7;8;9;\ldots\}$.
\end{itemize}

\subsubsection{Sveikieji skaičiai}

\begin{itemize}
      \item \textbf{Apibrėžimas}: 0, visi natūralieji skaičiai ir jiems
            atvirkštiniai skaičiai (natūralieji skaičiai su minuso ženklu).
      \item \textbf{Simbolis}: $\mathbb{Z}$.
      \item \textbf{Pavyzdys}:

            $\mathbb{Z}=\{\ldots;-9;-8;-7;-6;-5;-4;-3;-2;-1;0;1;2;3;4;5;6;7;8;9;\ldots\}$.
\end{itemize}

\subsubsection{Racionalieji skaičiai}

\begin{itemize}
      \item \textbf{Apibrėžimas}: Skaičiai, kurie gali būti užrašyti trupmena
            $\frac{a}{b}$, kur $a$ ir $b$ sveikieji skaičiai, o $b \neq 0$.
      \item \textbf{Simbolis}: $\mathbb{Q}$.
      \item \textbf{Pavyzdys}: $\mathbb{Q}=\{\ldots;-\frac{8}{6}, -\frac{1}{1},
                  \frac{0}{1}, \frac{1}{2}, \frac{22}{7}\ldots\}$.

            Ankščiau aprašytą sveikųjų skaičių aibę $\mathbb{Z}$, taip pat
            galima
            išreikšti per racionaliųjų skaičių aibę: sveikieji skaičiai yra
            tie,
            racionalieji skaičiai $\frac{a}{b}$, kurių vardiklis $b$ yra lygus
            $1$.
\end{itemize}

\subsubsection{Iracionalieji skaičiai}

\begin{itemize}
      \item \textbf{Apibrėžimas}: Skaičiai, kurių negalima išreikšti trupmena
            $\frac{a}{b}$, kur $a$ ir $b$ sveikieji skaičiai. Šių skaičių
            dešimtainė dalis yra nesikartojanti ir nesibaigianti.
      \item \textbf{Simbolis}: $\mathbb{Q'}$ arba $\mathbb{R} \setminus
                  \mathbb{Q}$ (realiųjų skaičių ir racionaliųjų skaičių aibės
            skirtumas).
      \item \textbf{Pavyzdys}: $\sqrt{2}$, $\pi$, $e$.
\end{itemize}

\subsubsection{Realių skaičių aibės ir poaibių hierachija}
Visos aukščiau nurodytos aibės yra kažkokios tai kitos aibės poaibis. Šį ryšį
galima pamatyti veno diagramose \ref{fig:real_number_set} paveiksle.

\begin{itemize}
      \item $\mathbb{N}$ yra sveikųjų skaičių aibės $\mathbb{Z}$ poaibis.
      \item $\mathbb{Z}$ yra racionaliųjų skaičių aibės $\mathbb{Q}$ poaibis.
      \item $\mathbb{Q}$ yra realiųjų skaičių aibės $\mathbb{R}$ poaibis.
      \item Iracionalieji skaičiai taip pat yra realiųjų skaičių aibės
            $\mathbb{R}$ poaibis.
\end{itemize}

Šiuos ryšius galima taip pat pavaizduoti su simboliais:
\[ \mathbb{N} \subset \mathbb{Z} \subset \mathbb{Q} \subset \mathbb{R}; \]

\subsection{Veiksmai su aibėmis}

Kaip ir su skaičiais, taip ir su aibėmis galima atlikti tam tikras veiksmus.
Mokykliniame kurse reikia žinoti keturias operacijas: \textbf{sąjunga},
\textbf{sankirta}, \textbf{skirtumas}, \textbf{poaibis}. Jų yra ir daugiau, bet
sužinosite vėliau, jeigu mokysitės matematikos giliau.

\subsubsection{Sąjunga}

Dviejų aibių, $A$ ir $B$, sąjunga žymima $A \cup B$. Šio veiksmo rezultatas yra
aibė sudaryta iš elementų, kurie yra arba aibėje $A$, arba aibėje $B$.

Pavyzdžiui, turime aibes $A=\{1;2;3\}$ ir  $B=\{3;4;5\}$, tada $A \cup B =
      \{1;2;3;4;5\}$. Aibė $A$ sudaryta iš elementų 1, 2, 3, o aibė $B$ - 3, 4,
5. Tai šių aibių sąjunga yra nauja aibė, kuri sudaryta iš 1, 2, 3, 4, 5, 6.
Kadangi abi $A$ ir $B$ aibės turi tą patį elementą 3, jo du kart neįtraukiame,
nes aibė turi būti sudaryta tik iš skirtingų elementų.

Šių aibių sąryšį galima atvaizduoti ir Veno diagramomis
(\ref{fig:set_union} pav.):

\begin{figure}[!htbp]
      \centering
      \begin{tikzpicture}
            \draw[filled] \firstcircle node {$A$}
            \secondcircle node {$B$};
            \node[anchor=south] at (current bounding box.north) {$A \cup B$};
      \end{tikzpicture}
      \caption{aibių sąjunga} \label{fig:set_union}
\end{figure}

Taip pat, žinome, kad aibes galima nurodyti intervalais. Tarkime turime aibes
$C=[-3;2)$ ir $D=(1;5]$. Prisiminkime, kad intervalai gali būti:
\begin{itemize}
      \item atviri: $(a;b)$ aibė sudaryta iš visų realių skaičių tarp $a$ ir
            $b$, bet neįskaitant pačių $a$ ir $b$.
      \item uždari: $[a;b]$ aibė sudaryta iš visų realių skaičių tarp $a$ ir
            $b$, įskaitant pačius $a$ ir $b$.
      \item pusiau-atviri: $[a;b)$ arba $(a;b]$ aibė sudaryta iš visų realių
            skaičių tarp $a$ ir $b$, bet tik vienas iš skaičių $a$ arba $b$
            įtrauktas į
            aibę.
\end{itemize}

Ieškosime aibių $C$ ir $D$ sąjungos. Aibė $C$ yra sudaryta iš skaičių nuo -3
(įskaitant) iki 2 (neįskaitant), o aibė $D$ yra sudaryta iš skaičių nuo 1
(neįskaitant) iki 5 įskaitant. Šių aibių sąjunga bus visi skaičiai, kurie yra
aibėje $C$ arba aibėje $D$ (arba abiejose). Šitai galima užrašyti:
\[  C \cup D = [-3;2) \cup (1;5] \]
Tokią aibę galima išreikšti dar paprasčiau. Kai imame dviejų aibių sąjungą,
galutinis intervalas prasideda nuo mažiausios intervalų reikšmės (žiūrėti aibę
$C$), kuri, šiuo atveju, lygi -3, ir tęsiasi iki didžiausios reikšmės, kuri yra
5 (žiūrėti aibę $D$). Aibių sąjungos intervalą sudaro visi skaičiai tarp šių
dviejų taškų (mažiausio ir didžiausio - $-3$ ir $5$). Taip pat galima pamatyti,
kad intervalai nuo 1 (neįskaitant) iki 2 (neįskaitant) persidengia - priklauso
abiems intervalams. Tokiu atveju galutinis rezultatas:
\[  C \cup D = [-3;2) \cup (1;5] = [-3;5] \]
Intervalus patogu atvaizduoti skaičių tiesėmis, tai sprendimą pavaizduosime
tokiu būdu (\ref{fig:set_union_example_interval} pav.):

\begin{figure}[!htbp]
      \centering
      \begin{tikzpicture}
            \begin{scope}[>=latex]
                  \draw [line width=0.7pt][->](-6,0)--(6,0);
            \end{scope}
            % Aibė D
            \draw [pattern={Lines[angle=45,distance={3pt/sqrt(2)}]},pattern
                  color=blue,line
                  width=0.6pt](1,0)--(1,0.5)--(5,0.5)--(5,0)--cycle;
            % Aibė C
            \draw [pattern={Lines[angle=135,distance={3pt/sqrt(2)}]},pattern
                  color=red,line
                  width=0.6pt](-3,0)--(-3,0.5)--(2,0.5)--(2,0)--cycle;
            \fill [color=black](-3,0.5) circle (2.3pt);
            \fill [color=white,draw=black,line width=0.7pt] (2,0.5) circle
            (2.3pt);
            \fill [color=white,draw=black,line width=0.7pt] (1,0.5) circle
            (2.3pt);
            \fill [color=black] (5,0.5) circle (2.3pt);
            \draw [line width=0.7pt](0,0)--(0,-0.3);
            \pgftext[base,x=-3cm,y=-0.3cm,] {\small $-3$};
            \pgftext[base,x=2cm,y=-0.3cm] {\small $2$};
            \pgftext[base,x=0cm,y=-0.6cm] {\small $0$};
            \pgftext[base,x=1cm,y=-0.3cm] {\small $1$};
            \pgftext[base,x=5cm,y=-0.3cm] {\small $5$};
      \end{tikzpicture}
      \caption{$C$ ir $D$ aibių sąjunga} \label{fig:set_union_example_interval}
\end{figure}

Raudona spalva pavaizduota $C=[-3;2)$, o mėlyna $D =(1;5]$. Atkreipkite dėmesį
į intervalo pradžios ir galo taškus. Pilnaviduriais rutuliukais pažymėti taškai
reiškia, kad šis skaičius įeina į intervalą, o tuščiaviduriu - neįeina į
intervalą. Iš skaičių tiesės galima matyti rezultatą - tai tas plotas, kur
skaičių tiesė
pažymėta raudonai arba mėlynai (arba abiem spalvomis).

\subsubsection{Sankirta}

Dviejų aibių, $A$ ir $B$, sankirta žymima $A \cap B$. Aibių sankirta yra
aibė sudaryta iš elementų, kurie yra \textbf{bendri} aibei $A$ ir aibei $B$.

Pavyzdžiui, turime vėl tas pačias aibes $A=\{1;2;3\}$ ir  $B=\{3;4;5\}$, tada
$A \cap B = \{3\}$. Kadangi aibė $A$ ir $B$ bendrą elementą 3.

Šių aibių sąryšį galima atvaizduoti ir Veno diagramomis
(\ref{fig:set_intersection} pav.).

\begin{figure}[!htbp]
      \centering
      % Set A and B
      \begin{tikzpicture}
            \begin{scope}
                  \clip \firstcircle;
                  \fill[filled] \secondcircle;
            \end{scope}
            \draw[outline] \firstcircle node {$A$};
            \draw[outline] \secondcircle node {$B$};
            \node[anchor=south] at (current bounding box.north) {$A \cap B$};
      \end{tikzpicture}
      \caption{aibių sankirta} \label{fig:set_intersection}
\end{figure}

Diagromeje matome akivaizdų bendrą plotą - sankirtą. Šis pažymėtas plotas
vaizduoja $A \cap B$.

Sankirta galima ir intervalams. Turime turime aibes $C=[-3;2)$ ir $D=(1;5]$.
Apie šiuos intervalus plačiau galite paskaityti skyriuje apie sąjunga. Ieškant
sankirtos, ieškosime, kuriose vietose intervalai turi tuos pačius elementus.

Mažiausias skaičius, kuris yra abiejuose intervaluose - \textbf{šiek tiek
      daugiau nei} 1. Šis skaičius (1) priklauso aibei $C$, nes $1 \in [-3;2)$,
bet nepriklauso aibei $D$, nes prie intervalo nurodytas \germanqq{(}
skliaustelis. Todėl ir bendras mažiausias skaičius yra \textbf{šiek tiek
      daugiau nei} $1 (1.00\ldots1)$.

Didžiausias skaičius priklausantis abiems intervalams yra \textbf{šiek tiek
      mažiau nei} $2 (1.99\ldots)$. Nors ir 2 priklauso aibei $D$, bet
nepriklauso
aibe $C$, nes 2 neįtrauktas į intervalą - \germanqq{)} prie skaičiaus.

Tokiu atveju mūsų aibių sankirta galima užrašyti:
\[  C \cap D = \text{skliaustas } \text{mažiausas bendras elementas};
      \text{didžiausias bendras elementas} \text{ skliaustas}  \]
\[  C \cap D = (1; 2)\]

Mūsų sankirta taip pat galime pamatyti, atvaizduojant intervalus skaičių
tiesėje:

\begin{figure}[!htbp]
      \centering
      \begin{tikzpicture}
            \begin{scope}[>=latex]
                  \draw [line width=0.7pt][->](-6,0)--(6,0);
            \end{scope}
            % Aibė D
            \draw [pattern={Lines[angle=45,distance={3pt/sqrt(2)}]},pattern
                  color=blue,line
                  width=0.6pt](1,0)--(1,0.5)--(5,0.5)--(5,0)--cycle;
            % Aibė C
            \draw [pattern={Lines[angle=135,distance={3pt/sqrt(2)}]},pattern
                  color=red,line
                  width=0.6pt](-3,0)--(-3,0.5)--(2,0.5)--(2,0)--cycle;
            \fill [color=black](-3,0.5) circle (2.3pt);
            \fill [color=white,draw=black,line width=0.7pt] (2,0.5) circle
            (2.3pt);
            \fill [color=white,draw=black,line width=0.7pt] (1,0.5) circle
            (2.3pt);
            \fill [color=black] (5,0.5) circle (2.3pt);
            \draw [line width=0.7pt](0,0)--(0,-0.3);
            \pgftext[base,x=-3cm,y=-0.3cm,] {\small $-3$};
            \pgftext[base,x=2cm,y=-0.3cm] {\small $2$};
            \pgftext[base,x=0cm,y=-0.6cm] {\small $0$};
            \pgftext[base,x=1cm,y=-0.3cm] {\small $1$};
            \pgftext[base,x=5cm,y=-0.3cm] {\small $5$};
      \end{tikzpicture}
      \caption{$C$ ir $D$ aibių sankirta}
      \label{fig:set_intersection_example_interval}
\end{figure}

Skaičių tiesėje aibių sankirta yra ta vieta, kur intervalai persikloja. Šiuo
atveju ten kur raudona ir mėlyna spalvos kartu - nuo 1 (neįskaitant) iki 2
(neįskaitant).

Bendru atveju, jeigu intervalai neturi sankirtos, tai tokia aibė yra tuščia ir
žymima $\{\}$
(aibė be nurodytų elementų) arba $\varnothing$.

\subsubsection{Skirtumas}

Dviejų aibių, $A$ ir $B$, sankirta žymima $A \setminus B$. Šių aibių sankirta
yra aibė sudaryta iš elementų, kurie priklauso $A$, bet nepriklauso $B$. Iki
šiol aprašytos operacijos (sąjunga ir sankirta) yra simetriškos operacijos. Tai
reiškia, kad apsukus narius (operandus) niekas nepasikeis.

Pavyzdžiui, turime vėl tas pačias aibes $A=\{1;2;3\}$ ir  $B=\{3;4;5\}$, tada
$A \cap B = \{3\}$, o $B \cap A = \{3\}$, tai reiškia, kad $ A \cap B = B \cap
      A $.

Taip pat ir su sąjunga: $A \cup B = \{1;2;3;4;5\}$, $B \cup A = \{1;2;3;4;5\}$,
tai reiškia, kad $ A \cup B = B \cup A $.

Nagrinėjamu atveju, aibės $A$ ir aibės $B$ skirtumas yra lygus $\{1;2\}$. Šią
aibę sudaro visi elementai iš aibės $A$, išskyrus tuos, kurie yra ir aibėje
$B$. Šiuo atveju aibė $A$ ir aibėjė $B$ turi bendrą elementą $3$, kurio ir
neįrašome rezultate.

Dabar apkeisime vietomis narius ir pažiūrėsime, kam lygus aibės $B$ ir $A$
skirtumas. Tokią aibę sudarys visi aibės $B$ elementai, išskyrus bendrus
elementus su aibe $A$. Vėl aibė $A$ ir aibėjė $B$ turi bendrą elementą $3$,
todėl jo neįtrauksime rezultatą. Tai $B \setminus A = \{4;5\}$

Aibių skirtumas nėra simetriška operacija, $A \setminus B \neq B \setminus A$,
todėl svarbu kokia tvarka rašomos aibės.

Šių aibių sąryšį galima atvaizduoti ir Veno diagramomis
(\ref{fig:set_difference} pav.), mėlyni plotai žymi aibių skirtumus.

\begin{figure}[!htbp]%
      \centering
      \subfloat[\centering aibių $A$ ir $B$ sankirta ]{{
                        \begin{tikzpicture}
                              \begin{scope}
                                    \clip \firstcircle;
                                    \draw[filled, even odd rule] \firstcircle
                                    node {$A$}
                                    \secondcircle;
                              \end{scope}
                              \draw[outline] \firstcircle
                              \secondcircle node {$B$};
                              \node[anchor=south] at (current bounding
                              box.north)  {$A \setminus B$};
                        \end{tikzpicture}
                  }}%
      \qquad
      \subfloat[\centering  aibių $B$ ir $A$ sankirta]{
            {
                        \begin{tikzpicture}
                              \begin{scope}
                                    \clip \secondcircle;
                                    \draw[filled, even odd rule] \firstcircle
                                    \secondcircle
                                    node {$B$};
                              \end{scope}
                              \draw[outline] \firstcircle node {$A$}
                              \secondcircle;
                              \node[anchor=south] at (current bounding
                              box.north) {$B \setminus A$};
                        \end{tikzpicture}
                  }}%
      \caption{Aibių sankirta}%
      \label{fig:set_difference}%
\end{figure}

Aibių skirtumas galimas ir intervalams. Turime turime aibes $C=[-3;2)$ ir
$D=(1;5]$. Apie šiuos intervalus plačiau galite paskaityti skyriuje apie
sąjunga. Ieškant skirtumo, ieškosime, intervalo (-ų), kurių skaičiai priklauso
pirmai aibei, bet nepriklauso antrąjai.

Ieškosime $C \setminus D$. Pirmiausia suraskime, kurie skaičiai yra intervale
$C$, bet nėra intevale $D$. Tokie skaičiai yra nuo -3 (įskaitant, dėl
skliaustelio prie skaičiaus) iki 1 (įskaitant, nes šis skaičius nepriklauso
aibei $D$, dėl skliaustelio, bet priklauso aibei $C$). Todėl galime užrašyti,
kad aibių skirtumas $C \setminus D = [-3;1]$.

Apkeiskime vietomis narius ir ieškokime $D \setminus C$. Pirmiausia suraskime,
kurie skaičiai yra intervale
$D$, bet nėra intevale $C$. Tokie skaičiai yra nuo 2, įskaitant. Skaičiai nuo 1
(1,1; 1,2; 1,5, 1,99 ir t.t.) priklauso aibei $C$ ir aibei $D$. Todėl ir
skirtumo intervalas, šiuo atveju, prasideda nuo 2, įskaitant. Galinis intervalo
skaičius yra 5, įskaitant. Tai $D \setminus C = [2;5]$

Skirtumus taip pat galime pamatyti, atvaizduojant intervalus skaičių
tiesėje (\ref{fig:set_difference_number_line} pav.):

\begin{figure}[!htbp]
      \centering
      \begin{tikzpicture}
            \begin{scope}[>=latex]
                  \draw [line width=0.7pt][->](-6,0)--(6,0);
            \end{scope}
            % Aibė D
            \draw [pattern={Lines[angle=45,distance={3pt/sqrt(2)}]},pattern
                  color=blue,line
                  width=0.6pt](1,0)--(1,0.5)--(5,0.5)--(5,0)--cycle;
            % Aibė C
            \draw [pattern={Lines[angle=135,distance={3pt/sqrt(2)}]},pattern
                  color=red,line
                  width=0.6pt](-3,0)--(-3,0.5)--(2,0.5)--(2,0)--cycle;
            \fill [color=black](-3,0.5) circle (2.3pt);
            \fill [color=white,draw=black,line width=0.7pt] (2,0.5) circle
            (2.3pt);
            \fill [color=white,draw=black,line width=0.7pt] (1,0.5) circle
            (2.3pt);
            \fill [color=black] (5,0.5) circle (2.3pt);
            \draw [line width=0.7pt](0,0)--(0,-0.3);
            \pgftext[base,x=-3cm,y=-0.3cm,] {\small $-3$};
            \pgftext[base,x=2cm,y=-0.3cm] {\small $2$};
            \pgftext[base,x=0cm,y=-0.6cm] {\small $0$};
            \pgftext[base,x=1cm,y=-0.3cm] {\small $1$};
            \pgftext[base,x=5cm,y=-0.3cm] {\small $5$};
      \end{tikzpicture}
      \caption{$C$ ir $D$ intervalai}\label{fig:set_difference_number_line}
\end{figure}

Intervalas $C$ pažymėtas raudonai, intervalas $D$ pažymėtas mėlynai. $C
      \setminus D$ yra skaičiai priklausantys aibei $C$, bet nepriklausantys
aibei
$D$, tai skaičių tiesėjė šita vieta yra pažymėta tik raudona spalva (mėlyna ir
persiklojusios raudona ir mėlyna netinka).

$D \setminus C$ yra skaičiai priklausantys aibei $D$, bet nepriklausantys aibei
$C$, tai skaičių tiesėjė šita vieta yra pažymėta tik mėlyna spalva (raudona ir
persiklojusios raudona ir mėlyna netinka).

Bendru atveju, jeigu intervalai neturi skirtumo, tai tokia aibė yra tuščia ir
žymima $\{\}$
(aibė be nurodytų elementų) arba $\varnothing$.

Paanalizuokime kitą situaciją, kai intervalai nepersikloja (skaičių tiesėje
nėra plotų su abiem spalvomis arba nėra brukšnelių iš apačios ir viršaus).
Turime aibes $F=[1; 3]$ ir $E=(5;7)$. Pavaizduokime šias aibes skaičių tiesėje
(\ref{fig:set_difference_number_line_2} pav.):

\begin{figure}[!htbp]
      \centering
      \begin{tikzpicture}
            \begin{scope}[>=latex]
                  \draw [line width=0.7pt][->](-1,0)--(8,0);
            \end{scope}
            % Aibė F
            \draw [pattern={Lines[angle=45,distance={3pt/sqrt(2)}]},pattern
                  color=blue,line
                  width=0.6pt](1,0)--(1,0.5)--(3,0.5)--(3,0)--cycle;
            % Aibė E
            \draw [pattern={Lines[angle=135,distance={3pt/sqrt(2)}]},pattern
                  color=red,line
                  width=0.6pt](5,0)--(5,0.5)--(7,0.5)--(7,0)--cycle;
            \fill [color=black](1,0.5) circle (2.3pt);
            \fill [color=black] (3,0.5) circle (2.3pt);
            \fill [color=white,draw=black,line width=0.7pt] (5,0.5) circle
            (2.3pt);
            \fill [color=white,draw=black,line width=0.7pt] (7,0.5) circle
            (2.3pt);
            \draw [line width=0.7pt](0,0)--(0,-0.3);
            \pgftext[base,x=1cm,y=-0.3cm,] {\small $1$};
            \pgftext[base,x=3cm,y=-0.3cm] {\small $3$};
            \pgftext[base,x=0cm,y=-0.6cm] {\small $0$};
            \pgftext[base,x=5cm,y=-0.3cm] {\small $5$};
            \pgftext[base,x=7cm,y=-0.3cm] {\small $7$};
      \end{tikzpicture}
      \caption{$F$ ir $E$ intervalai}\label{fig:set_difference_number_line_2}
\end{figure}

$F \setminus E$ yra aibės $F$ visu elementai, išskyrus bendrus elementus su
aibe $E$. Tarp aibių bendrų elementų nėra, nes skaičių tiesėje nematome
persiklojusių plotų. Tai todėl
\[F \setminus E = F = [1;3];\]
Apkeitus narius vietomis - situacija ta pati, todėl
\[E \setminus F = E = (5;7);\]

Paanalizuokime dar vieną išskirtinę situaciją, kai vienas intervalas yra viduje
kito - viena aibė yra kitos aibės poaibis. Turime aibes $G=[1;6]$ ir
$H=(3;5)$. Čia aibė $H$ yra aibės $G$ poaibis. Kad surastumėme aibių skirtumą
$G \setminus H$, reikia visų skaičių iš $G$, bet ne iš $H$. Atvaizdavus
intervalus skaičių tiesėje (\ref{fig:set_difference_number_line_3} pav.),
matome, kad vienas intervalas yra \germanqq{po} kitu intervalu.

\begin{figure}[!htbp]
      \centering
      \begin{tikzpicture}
            \begin{scope}[>=latex]
                  \draw [line width=0.7pt][->](-1,0)--(7,0);
            \end{scope}
            % set G
            \draw [pattern={Lines[angle=45,distance={3pt/sqrt(2)}]},pattern
                  color=blue,line
                  width=0.6pt](1,0)--(1,0.5)--(6,0.5)--(6,0)--cycle;
            % set H
            \draw [pattern={Lines[angle=135,distance={3pt/sqrt(2)}]},pattern
                  color=red,line
                  width=0.6pt](3,0)--(3,0.5)--(5,0.5)--(5,0)--cycle;
            \fill [color=black](1,0.5) circle (2.3pt);
            \fill [color=black] (6,0.5) circle (2.3pt);
            \fill [color=white,draw=black,line width=0.7pt] (3,0.5) circle
            (2.3pt);
            \fill [color=white,draw=black,line width=0.7pt] (5,0.5) circle
            (2.3pt);
            \draw [line width=0.7pt](0,0)--(0,-0.3);
            \pgftext[base,x=1cm,y=-0.3cm,] {\small $1$};
            \pgftext[base,x=6cm,y=-0.3cm] {\small $6$};
            \pgftext[base,x=0cm,y=-0.6cm] {\small $0$};
            \pgftext[base,x=3cm,y=-0.3cm] {\small $3$};
            \pgftext[base,x=5cm,y=-0.3cm] {\small $5$};
      \end{tikzpicture}
      \caption{$G$ ir $H$ intervalai}\label{fig:set_difference_number_line_3}
\end{figure}

Tokiu atveju skaičiai, kurie priklauso \textbf{tik} aibei $G$ yra nuo 1,
įskaitant, iki 3 (įskaitant, nes priklauso tik aibei $G$, bet nepriklauso $H$,
nes \germanqq{(} skliaustelis). Bet matome, kad yra dar viena sritis dešinėje,
kuri irgi priklauso tik aibei $G$. Tai dar turime sujungti kitą skaičių
intervalą, kuris yra nuo 5, įskaitant (ta pati situacija, kaip ir su 3), iki
6, įskaitant. Tai šį skirtumą galima užrašyti taip
\[G \setminus H = [1;3] \cup [5;6];\]
Patikrinkime, kas gausis apsukus narius - $ H \setminus G $. Turime parašyti
intervalą (-us), kurie priklauso tik aibei $H$, bet nepriklauso $G$.
Pasižiūrėjus į skaičių tiesę (\ref{fig:set_difference_number_line_3} pav.),
matome, kad tokių sričių nėra. Tai aibių skirtumą galime užrašyti:
\[H \setminus G = \varnothing = \{\} \]
Bendru atveju, Kai viena aibė yra kitos aibės poaibis, aibių skirtumas
paprastai gaunamas didesnės aibės intervalas (arba nesujungti intervalai), į
kurį neįeina mažesnė aibė. Tikslūs intervalai priklausys nuo konkrečių
susijusių aibių galinių taškų ir nuo to, ar jie įtraukti, ar neįtraukti.

Taip pat, kai viena aibė yra kitos aibės poaibis, aibių skirtumas tarp mažesnės
ir didesnės aibės visada bus tuščia aibė, nes mažesnėje aibėje nėra elementų,
kurių nebūtų ir didesnėje aibėje.

\subsubsection{Poaibiai}

Poaibis yra sąvoka apibūdinti tam tikrą santykį tarp aibių.

Pastebėjimas: mokyklos kurse mokomosi poaibių, bet yra naudojamas tikrinio
poaibio ženklas (vietoje $\subset$, turėtų būti naudojamas $\subseteq$), nors
apibrėžimas yra (paprasto) poaibio.

Apibrėžimas - kai aibė $E$ yra sudaryta iš aibės kurių nors aibės $A$ elementū,
vadinama aibės $A$ poaibiu. Rašoma, kad $E \subset A $.

Dar galima ir kitaip apibrėžti poaibį: aibė $A$ yra aibės $B$ poaibis, jei
kiekvienas $A$ elementas yra ir $B$ elementas.

Pagrindiniai poaibų aspektai:

\begin{itemize}
      \item Kiekviena aibė yra poaibis sau pačiai. ($A \subset A$, $B \subset
                  B$ ir t.t.);
      \item Tuščia aibė yra kiekvienos aibės poaibis. ($\varnothing \subset A$,
            $\varnothing \subset B$ ir t.t.);
\end{itemize}

Pasižiūrėkime, kaip sudaromi poaibiai turimos aibės. Tarkime, turime aibę
$A=\{15;20;60\}$. Tai šios aibės visi galimi poaibiai:
\begin{enumerate}
      \item $A$ arba $\{15;20;60\}$ (žiūrėti pagrindinius aspektus aukščiau);
      \item $\varnothing$ (žiūrėti pagrindinius aspektus aukščiau);
      \item $\{15\}$ (tik pirmas aibės elementas);
      \item $\{20\}$ (tik antras aibės elementas);
      \item $\{60\}$ (tik antras aibės elementas);
      \item $\{15;20\}$ (pirmas ir antras aibės elementai);
      \item $\{20;60\}$ (antras ir trečias aibės elementai);
      \item $\{15;60\}$ (pirmas ir trečias aibės elementai);
\end{enumerate}

Taip pat aibę galima atvaizduoti ir veno diagramomis (\ref{fig:set_subsets}
pav.):

\begin{figure}[!htbp]%
      \centering
      \subfloat[\centering aibės $A$ poaibis $B$ ]{{
                        \begin{tikzpicture}
                              % Define circles
                              \def\firstcircle{(0.50,0) circle (.45 cm)}
                              \def\secondcircle{(0,0) circle (2 cm)}

                              % Fill the inner circle A with the defined fill color
                              \fill[circle area] \firstcircle;

                              % Draw the circles and label them with the defined edge color
                              \draw[line width=1pt, circle edge] \firstcircle
                              node[text=black] {$B$};
                              \draw[line width=1pt, circle edge] \secondcircle
                              node[text=black, yshift=1.5cm] {$A$};

                              % Draw the label above the figure
                              \node[anchor=south] at (current bounding
                              box.north) {$B
                                          \subset A$};
                        \end{tikzpicture}
                  }}%
      \qquad
      \subfloat[\centering aibės $D$ poaibis $E$ ]{
            {
                        \begin{tikzpicture}
                              % Define circles
                              \def\firstcircle{(0,0) circle (1 cm)}
                              \def\secondcircle{(0,0) circle (2 cm)}

                              % Fill the inner circle A with the defined fill color
                              \fill[circle area] \firstcircle;

                              % Draw the circles and label them with the defined edge color
                              \draw[line width=1pt, circle edge] \firstcircle
                              node[text=black] {$E$};
                              \draw[line width=1pt, circle edge] \secondcircle
                              node[text=black, yshift=1.5cm] {$D$};

                              % Draw the label above the figure
                              \node[anchor=south] at (current bounding
                              box.north) {$E
                                          \subset D$};
                        \end{tikzpicture}
                  }}%
      \caption{Aibių ir
            poaibiai}\label{fig:set_subsets}%
\end{figure}

Poabiai galimi, kai aibės nurodytos intervalais. Tarkime turime aibe $Z =
      [-500; 2023)$. Tokia aibė turi begalę poaibių, nes pati aibė yra
begalinė. Sudarykime keleta šios aibės poaibių:
\begin{enumerate}
      \item pati aibė (žiūrėti pagrindinius aspektus aukščiau)

            $Z \subset Z$ arba $[-500; 2023) \subset Z$;

      \item tuščia aibė (žiūrėti pagrindinius aspektus aukščiau)

            $ \varnothing \subset Z$ arba $\{\} \subset Z$;

      \item Aibė su vienu elementu, kuris priklauso intervalui

            $\{1998\} \subset Z$;

      \item Aibė su su dviem elementais, kurie priklauso intervalui

            $\{1009;1410\} \subset Z$;

      \item Aibė su su trimis elementais, kurie priklauso intervalui

            $\{3;11; 1990\} \subset Z$;

      \item Aibė su \textit{n} elementų, kurie priklauso intervalui;
      \item Kitas intervalas, kuris yra aibės intervalo ribose

            $[0; 2023] \subset Z$;

      \item Aibių sąjunga, kurių nariai yra aibės intervalo ribose

            $ [0; 1010) \cup (1010; 1990] \subset Z$;

      \item ir taip iki begalybės\ldots

\end{enumerate}

\subsection{Lygtys, nelygybės ir skaičių aibės}

Sprendžiant matematikos užduotis mokykloje aibių tema, dažnai galime sutikti
prašant išskirti specifinę sprendinių aibę. Pavyzdžiui:

\begin{itemize}
      \item \germanqq{Raskite lygties sveikųjų sprendinių aibę};
      \item \germanqq{Raskite lygties natūraliųjų sprendinių aibę};
      \item \germanqq{Raskite nelygybės natūraliuosius sprendinius};
      \item \germanqq{Raskite nelygybės neigiamus sveikuosius sprendinius};
      \item ir t.t.
\end{itemize}

Tokiems uždaviniams išspręsti reikalingas supratimas apie \textbf{lygtis},
\textbf{nelygybes}, \textbf{skaičių aibes}.

\subsection{Pavyzdys \#1}

Turime užduotį: raskite $x^2=16$ lygties natūraliuosius sprendinius. Tokio tipo
lygčių sprendimą galite rasti \ref{sec:ax_square_equal_number} skyriuje. Tokios
lygties sprendiniai yra
\[x=\{-4;4\};\]

Užduotis prašom natūraliųjų sprendinių, o -4 nėra natūralusis ($-4 \notin
      \mathbb{N} $), tai atsakymas:
\[\text{Ats.:}\{4\};\]

\subsection{Pavyzdys \#2}

Turime užduotį: raskite $-5 \leq 2x + 3$ nelygybės \textbf{neigiamus}
sveikuosius sprendinius. Tokio tipo
lygčių sprendimą galite rasti \ref{sec:ax_minus_b_inequality} skyriuje.
Išsprendus gauname, kad
\[x\geq-4\]
arba
\[x=[-4;+\infty)\]
Į pradinę nelygybę įdėjus skaičius (pagal gautą rezultatą) -4; -3; -2,5; 0;
10,1; 100 ir t.t. (bet
kokius skaičius didesnius arba lygius -4), gauname, kad gautas nelygybės
rezultatas geras. Pagal užduotį šis atsakymas netinkamas, nes prašoma neigiamų
sveikųjų skaičių. Pagal intervalą, galime lengvai išrašyti neigiamus
sveikuosius skaičius: -4 (nes laužtinis skliaustas); -3; -2; -1. Tai atsakymas
yra
\[\text{Ats.:} \{-4; -3 ;-2; -1\};\]
Sprendžiant tokį užduotį svarbu atsakyme neparašyti intervalo nuo -4,
įskaitant, iki 0, neįskaitant: $[-4;0)$. Kadangi skaičiai į tokį intervalą
įeina ir skaičiai $-3,6; 2,55;-\sqrt{2}$ ir t.t., kurie nėra sveikieji.

Įmanoma atsakymą pateikti ir kitaip. Žinant aibių aprašymo taisykles, galima
užrašyti ir jomis:
\begin{equation}
      \{x \in \mathbb{Z}, -4 \leq x < 0\};
\end{equation}
arba
\begin{equation}
      \{x \in \mathbb{Z}, -4 \leq x < \infty, x < 0\};
\end{equation}

Toks užrašymas (1 formulė) reiškia, kad aibė sudaryta iš $x$, kurie priklauso
sveikųjų skaičių aibei ir yra tarp -4, įskaitant,
ir 0, neįskaitant. Antru aveju (2 formulė), reiškia, kad aibė sudaryta iš
elementų $x$, kurie priklauso sveikųjų skaičių aibei, yra didesni arba lygūs -4
ir neigiami. Tokį užrašymą, pagal matematikos programą, žinoti reiktų.

\section{Iracionalumo pašalinimas vardiklyje}

Pagal atnaujintas matematikos programas, jums reikia žinoiti, kaip trupmenos
vardiklyje panaikinti iracionalumą, kai vardiklyje yra iracionalieji skaičiai:
$\sqrt{a}$, $\sqrt{a}+b$, $\sqrt{a}-b$.

\subsection{Paprastas atvejis 1: šaknis vardiklyje, skaičius skaitiklyje}

Jeigu trupmena atrodo taip - $\frac{a}{\sqrt{b}}$, kur $a$ ir $b$ yra skaičiai
ir $\sqrt{b}$ yra iracionali kvadratinė šaknis, galima vardiklį ir skaitiklį
padauginti iš vardiklio. Tokiu būdu bus panaikinta šaknis vardiklyje:
\[\frac{a}{\sqrt{b}}=\frac{a}{\sqrt{b}}\cdot\frac{\sqrt{b}}{\sqrt{b}}=\frac{a\sqrt{b}}{b};\]
Prisiminkite, kad sudauginus dvi vienodas šaknis gaunamas skaičius po šaknimi:
\[ \pmb{\sqrt{b}\cdot\sqrt{b}}=b^{\frac{1}{2}}\cdot
      b^{\frac{1}{2}}=b^{\frac{1}{2}+\frac{1}{2}}=\pmb{b}; \]

\textbf{Pavyzdys}: reikia pašalinti iracionalumą iš
\[ \frac{3}{\sqrt{2}} \]
tai padauginame skaitiklį ir vardiklį iš šaknies
\[ \frac{3}{\sqrt{2}} \cdot \frac{\sqrt{2}}{\sqrt{2}}= \frac{3\sqrt{2}}{2}; \]

\subsection{Atvejis 2: šaknis vardiklyje, reiškinys skaitiklyje}

Jeigu trupmena atrodo taip - $\frac{ax-c}{\sqrt{b}}$, kur $ax-c$
\textbf{reiškinys}, o $b$ yra skaičius
ir $\sqrt{b}$ yra iracionali kvadratinė šaknis, tai, kaip ir pirmu atveju,
galima vardiklį ir skaitiklį
padauginti iš vardiklio. Tokiu būdu bus panaikinta šaknis vardiklyje:
\[\frac{ax-c}{\sqrt{b}}=\frac{ax-c}{\sqrt{b}}\cdot\frac{\sqrt{b}}{\sqrt{b}}=\frac{(ax-c)\sqrt{b}}{b}=\frac{ax\sqrt{b}-c\sqrt{b}}{b};\]

\textbf{Būtina dauginti visą skaitiklį} iš vardiklio! Taip padaroma
apskliaudžiant skaitklį, o po to atskliaudžiant sudauginamas kiekvienas narys.

\textbf{Pavyzdys}: reikia pašalinti iracionalumą iš
\[ \frac{3x+5}{\sqrt{2}} \]
tai padauginame skaitiklį ir vardiklį iš šaknies
\[ \frac{3x+5}{\sqrt{2}} \cdot \frac{\sqrt{2}}{\sqrt{2}}=
      \frac{(3x+5)\sqrt{2}}{2} = \frac{3x\sqrt{2}+5\sqrt{2}}{2}; \]

Jeigu matome, kad toliau suprastinti trupmenos neišeis, ją galima palikti ir su
skliaustais vardiklyje.

\subsection{Atvejis 3: binominis reiškinys vardiklyje}

Jeigu trupmena atrodo taip - $\frac{a}{b+\sqrt{c}}$, kur $a$, $b$, $c$ skaičiai
ir $\sqrt{b}$ yra iracionali kvadratinė šaknis. Šiuo atveju naikinant
iracionalumą, reikia vardiklį ir skaitiklį padauginti \textbf{iš viso}
vardiklio, bet \textbf{su skirtingu ženklu tarp narių}. Šiuo atveju, tarp $b$
ir $\sqrt{c}$ yra \germanqq{+}, tai reikia padauginti iš $(b-\sqrt{c})$
\[\frac{a}{b+\sqrt{c}}=\frac{a}{(b+\sqrt{c})}\cdot\frac{(b-\sqrt{c})}{(b-\sqrt{c})}=\frac{a(b-\sqrt{c})}{(b+\sqrt{c})(b-\sqrt{c})};
\]

\textbf{Būtina dauginti visą skaitiklį ir vardiklį} iš vardiklio! Reikia
apskliausti!

Toliau galima panaudoti greitosios daugybos formulę:
\[(a-b)(a+b)=a^{2}-b^{2};\]
\[\frac{a(b-\sqrt{c})}{(b+\sqrt{c})(b-\sqrt{c})} =
      \frac{a(b-\sqrt{c})}{b^{2}-{(\sqrt{c})}^{2}} =
      \frac{a(b-\sqrt{c})}{b^{2}-c} ;
\]

Toliau pagal situaciją sutraukiami, suprastinami panašūs nariai. Tas pats
galioja ir situacijai, kai skaitiklyje yra dviejų narių atimtis
-$\frac{a}{b-\sqrt{c}}$. Tokiu atveju skaitiklis ir vardiklis dauginami iš
vardiklio pakeitus \germanqq{-} į \germanqq{+}:
$$ (b-\sqrt{c}) \rightarrow (b+\sqrt{c}); $$

\textbf{Pavyzdys} 1: reikia pašalinti iracionalumą iš
\[ \frac{16}{1-\sqrt{5}}; \]
tai padauginame skaitiklį ir vardiklį iš vardiklio pakeitus ženklą (šiuo atveju
\germanqq{-} pakeičiam į \germanqq{+})
$$\frac{16}{(1-\sqrt{5})}\cdot\frac{(1+\sqrt{5})}{(1+\sqrt{5})}=\frac{16(1+\sqrt{5})}{(1-\sqrt{5})(1+\sqrt{5})}=\frac{16(1+\sqrt{5})}{1^{2}-{(\sqrt{5})}^{2}}=\frac{16(1+\sqrt{5})}{1-5}=\frac{16(1+\sqrt{5})}{-4};$$

Toliau dar galima suprastinti vardiklį ir skaitiklį (prisiminkite, kad
prastinti galima, kai yra daugyba tarp narių).
\[\frac{16(1+\sqrt{5})}{-4} = -4(1+\sqrt{5}) = -4-4\sqrt{5}; \]

\textbf{Pavyzdys} 2: reikia pašalinti iracionalumą iš
\[ \frac{3-\sqrt{6}}{\sqrt{3}+\sqrt{2}}; \]
Galioja visos tos pačios taisyklės, kaip ir anksčiau. Padauginame \textbf{visą}
skaitiklį ir \textbf{visą} vardiklį iš \textbf{viso} vardiklio, bet su pakeistu
ženklu tarp narių (šiuo atveju pakeičiam \germanqq{-} į \germanqq{+})
$$\frac{3-\sqrt{6}}{\sqrt{3}+\sqrt{2}}=\frac{(3-\sqrt{6})}{(\sqrt{3}+\sqrt{2})}\cdot\frac{(\sqrt{3}-\sqrt{2})}{(\sqrt{3}-\sqrt{2})}=\frac{(3-\sqrt{6})(\sqrt{3}-\sqrt{2})}{(\sqrt{3}+\sqrt{2})(\sqrt{3}-\sqrt{2})}$$

Toliau sudauginame kiekvieną narį skliaustiuose su kitų skliaustių kiekvieniu nariu:
$$\frac{(3-\sqrt{6})(\sqrt{3}-\sqrt{2})}{(\sqrt{3}+\sqrt{2})(\sqrt{3}-\sqrt{2})}=\frac{3\sqrt{3}-3\sqrt{2}-\sqrt{6}\sqrt{3}+\sqrt{2}\sqrt{6}}{{(\sqrt{3})}^2-{(\sqrt{2})}^2};$$

Dar galime pakelti kvadratu, sutvarkyti panašius narius (su vienodais skaičiais šaknyse), sujungti vienodo laipsnio šaknis:
$$\frac{3\sqrt{3}-3\sqrt{2}-\sqrt{6}\sqrt{3}+\sqrt{2}\sqrt{6}}{{(\sqrt{3})}^2-{(\sqrt{2})}^2}=\frac{3\sqrt{3}-3\sqrt{2}-\sqrt{6 \cdot 3}+\sqrt{2 \cdot 6}}{3-2}=3\sqrt{3}-3\sqrt{2}-\sqrt{18}+\sqrt{12};$$

Iš šaknų galime dalinai ištraukti šaknį, išskaidant skaičius dauginamaisiais:

$$3\sqrt{3}-3\sqrt{2}-\sqrt{18}+\sqrt{12}=3\sqrt{3}-3\sqrt{2}-\sqrt{9 \cdot 2}+\sqrt{4 \cdot 3}=3\sqrt{3}-3\sqrt{2}-\sqrt{9}\sqrt{2}+\sqrt{4}\sqrt{3};$$
$$ 3\sqrt{3}-3\sqrt{2}-\sqrt{9}\sqrt{2}+\sqrt{4}\sqrt{3}=3\sqrt{3}-3\sqrt{2}-3\sqrt{2}+2\sqrt{3};$$ 

Dabar jau turime panašiųjų narių, kuriuos galime sutvarkyti
$$ 3\sqrt{3}-3\sqrt{2}-3\sqrt{2}+2\sqrt{3}=5\sqrt{3}-6\sqrt{2};$$ 

\clearpage

% \bibliographystyle{plain}
% \bibliography{bibliography.bib}
\end{document}